\begin{enumerate}[label=\thechapter.\arabic*,ref=\thechapter.\theenumi]
\item Let $\phi(.)$ denote the cumulative distribution function of a standard normal
random variable. If the random variable $X$ has the cumulative distribution
function 
\begin{align}
	F(x)&= 
    \begin{cases}
        \phi(x), &  x < -1 \\
        \phi(x+1) , &  x \ge -1
    \end{cases} 
\end{align}
then which one of the following statements is true?
\begin{enumerate}
\item $P(X \leq -1) = \frac{1}{2}$
\item $P(X = -1) = \frac{1}{2}$
\item $P(X < -1) = \frac{1}{2}$
\item $P(X \leq 0) = \frac{1}{2}$
\end{enumerate}
\hfill(GATE ST 2023)
\\

\let\negmedspace\undefined
\let\negthickspace\undefined
\documentclass[article]{IEEEtran}
       \def\inputGnumericTable{}                                 %%
\usepackage{cite}
\usepackage{amsmath,amssymb,amsfonts,amsthm}
\usepackage{algorithmic}
\usepackage{graphicx}
\usepackage{textcomp}
\usepackage{xcolor}
\usepackage{txfonts}
\usepackage{listings}
\usepackage{enumitem}
\usepackage{mathtools}
\usepackage{gensymb}
\usepackage[breaklinks=true]{hyperref}
\usepackage{tkz-euclide} % loads  TikZ and tkz-base
\usepackage{listings}
\renewcommand{\theenumi}{\Alph{enumi}}
%
%\usepackage{setspace}
%\usepackage{gensymb}
%\doublespacing
%\singlespacing

%\usepackage{graphicx}
%\usepackage{amssymb}
%\usepackage{relsize}
%\usepackage[cmex10]{amsmath}
%\usepackage{amsthm}
%\interdisplaylinepenalty=2500
%\savesymbol{iint}
%\usepackage{txfonts}
%\restoresymbol{TXF}{iint}
%\usepackage{wasysym}
%\usepackage{amsthm}
%\usepackage{iithtlc}
%\usepackage{mathrsfs}
%\usepackage{txfonts}
%\usepackage{stfloats}
%\usepackage{bm}
%\usepackage{cite}
%\usepackage{cases}
%\usepackage{subfig}
%\usepackage{xtab}
%\usepackage{longtable}
%\usepackage{multirow}
%\usepackage{algorithm}
%\usepackage{algpseudocode}
%\usepackage{enumitem}
%\usepackage{mathtools}
%\usepackage{tikz}
%\usepackage{circuitikz}
%\usepackage{verbatim}
%\usepackage{tfrupee}
%\usepackage{stmaryrd}
%\usetkzobj{all}
    \usepackage{color}                                            %%
    \usepackage{array}                                            %%
    \usepackage{longtable}                                        %%
    \usepackage{calc}                                             %%
    \usepackage{multirow}                                         %%
    \usepackage{hhline}                                           %%
    \usepackage{ifthen}                                           %%
 %optionally (for landscape tables embedded in another document): %%
    \usepackage{lscape}     
%\usepackage{multicol}
%\usepackage{chngcntr}
%\usepackage{enumerate}

%\usepackage{wasysym}
%\documentclass[conference]{IEEEtran}
%\IEEEoverridecommandlockouts
% The preceding line is only needed to identify funding in the first footnote. If that is unneeded, please comment it out.

\newtheorem{theorem}{Theorem}[section]
\newtheorem{problem}{Problem}
\newtheorem{proposition}{Proposition}[section]
\newtheorem{lemma}{Lemma}[section]
\newtheorem{corollary}[theorem]{Corollary}
\newtheorem{example}{Example}[section]
\newtheorem{definition}[problem]{Definition}
%\newtheorem{thm}{Theorem}[section] 
%\newtheorem{defn}[thm]{Definition}
%\newtheorem{algorithm}{Algorithm}[section]
%\newtheorem{cor}{Corollary}
\newcommand{\BEQA}{\begin{eqnarray}}
\newcommand{\EEQA}{\end{eqnarray}}
\newcommand{\define}{\stackrel{\triangle}{=}}
\theoremstyle{remark}
\newtheorem{rem}{Remark}

\lstset{
    language=Python,   % Set the programming language for syntax highlighting
    basicstyle=\ttfamily, % Set the font style for the code
    keywordstyle=\color{blue}, % Customize keywords
    commentstyle=\color{green}, % Customize comments
    stringstyle=\color{red},   % Customize strings
    numbers=left,      % Display line numbers
    numberstyle=\tiny, % Set the style for line numbers
    breaklines=true,   % Automatically wrap long lines
}
\lstset{
    language=C,               % Set the programming language
    basicstyle=\ttfamily,     % Font style for the code
    keywordstyle=\color{blue}, % Keyword style
    commentstyle=\color{green},% Comment style
    stringstyle=\color{red},   % String style
    numbers=left,             % Display line numbers
    numberstyle=\tiny,        % Style for line numbers
    breaklines=true,           % Automatically wrap long lines
    frame=single,             % Add a frame around the code
    showstringspaces=false,    % Don't show spaces within strings
    tabsize=4,                % Set tab size to 4 spaces
    morekeywords={printf, scanf, int, main, if, else, while}, % Additional keywords
    extendedchars=true,       % Allow extended characters like underscores
    literate={~} {$\sim$}{1}, % Replace ~ with a tilde symbol
    backgroundcolor=\color{gray!10}, % Background color for the code
    escapeinside={(*@}{@*)},  % Define an escape sequence for LaTeX code within comments
}

\begin{document}
\providecommand{\pr}[1]{\ensuremath{\Pr\left(#1\right)}}
\providecommand{\prt}[2]{\ensuremath{p_{#1}^{\left(#2\right)} }}        % own macro for this question
\providecommand{\qfunc}[1]{\ensuremath{Q\left(#1\right)}}
\providecommand{\sbrak}[1]{\ensuremath{{}\left[#1\right]}}
\providecommand{\lsbrak}[1]{\ensuremath{{}\left[#1\right.}}
\providecommand{\rsbrak}[1]{\ensuremath{{}\left.#1\right]}}
\providecommand{\brak}[1]{\ensuremath{\left(#1\right)}}
\providecommand{\lbrak}[1]{\ensuremath{\left(#1\right.}}
\providecommand{\rbrak}[1]{\ensuremath{\left.#1\right)}}
\providecommand{\cbrak}[1]{\ensuremath{\left\{#1\right\}}}
\providecommand{\lcbrak}[1]{\ensuremath{\left\{#1\right.}}
\providecommand{\rcbrak}[1]{\ensuremath{\left.#1\right\}}}
\newcommand{\sgn}{\mathop{\mathrm{sgn}}}
\providecommand{\abs}[1]{\left\vert#1\right\vert}
\providecommand{\res}[1]{\Res\displaylimits_{#1}} 
\providecommand{\norm}[1]{\left\lVert#1\right\rVert}
%\providecommand{\norm}[1]{\lVert#1\rVert}
\providecommand{\mtx}[1]{\mathbf{#1}}
\providecommand{\mean}[1]{E\left[ #1 \right]}
\providecommand{\cond}[2]{#1\middle|#2}
\providecommand{\fourier}{\overset{\mathcal{F}}{ \rightleftharpoons}}
\newenvironment{amatrix}[1]{%
  \left(\begin{array}{@{}*{#1}{c}|c@{}}
}{%
  \end{array}\right)
}
%\providecommand{\hilbert}{\overset{\mathcal{H}}{ \rightleftharpoons}}
%\providecommand{\system}{\overset{\mathcal{H}}{ \longleftrightarrow}}
	%\newcommand{\solution}[2]{\textbf{Solution:}{#1}}
\newcommand{\solution}{\noindent \textbf{Solution: }}
\newcommand{\cosec}{\,\text{cosec}\,}
\providecommand{\dec}[2]{\ensuremath{\overset{#1}{\underset{#2}{\gtrless}}}}
\newcommand{\myvec}[1]{\ensuremath{\begin{pmatrix}#1\end{pmatrix}}}
\newcommand{\mydet}[1]{\ensuremath{\begin{vmatrix}#1\end{vmatrix}}}
\newcommand{\myaugvec}[2]{\ensuremath{\begin{amatrix}{#1}#2\end{amatrix}}}
\providecommand{\rank}{\text{rank}}
\providecommand{\pr}[1]{\ensuremath{\Pr\left(#1\right)}}
\providecommand{\qfunc}[1]{\ensuremath{Q\left(#1\right)}}
	\newcommand*{\permcomb}[4][0mu]{{{}^{#3}\mkern#1#2_{#4}}}
\newcommand*{\perm}[1][-3mu]{\permcomb[#1]{P}}
\newcommand*{\comb}[1][-1mu]{\permcomb[#1]{C}}
\providecommand{\qfunc}[1]{\ensuremath{Q\left(#1\right)}}
\providecommand{\gauss}[2]{\mathcal{N}\ensuremath{\left(#1,#2\right)}}
\providecommand{\diff}[2]{\ensuremath{\frac{d{#1}}{d{#2}}}}
\providecommand{\myceil}[1]{\left \lceil #1 \right \rceil }
\newcommand\figref{Fig.~\ref}
\newcommand\tabref{Table~\ref}
\newcommand{\sinc}{\,\text{sinc}\,}
\newcommand{\rect}{\,\text{rect}\,}
%%
%	%\newcommand{\solution}[2]{\textbf{Solution:}{#1}}
%\newcommand{\solution}{\noindent \textbf{Solution: }}
%\newcommand{\cosec}{\,\text{cosec}\,}
%\numberwithin{equation}{section}
%\numberwithin{equation}{subsection}
%\numberwithin{problem}{section}
%\numberwithin{definition}{section}
%\makeatletter
%\@addtoreset{figure}{problem}
%\makeatother

%\let\StandardTheFigure\thefigure
\let\vec\mathbf

\bibliographystyle{IEEEtran}
\title{
%	\logo{
Assignment
%	}
}
\author{ Karthikeya hanu prakash kanithi (EE22BTECH11026)}
\maketitle
\parindent0px
\vspace{3cm}
Question : Let $\phi(.)$ denote the cumulative distribution function of a standard normal
random variable. If the random variable $X$ has the cumulative distribution
function 
\begin{align}
	F(x)&= 
    \begin{cases}
        \phi(x), &  x < -1 \\
        \phi(x+1) , &  x \ge -1
    \end{cases} \label{eq:15st/2022}
\end{align}
then which one of the following statements is true?
\begin{enumerate}
\item $P(X \leq -1) = \frac{1}{2}$
\item $P(X = -1) = \frac{1}{2}$
\item $P(X < -1) = \frac{1}{2}$
\item $P(X \leq 0) = \frac{1}{2}$
\end{enumerate}
\solution 
\textbf{Gaussian}\\
Q function is defined
\begin{align}
	\qfunc{x} &= \frac{1}{\sqrt{2\pi}}\int_{x}^{\infty} e^{\frac{-u^2}{2}} du \label{eq:15st/2022/1}
\end{align}
From \eqref{eq:15st/2022} and \eqref{eq:15st/2022/1};
\begin{align}
	F_X(x) &= 
	\begin{cases}
        \qfunc{-x} , &  x < -1 \\
        1-\qfunc{x+1} , &  x \ge -1
        \end{cases} \label{eq:15st/2022/3}
\end{align}
From \eqref{eq:15st/2022/3};
\begin{enumerate}
\item \begin{align}
	\pr{X \le -1} = F_X(-1) &= 1 - \qfunc{0} \\
	&= 0.5
\end{align}
So Option A i.e., $P(X < -1) = \frac{1}{2}$ is correct
\item The pdf of X can be defined in terms of cdf as 
\begin{align}
	\pr{X = b} = F_X(b) - \lim_{x \to b^-}F_X(x) \label{eq:15st/2022/3}
\end{align}
From \eqref{eq:15st/2022/3};
\begin{align}
	\pr{X = -1} &= F_X(-1) - \lim_{x \to -1^-}F_X(x)\\
	&= 1-\qfunc{0} - \qfunc{-(-1)}\\
	&= 0.341
\end{align} 
So Option B i.e., $P(X = -1) = \frac{1}{2}$ is incorrect
\item  \begin{align}
	\pr{X < -1} = \lim_{x \to -1^-}F_X(x)&= F_X(-1) \\
	&= \qfunc{-(-1)}\\
	&= 0.159
\end{align}
So Option C i.e., $P(X < -1) = \frac{1}{2}$ is incorrect
\item \begin{align}
	\pr{X \leq 0} = F_X(0)	&= 1 - \qfunc{1} \\
	&= 0.8413
\end{align}
So Option D i.e., $P(X \leq 0) = \frac{1}{2}$ is incorrect
\end{enumerate}
\newpage
Guassian CDF plot of X is given in fig\ref{fig:15st/2022} \\
\begin{figure}[ht!]
    \centering
    \includegraphics[width=\columnwidth]{/home/karthikeya/EE23010/15/figs/figure.png}
    \caption{}
    \label{fig:15st/2022}
\end{figure}
\newpage
\textbf{Box-Muller}\\
\textbf{STATEMENT}:
  Suppose U1 and U2 are independent samples chosen from the uniform distribution on the unit interval (0, 1). Let
  \begin{align}
  	Z_0 = R \cos(\Theta) =\sqrt{-2 \ln U_1} \cos(2 \pi U_2)\,
  \end{align}
   and
  \begin{align}
  	Z_1 = R \sin(\Theta) =\sqrt{-2 \ln U_1} \cos(2 \pi U_2)\, 
  \end{align}
  Then $Z_0$ and $Z_1$ are independent random variables with a standard normal distribution.
  So, now we will generate $Z_0$ using C Code as given below \\
\textbf{PROOF}:
  Let $X$ and $Y$ be independent standard normal variables 
  \begin{align}
  	X,Y \sim \mathcal{N}(0,1) \quad and \quad X\perp Y
  \end{align}
  The joint pdf of $X$ and $Y$ is given by 
  \begin{align}
  	f_{XY}(X,Y) &= f(x)f(y)\\
  	&= \frac{1}{\sqrt{2\pi}}e^{\frac{-x^2}{2}}.\frac{1}{\sqrt{2\pi}}e^{\frac{-y^2}{2}}\\
  	&= \frac{1}{2\pi}e^{\frac{-(x^2+y^2)}{2}}
  \end{align}
  The relationship between Cartesian coordinates $(x,y)$ and polar coordinates $(r,\theta)$ is as follows 
  \begin{align}
  	x=r\cos\theta \\
  	y=r\sin\theta
  \end{align}
  Change $f_{XY}(x,y)$ to polar coordinates :
  \begin{align}
  	f_{XY}(x,y) dxdy = f_{R\theta}(r,\theta) drd\theta
  \end{align}
  i.e.,
  \begin{align}
  	f_{R\theta}(r,\theta) = f_{XY}(x,y) \frac{dxdy}{drd\theta}
  	= f_{XY}(x,y) \mydet{\frac{\partial(x,y)}{\partial(r,\theta)}}
  \end{align}
  where J is the Jacobian 
  \begin{align}
  	J &= \mydet{\frac{\partial x}{\partial r} \,\, \frac{\partial x}{\partial \theta}\\\frac{\partial y}{\partial r} \,\, \frac{\partial y}{\partial \theta}}\\
  	&= \mydet{\cos\theta \,\, -r\sin\theta \\ \sin\theta \,\, r\cos\theta} = r
  \end{align}
  For $r\ge$ 0 and $\theta \in [0,2\pi)$, we have 
  \begin{align}
  	f_{R\theta}(r,\theta) drd\theta = \frac{1}{2\pi}e^{\frac{-(r^2)}{2}}rdrd\theta
  \end{align}
  Now we change the variable from $(r,\theta)$ to $(r^2,\theta)$;
  \begin{align}
  	rdr = \frac{1}{2}dr^2
  \end{align}
  Now it can be written as, 
  \begin{align}
  	f_{R\theta}(r,\theta)drd\theta &= f_{R^2\theta}(r^2,\theta)dr^2d\theta\\
  	&= \frac{1}{2\pi}e^{\frac{-(r^2)}{2}}\frac{1}{2}dr^2d\theta\\
  	&= \brak{\frac{1}{2}e^{\frac{-(r^2)}{2}}dr^2}\brak{\frac{1}{2}d\theta}\\
  	&= f_{R^2}(r^2)dr^2 f_{\theta}(\theta)d\theta
  \end{align}
  from the above equation, we can say that 
  \begin{align}
        R^2\perp\theta \,\, (i.e., R^2 \text{ and } \theta \text{ are independent})
  \end{align}
  Generate, $\theta \sim Unif(0,2\pi)$ 
  Generate, $V \sim Exp(\lambda = \frac{1}{2})$ (i.e., $V = R^2$) and compute 
  \begin{align}
        R=\sqrt{V}
  \end{align}
  Compute 
  \begin{align}
        X = R\cos\theta\\
        Y = R\sin\theta\\
  \end{align}
  where, $X,Y$ are i.i.d in $\sim \mathcal{N}(0,1)$
  Then  
  \begin{align}
        \theta &= 2\pi U_1\\
        V &= -2log(U_2)
  \end{align}
  We can prove $V = -2log(U_2)$ using the c.d.f definition of the exponential distribution\\
  Let $X$ be a random variable following an exponential distribution with rate parameter $\lambda = \frac{1}{2}$, denoted as $X \sim \text{Exp}\brak{\frac{1}{2}}$. The cumulative distribution function (CDF) of the exponential distribution is given by:
\begin{align}
F(x) &= 1 - e^{-\frac{x}{2}}
\end{align}
Now, suppose we have a random variable $U$ following a uniform distribution in the interval $[0, 1]$, denoted as $U \sim \text{U}(0, 1)$. The CDF of the uniform distribution is simply:
\begin{align}
F_U(u) &= u, \text{ for } 0 \leq u \leq 1
\end{align}
We can use the probability integral transform to express the exponential random variable $X$ in terms of the uniform random variable $U$:
\begin{align}
F(x) &= F_U(u) \\
1 - e^{-\frac{x}{2}} &= u
\end{align}
Now, solve for $x$:
\begin{align}
e^{-\frac{x}{2}} &= 1 - u \\
-\frac{x}{2} &= \ln(1 - u) \\
x &= -2 \ln(1 - u)
\end{align}
So, $X \sim -2 \ln(1 - U)$ for $U \sim \text{U}(0, 1)$. This expression represents the exponential random variable $X$ in terms of a uniform random variable $U$.\\
C and Python codes are given below :
  \lstinputlisting[language=C]{/home/karthikeya/EE23010/15/codes/rand.c}
  \lstinputlisting[language=Python]{/home/karthikeya/EE23010/15/codes/rand.py}
\begin{figure}[!ht]
  \centering
  \includegraphics[width=\columnwidth]{/home/karthikeya/EE23010/15/figs/figure1.png}  % Replace 'image_filename' with your image file name
  \caption{Histogram plot of density of $Z_0$}
  \label{fig:your_label}
\end{figure}
\end{document}





















\item Let $X$ be a random variable with the probability density function $f(x)$ such that
\begin{align}
f(x) &= 
	\begin{cases}
		\frac{1}{2\sqrt{3}}, & -\sqrt{3} \leq x \leq \sqrt{3} \\
		0, & \text{otherwise}
	\end{cases}
\end{align}
Then the variance of $X$ is?
\hfill(GATE XH-C1 2023)
\\
\iffalse
\let\negmedspace\undefined
\let\negthickspace\undefined
\documentclass[journal,12pt,twocolumn]{IEEEtran}
\usepackage{cite}
\usepackage{amsmath,amssymb,amsfonts,amsthm}
\usepackage{algorithmic}
\usepackage{graphicx}
\usepackage{textcomp}
\usepackage{xcolor}
\usepackage{txfonts}
\usepackage{listings}
%\usepackage{enumitem}
\usepackage{mathtools}
\usepackage{gensymb}
\usepackage[breaklinks=true]{hyperref}
\usepackage{tkz-euclide} % loads  TikZ and tkz-base
\usepackage{listings}
\usepackage[inline]{enumitem}
\DeclareMathOperator*{\Res}{Res}
\renewcommand\thesection{\arabic{section}}
\renewcommand\thesubsection{\thesection.\arabic{subsection}}
\renewcommand\thesubsubsection{\thesubsection.\arabic{subsubsection}}


\def\inputGnumericTable{}

\usepackage[latin1]{inputenc}                                 
\usepackage{color}                                            
\usepackage{array}                                            
\usepackage{longtable}                                        
\usepackage{calc}                                             
\usepackage{multirow}                                         
\usepackage{hhline}                                           
\usepackage{ifthen}
\usepackage{caption} 
\captionsetup[table]{skip=3pt}  
\providecommand{\pr}[1]{\ensuremath{\Pr\left(#1\right)}}
\providecommand{\cbrak}[1]{\ensuremath{\left\{#1\right\}}}

\renewcommand\thesectiondis{\arabic{section}}
\renewcommand\thesubsectiondis{\thesectiondis.\arabic{subsection}}
\renewcommand\thesubsubsectiondis{\thesubsectiondis.\arabic{subsubsection}}

\def\inputGnumericTable{}                                 %%

\lstset{
frame=single, 
breaklines=true,
columns=fullflexible
}

\begin{document}

\newtheorem{theorem}{Theorem}[section]
\newtheorem{problem}{Problem}
\newtheorem{proposition}{Proposition}[section]
\newtheorem{lemma}{Lemma}[section]
\newtheorem{corollary}[theorem]{Corollary}
\newtheorem{example}{Example}[section]
\newtheorem{definition}[problem]{Definition}
\newcommand{\BEQA}{\begin{eqnarray}}
\newcommand{\EEQA}{\end{eqnarray}}
\newcommand{\define}{\stackrel{\triangle}{=}}
\newcommand{\xor}{\oplus}
\bibliographystyle{IEEEtran}

\providecommand{\mbf}{\mathbf}
\providecommand{\pr}[1]{\ensuremath{\Pr\left(#1\right)}}
\providecommand{\qfunc}[1]{\ensuremath{Q\left(#1\right)}}
\providecommand{\sbrak}[1]{\ensuremath{{}\left[#1\right]}}
\providecommand{\lsbrak}[1]{\ensuremath{{}\left[#1\right.}}
\providecommand{\rsbrak}[1]{\ensuremath{{}\left.#1\right]}}
\providecommand{\brak}[1]{\ensuremath{\left(#1\right)}}
\providecommand{\lbrak}[1]{\ensuremath{\left(#1\right.}}
\providecommand{\rbrak}[1]{\ensuremath{\left.#1\right)}}
\providecommand{\cbrak}[1]{\ensuremath{\left\{#1\right\}}}
\providecommand{\lcbrak}[1]{\ensuremath{\left\{#1\right.}}
\providecommand{\rcbrak}[1]{\ensuremath{\left.#1\right\}}}
\theoremstyle{remark}
\newtheorem{rem}{Remark}
\newcommand{\sgn}{\mathop{\mathrm{sgn}}}

\newcommand{\solution}{\noindent \textbf{Solution: }}
\newcommand{\cosec}{\,\text{cosec}\,}
\providecommand{\dec}[2]{\ensuremath{\overset{#1}{\underset{#2}{\gtrless}}}}
\newcommand{\myvec}[1]{\ensuremath{\begin{pmatrix}#1\end{pmatrix}}}
\newcommand{\mydet}[1]{\ensuremath{\begin{vmatrix}#1\end{vmatrix}}}

\let\vec\mathbf


\vspace{3cm}

\title{
  

  Assignment -5 in \LaTeX
    
  }
  \author{ Muzaan Mohammed Faizel A P\\
  EE22BTECH11036
  }	
% make the title area
\maketitle
\newpage
\bigskip
\renewcommand{\thefigure}{\theenumi}
\renewcommand{\thetable}{\theenumi}
\renewcommand{\thetable}{\arabic{table}} 
\textbf{GATE 2023 BM QN.12}
For a Binomial random variable $X$, E($X$) and Var($X$) are the expectation and
variance, respectively. Which one of the following statements CANNOT be true?
\begin{table}[ht!]
		\centering
		\input{2023/BM/12/tables/table1.tex}
		\caption{}
		\label{table:table1}	
\end{table}
\fi
\solution
\begin{align}
X \sim \text{Bin}\brak{n,p} \nonumber
\end{align}

We know ,
\begin{align} 
	E\brak{X}=np\\
	Var\brak{X}=np\brak{1-p}\\
	0\leq p\leq1 \\
	\implies -1\leq -p \leq0\\
	\implies
	0 \leq 1-p \leq1\\
	\implies np\brak{1-p}\leq np
\end{align}
Therefore,
\begin{align} 
	Var\brak{X}\leq E\brak{X}
\end{align}
From the four options,the statement that cannot be true is option \brak{3}\\

\textbf{Simulation steps}\\
\textbf{Step 1: Generate a Range of Probabilities}\\
The program generates a range of probabilities (\(p\)) in increments of 0.0098, ranging from 0.01 to 0.99. This range is used to create different binomial distributions for subsequent calculations.
\\
% It then calculates the variance of the generated binomial random variable.\\
\textbf{Step 2: Generating binomial r.v from Uniform distribution}\\
\( U \sim \text{Uniform}(0, 1) \).

Defining a random variable X as:
\[
X = 
\begin{cases}
    1 & \text{if } U > p \\
    0 & \text{otherwise}
\end{cases}
\]
This becomes a Bernoulli rv.
The count variable evaluates the Binomial r.v by the summation of Bernoulli r.v

\[
count= \sum_{i=1}^{n} X
\]
\textbf{Step 3: Calculate Variances for Each Probability}\\
For each probability (\(p\)) in the generated range, the program generates a binomial random variable using the given mean and the inverse of the probability (\(mean/p\)) from uniform distribution.\\
\textbf{Step 4: Find Maximum and Minimum Variances}\\
After calculating variances for each probability, the program identifies the maximum and minimum variances in the generated set.\\
\textbf{Step 5: User Input}\\
The program prompts the user to input a variance value for validation.
\\
\textbf{Step 6: Check Validity}\\
The user-input variance is compared against the computed maximum and minimum variances. If the input variance falls within this range (inclusive of the minimum and exclusive of the maximum), the program outputs "Valid." Otherwise, it outputs "Invalid."
\begin{figure}[ht!]
    \centering
    \includegraphics[width=\columnwidth]{2023/BM/12/codes/ss10.png}
    \caption{Variance for mean=10}
    \label{fig:ss10}
\end{figure}

	%\includegraphics[width=\columnwidth]{./codes/ss10.png}
	

	

\item Two defective bulbs are present in a set of five bulbs. To remove the two
defective bulbs, the bulbs are chosen randomly one by one and tested. If $X$
denotes the minimum number of bulbs that must be tested to find out the two
defective bulbs, then $\pr{X=3}$ (rounded off to two decimal places)
equals\\
\hfill(GATE ST 2023)\\
\iffalse
\let\negmedspace\undefined
\let\negthickspace\undefined
\documentclass[journal,12pt,twocolumn]{IEEEtran}
\usepackage{cite}
\usepackage{amsmath,amssymb,amsfonts,amsthm}
\usepackage{algorithmic}
\usepackage{graphicx}
\usepackage{textcomp}
\usepackage{xcolor}
\usepackage{txfonts}
\usepackage{listings}
%\usepackage{enumitem}
\usepackage{mathtools}
\usepackage{gensymb}
\usepackage[breaklinks=true]{hyperref}
\usepackage{tkz-euclide} % loads  TikZ and tkz-base
\usepackage{listings}
\usepackage[inline]{enumitem}
\DeclareMathOperator*{\Res}{Res}
\renewcommand\thesection{\arabic{section}}
\renewcommand\thesubsection{\thesection.\arabic{subsection}}
\renewcommand\thesubsubsection{\thesubsection.\arabic{subsubsection}}


\def\inputGnumericTable{}

\usepackage[latin1]{inputenc}                                 
\usepackage{color}                                            
\usepackage{array}                                            
\usepackage{longtable}                                        
\usepackage{calc}                                             
\usepackage{multirow}                                         
\usepackage{hhline}                                           
\usepackage{ifthen}
\usepackage{caption} 
\captionsetup[table]{skip=3pt}  
\providecommand{\pr}[1]{\ensuremath{\Pr\left(#1\right)}}
\providecommand{\cbrak}[1]{\ensuremath{\left\{#1\right\}}}

\renewcommand\thesectiondis{\arabic{section}}
\renewcommand\thesubsectiondis{\thesectiondis.\arabic{subsection}}
\renewcommand\thesubsubsectiondis{\thesubsectiondis.\arabic{subsubsection}}

\def\inputGnumericTable{}                                 %%

\lstset{
frame=single, 
breaklines=true,
columns=fullflexible
}

\begin{document}

\newtheorem{theorem}{Theorem}[section]
\newtheorem{problem}{Problem}
\newtheorem{proposition}{Proposition}[section]
\newtheorem{lemma}{Lemma}[section]
\newtheorem{corollary}[theorem]{Corollary}
\newtheorem{example}{Example}[section]
\newtheorem{definition}[problem]{Definition}
\newcommand{\BEQA}{\begin{eqnarray}}
\newcommand{\EEQA}{\end{eqnarray}}
\newcommand{\define}{\stackrel{\triangle}{=}}
\newcommand{\xor}{\oplus}
\bibliographystyle{IEEEtran}

\providecommand{\mbf}{\mathbf}
\providecommand{\pr}[1]{\ensuremath{\Pr\left(#1\right)}}
\providecommand{\qfunc}[1]{\ensuremath{Q\left(#1\right)}}
\providecommand{\sbrak}[1]{\ensuremath{{}\left[#1\right]}}
\providecommand{\lsbrak}[1]{\ensuremath{{}\left[#1\right.}}
\providecommand{\rsbrak}[1]{\ensuremath{{}\left.#1\right]}}
\providecommand{\brak}[1]{\ensuremath{\left(#1\right)}}
\providecommand{\lbrak}[1]{\ensuremath{\left(#1\right.}}
\providecommand{\rbrak}[1]{\ensuremath{\left.#1\right)}}
\providecommand{\cbrak}[1]{\ensuremath{\left\{#1\right\}}}
\providecommand{\lcbrak}[1]{\ensuremath{\left\{#1\right.}}
\providecommand{\rcbrak}[1]{\ensuremath{\left.#1\right\}}}
\theoremstyle{remark}
\newtheorem{rem}{Remark}
\newcommand{\sgn}{\mathop{\mathrm{sgn}}}

\newcommand{\solution}{\noindent \textbf{Solution: }}
\newcommand{\cosec}{\,\text{cosec}\,}
\providecommand{\dec}[2]{\ensuremath{\overset{#1}{\underset{#2}{\gtrless}}}}
\newcommand{\myvec}[1]{\ensuremath{\begin{pmatrix}#1\end{pmatrix}}}
\newcommand{\mydet}[1]{\ensuremath{\begin{vmatrix}#1\end{vmatrix}}}

\let\vec\mathbf


\vspace{3cm}

\title{
  

  Assignment -5 in \LaTeX
    
  }
  \author{ Muzaan Mohammed Faizel A P\\
  EE22BTECH11036
  }	
% make the title area
\maketitle
\newpage
\bigskip
\renewcommand{\thefigure}{\theenumi}
\renewcommand{\thetable}{\theenumi}
\renewcommand{\thetable}{\arabic{table}} 
\textbf{GATE 2023 BM QN.12}
For a Binomial random variable $X$, E($X$) and Var($X$) are the expectation and
variance, respectively. Which one of the following statements CANNOT be true?
\begin{table}[ht!]
		\centering
		\input{2023/BM/12/tables/table1.tex}
		\caption{}
		\label{table:table1}	
\end{table}
\fi
\solution
\begin{align}
X \sim \text{Bin}\brak{n,p} \nonumber
\end{align}

We know ,
\begin{align} 
	E\brak{X}=np\\
	Var\brak{X}=np\brak{1-p}\\
	0\leq p\leq1 \\
	\implies -1\leq -p \leq0\\
	\implies
	0 \leq 1-p \leq1\\
	\implies np\brak{1-p}\leq np
\end{align}
Therefore,
\begin{align} 
	Var\brak{X}\leq E\brak{X}
\end{align}
From the four options,the statement that cannot be true is option \brak{3}\\

\textbf{Simulation steps}\\
\textbf{Step 1: Generate a Range of Probabilities}\\
The program generates a range of probabilities (\(p\)) in increments of 0.0098, ranging from 0.01 to 0.99. This range is used to create different binomial distributions for subsequent calculations.
\\
% It then calculates the variance of the generated binomial random variable.\\
\textbf{Step 2: Generating binomial r.v from Uniform distribution}\\
\( U \sim \text{Uniform}(0, 1) \).

Defining a random variable X as:
\[
X = 
\begin{cases}
    1 & \text{if } U > p \\
    0 & \text{otherwise}
\end{cases}
\]
This becomes a Bernoulli rv.
The count variable evaluates the Binomial r.v by the summation of Bernoulli r.v

\[
count= \sum_{i=1}^{n} X
\]
\textbf{Step 3: Calculate Variances for Each Probability}\\
For each probability (\(p\)) in the generated range, the program generates a binomial random variable using the given mean and the inverse of the probability (\(mean/p\)) from uniform distribution.\\
\textbf{Step 4: Find Maximum and Minimum Variances}\\
After calculating variances for each probability, the program identifies the maximum and minimum variances in the generated set.\\
\textbf{Step 5: User Input}\\
The program prompts the user to input a variance value for validation.
\\
\textbf{Step 6: Check Validity}\\
The user-input variance is compared against the computed maximum and minimum variances. If the input variance falls within this range (inclusive of the minimum and exclusive of the maximum), the program outputs "Valid." Otherwise, it outputs "Invalid."
\begin{figure}[ht!]
    \centering
    \includegraphics[width=\columnwidth]{2023/BM/12/codes/ss10.png}
    \caption{Variance for mean=10}
    \label{fig:ss10}
\end{figure}

	%\includegraphics[width=\columnwidth]{./codes/ss10.png}
	

	

\item Let $X$ be a random variable with cumulative distribution function
\begin{align}
F_X(x) &= 
	\begin{cases}
		0 & \text{if $x < -1$}\\
		\frac{1}{4}\brak{x+1} & \text{if $-1 \leq x < 0$}\\
		\frac{1}{4}\brak{x+3} & \text{if $0 \leq x < 1$}\\
		1 & \text{if $ x \geq 1$}\\
	\end{cases}
\end{align}
Which one of the following statements is true?
\begin{enumerate}[label=(\Alph*)]
    \item\begin{align} \lim_{n \to \infty} \pr{-\frac{1}{2} + \frac{1}{n} < X < -\frac{1}{n}} = \frac{5}{8}\end{align}
    \item \begin{align}\lim_{n \to \infty} \pr{-\frac{1}{2} - \frac{1}{n} < X < \frac{1}{n}} = \frac{5}{8}\end{align}
    \item \begin{align}\lim_{n \to \infty} \pr{X = \frac{1}{n}} = \frac{1}{2}\end{align}
    \item \begin{align}\pr{X = 0} = \frac{1}{3}\end{align}
\end{enumerate}

\hfill(GATE ST 2023)\\
\iffalse
\let\negmedspace\undefined
\let\negthickspace\undefined
\documentclass[journal,12pt,twocolumn]{IEEEtran}
\usepackage{cite}
\usepackage{amsmath,amssymb,amsfonts,amsthm}
\usepackage{algorithmic}
\usepackage{graphicx}
\usepackage{textcomp}
\usepackage{xcolor}
\usepackage{txfonts}
\usepackage{listings}
%\usepackage{enumitem}
\usepackage{mathtools}
\usepackage{gensymb}
\usepackage[breaklinks=true]{hyperref}
\usepackage{tkz-euclide} % loads  TikZ and tkz-base
\usepackage{listings}
\usepackage[inline]{enumitem}
\DeclareMathOperator*{\Res}{Res}
\renewcommand\thesection{\arabic{section}}
\renewcommand\thesubsection{\thesection.\arabic{subsection}}
\renewcommand\thesubsubsection{\thesubsection.\arabic{subsubsection}}


\def\inputGnumericTable{}

\usepackage[latin1]{inputenc}                                 
\usepackage{color}                                            
\usepackage{array}                                            
\usepackage{longtable}                                        
\usepackage{calc}                                             
\usepackage{multirow}                                         
\usepackage{hhline}                                           
\usepackage{ifthen}
\usepackage{caption} 
\captionsetup[table]{skip=3pt}  
\providecommand{\pr}[1]{\ensuremath{\Pr\left(#1\right)}}
\providecommand{\cbrak}[1]{\ensuremath{\left\{#1\right\}}}

\renewcommand\thesectiondis{\arabic{section}}
\renewcommand\thesubsectiondis{\thesectiondis.\arabic{subsection}}
\renewcommand\thesubsubsectiondis{\thesubsectiondis.\arabic{subsubsection}}

\def\inputGnumericTable{}                                 %%

\lstset{
frame=single, 
breaklines=true,
columns=fullflexible
}

\begin{document}

\newtheorem{theorem}{Theorem}[section]
\newtheorem{problem}{Problem}
\newtheorem{proposition}{Proposition}[section]
\newtheorem{lemma}{Lemma}[section]
\newtheorem{corollary}[theorem]{Corollary}
\newtheorem{example}{Example}[section]
\newtheorem{definition}[problem]{Definition}
\newcommand{\BEQA}{\begin{eqnarray}}
\newcommand{\EEQA}{\end{eqnarray}}
\newcommand{\define}{\stackrel{\triangle}{=}}
\newcommand{\xor}{\oplus}
\bibliographystyle{IEEEtran}

\providecommand{\mbf}{\mathbf}
\providecommand{\pr}[1]{\ensuremath{\Pr\left(#1\right)}}
\providecommand{\qfunc}[1]{\ensuremath{Q\left(#1\right)}}
\providecommand{\sbrak}[1]{\ensuremath{{}\left[#1\right]}}
\providecommand{\lsbrak}[1]{\ensuremath{{}\left[#1\right.}}
\providecommand{\rsbrak}[1]{\ensuremath{{}\left.#1\right]}}
\providecommand{\brak}[1]{\ensuremath{\left(#1\right)}}
\providecommand{\lbrak}[1]{\ensuremath{\left(#1\right.}}
\providecommand{\rbrak}[1]{\ensuremath{\left.#1\right)}}
\providecommand{\cbrak}[1]{\ensuremath{\left\{#1\right\}}}
\providecommand{\lcbrak}[1]{\ensuremath{\left\{#1\right.}}
\providecommand{\rcbrak}[1]{\ensuremath{\left.#1\right\}}}
\theoremstyle{remark}
\newtheorem{rem}{Remark}
\newcommand{\sgn}{\mathop{\mathrm{sgn}}}

\newcommand{\solution}{\noindent \textbf{Solution: }}
\newcommand{\cosec}{\,\text{cosec}\,}
\providecommand{\dec}[2]{\ensuremath{\overset{#1}{\underset{#2}{\gtrless}}}}
\newcommand{\myvec}[1]{\ensuremath{\begin{pmatrix}#1\end{pmatrix}}}
\newcommand{\mydet}[1]{\ensuremath{\begin{vmatrix}#1\end{vmatrix}}}

\let\vec\mathbf


\vspace{3cm}

\title{
  

  Assignment -5 in \LaTeX
    
  }
  \author{ Muzaan Mohammed Faizel A P\\
  EE22BTECH11036
  }	
% make the title area
\maketitle
\newpage
\bigskip
\renewcommand{\thefigure}{\theenumi}
\renewcommand{\thetable}{\theenumi}
\renewcommand{\thetable}{\arabic{table}} 
\textbf{GATE 2023 BM QN.12}
For a Binomial random variable $X$, E($X$) and Var($X$) are the expectation and
variance, respectively. Which one of the following statements CANNOT be true?
\begin{table}[ht!]
		\centering
		\input{2023/BM/12/tables/table1.tex}
		\caption{}
		\label{table:table1}	
\end{table}
\fi
\solution
\begin{align}
X \sim \text{Bin}\brak{n,p} \nonumber
\end{align}

We know ,
\begin{align} 
	E\brak{X}=np\\
	Var\brak{X}=np\brak{1-p}\\
	0\leq p\leq1 \\
	\implies -1\leq -p \leq0\\
	\implies
	0 \leq 1-p \leq1\\
	\implies np\brak{1-p}\leq np
\end{align}
Therefore,
\begin{align} 
	Var\brak{X}\leq E\brak{X}
\end{align}
From the four options,the statement that cannot be true is option \brak{3}\\

\textbf{Simulation steps}\\
\textbf{Step 1: Generate a Range of Probabilities}\\
The program generates a range of probabilities (\(p\)) in increments of 0.0098, ranging from 0.01 to 0.99. This range is used to create different binomial distributions for subsequent calculations.
\\
% It then calculates the variance of the generated binomial random variable.\\
\textbf{Step 2: Generating binomial r.v from Uniform distribution}\\
\( U \sim \text{Uniform}(0, 1) \).

Defining a random variable X as:
\[
X = 
\begin{cases}
    1 & \text{if } U > p \\
    0 & \text{otherwise}
\end{cases}
\]
This becomes a Bernoulli rv.
The count variable evaluates the Binomial r.v by the summation of Bernoulli r.v

\[
count= \sum_{i=1}^{n} X
\]
\textbf{Step 3: Calculate Variances for Each Probability}\\
For each probability (\(p\)) in the generated range, the program generates a binomial random variable using the given mean and the inverse of the probability (\(mean/p\)) from uniform distribution.\\
\textbf{Step 4: Find Maximum and Minimum Variances}\\
After calculating variances for each probability, the program identifies the maximum and minimum variances in the generated set.\\
\textbf{Step 5: User Input}\\
The program prompts the user to input a variance value for validation.
\\
\textbf{Step 6: Check Validity}\\
The user-input variance is compared against the computed maximum and minimum variances. If the input variance falls within this range (inclusive of the minimum and exclusive of the maximum), the program outputs "Valid." Otherwise, it outputs "Invalid."
\begin{figure}[ht!]
    \centering
    \includegraphics[width=\columnwidth]{2023/BM/12/codes/ss10.png}
    \caption{Variance for mean=10}
    \label{fig:ss10}
\end{figure}

	%\includegraphics[width=\columnwidth]{./codes/ss10.png}
	

	

\item Three unbiased coins were tossed. Provided that at least two outcomes are tails,the probability of having all three outcomes as tails is\\
\hfill(GATE PI 2023)\\
\iffalse
\let\negmedspace\undefined
\let\negthickspace\undefined
\documentclass[journal,12pt,twocolumn]{IEEEtran}
\usepackage{cite}
\usepackage{amsmath,amssymb,amsfonts,amsthm}
\usepackage{algorithmic}
\usepackage{graphicx}
\usepackage{textcomp}
\usepackage{xcolor}
\usepackage{txfonts}
\usepackage{listings}
%\usepackage{enumitem}
\usepackage{mathtools}
\usepackage{gensymb}
\usepackage[breaklinks=true]{hyperref}
\usepackage{tkz-euclide} % loads  TikZ and tkz-base
\usepackage{listings}
\usepackage[inline]{enumitem}
\DeclareMathOperator*{\Res}{Res}
\renewcommand\thesection{\arabic{section}}
\renewcommand\thesubsection{\thesection.\arabic{subsection}}
\renewcommand\thesubsubsection{\thesubsection.\arabic{subsubsection}}


\def\inputGnumericTable{}

\usepackage[latin1]{inputenc}                                 
\usepackage{color}                                            
\usepackage{array}                                            
\usepackage{longtable}                                        
\usepackage{calc}                                             
\usepackage{multirow}                                         
\usepackage{hhline}                                           
\usepackage{ifthen}
\usepackage{caption} 
\captionsetup[table]{skip=3pt}  
\providecommand{\pr}[1]{\ensuremath{\Pr\left(#1\right)}}
\providecommand{\cbrak}[1]{\ensuremath{\left\{#1\right\}}}

\renewcommand\thesectiondis{\arabic{section}}
\renewcommand\thesubsectiondis{\thesectiondis.\arabic{subsection}}
\renewcommand\thesubsubsectiondis{\thesubsectiondis.\arabic{subsubsection}}

\def\inputGnumericTable{}                                 %%

\lstset{
frame=single, 
breaklines=true,
columns=fullflexible
}

\begin{document}

\newtheorem{theorem}{Theorem}[section]
\newtheorem{problem}{Problem}
\newtheorem{proposition}{Proposition}[section]
\newtheorem{lemma}{Lemma}[section]
\newtheorem{corollary}[theorem]{Corollary}
\newtheorem{example}{Example}[section]
\newtheorem{definition}[problem]{Definition}
\newcommand{\BEQA}{\begin{eqnarray}}
\newcommand{\EEQA}{\end{eqnarray}}
\newcommand{\define}{\stackrel{\triangle}{=}}
\newcommand{\xor}{\oplus}
\bibliographystyle{IEEEtran}

\providecommand{\mbf}{\mathbf}
\providecommand{\pr}[1]{\ensuremath{\Pr\left(#1\right)}}
\providecommand{\qfunc}[1]{\ensuremath{Q\left(#1\right)}}
\providecommand{\sbrak}[1]{\ensuremath{{}\left[#1\right]}}
\providecommand{\lsbrak}[1]{\ensuremath{{}\left[#1\right.}}
\providecommand{\rsbrak}[1]{\ensuremath{{}\left.#1\right]}}
\providecommand{\brak}[1]{\ensuremath{\left(#1\right)}}
\providecommand{\lbrak}[1]{\ensuremath{\left(#1\right.}}
\providecommand{\rbrak}[1]{\ensuremath{\left.#1\right)}}
\providecommand{\cbrak}[1]{\ensuremath{\left\{#1\right\}}}
\providecommand{\lcbrak}[1]{\ensuremath{\left\{#1\right.}}
\providecommand{\rcbrak}[1]{\ensuremath{\left.#1\right\}}}
\theoremstyle{remark}
\newtheorem{rem}{Remark}
\newcommand{\sgn}{\mathop{\mathrm{sgn}}}

\newcommand{\solution}{\noindent \textbf{Solution: }}
\newcommand{\cosec}{\,\text{cosec}\,}
\providecommand{\dec}[2]{\ensuremath{\overset{#1}{\underset{#2}{\gtrless}}}}
\newcommand{\myvec}[1]{\ensuremath{\begin{pmatrix}#1\end{pmatrix}}}
\newcommand{\mydet}[1]{\ensuremath{\begin{vmatrix}#1\end{vmatrix}}}

\let\vec\mathbf


\vspace{3cm}

\title{
  

  Assignment -5 in \LaTeX
    
  }
  \author{ Muzaan Mohammed Faizel A P\\
  EE22BTECH11036
  }	
% make the title area
\maketitle
\newpage
\bigskip
\renewcommand{\thefigure}{\theenumi}
\renewcommand{\thetable}{\theenumi}
\renewcommand{\thetable}{\arabic{table}} 
\textbf{GATE 2023 BM QN.12}
For a Binomial random variable $X$, E($X$) and Var($X$) are the expectation and
variance, respectively. Which one of the following statements CANNOT be true?
\begin{table}[ht!]
		\centering
		\input{2023/BM/12/tables/table1.tex}
		\caption{}
		\label{table:table1}	
\end{table}
\fi
\solution
\begin{align}
X \sim \text{Bin}\brak{n,p} \nonumber
\end{align}

We know ,
\begin{align} 
	E\brak{X}=np\\
	Var\brak{X}=np\brak{1-p}\\
	0\leq p\leq1 \\
	\implies -1\leq -p \leq0\\
	\implies
	0 \leq 1-p \leq1\\
	\implies np\brak{1-p}\leq np
\end{align}
Therefore,
\begin{align} 
	Var\brak{X}\leq E\brak{X}
\end{align}
From the four options,the statement that cannot be true is option \brak{3}\\

\textbf{Simulation steps}\\
\textbf{Step 1: Generate a Range of Probabilities}\\
The program generates a range of probabilities (\(p\)) in increments of 0.0098, ranging from 0.01 to 0.99. This range is used to create different binomial distributions for subsequent calculations.
\\
% It then calculates the variance of the generated binomial random variable.\\
\textbf{Step 2: Generating binomial r.v from Uniform distribution}\\
\( U \sim \text{Uniform}(0, 1) \).

Defining a random variable X as:
\[
X = 
\begin{cases}
    1 & \text{if } U > p \\
    0 & \text{otherwise}
\end{cases}
\]
This becomes a Bernoulli rv.
The count variable evaluates the Binomial r.v by the summation of Bernoulli r.v

\[
count= \sum_{i=1}^{n} X
\]
\textbf{Step 3: Calculate Variances for Each Probability}\\
For each probability (\(p\)) in the generated range, the program generates a binomial random variable using the given mean and the inverse of the probability (\(mean/p\)) from uniform distribution.\\
\textbf{Step 4: Find Maximum and Minimum Variances}\\
After calculating variances for each probability, the program identifies the maximum and minimum variances in the generated set.\\
\textbf{Step 5: User Input}\\
The program prompts the user to input a variance value for validation.
\\
\textbf{Step 6: Check Validity}\\
The user-input variance is compared against the computed maximum and minimum variances. If the input variance falls within this range (inclusive of the minimum and exclusive of the maximum), the program outputs "Valid." Otherwise, it outputs "Invalid."
\begin{figure}[ht!]
    \centering
    \includegraphics[width=\columnwidth]{2023/BM/12/codes/ss10.png}
    \caption{Variance for mean=10}
    \label{fig:ss10}
\end{figure}

	%\includegraphics[width=\columnwidth]{./codes/ss10.png}
	

	

\item Let $X$ be a random variable with probability density function
\begin{align}
p_{X}\brak{x}=\begin{cases}
		e^{-x} & if x\ge 0\\ 
		0 & otherwise
	\end{cases}
\end{align}
For $a<b$, if $U\brak{a,b}$ denotes the uniform distribution over the interval $\brak{a,b}$,
then which of the following statements is/are true?
\begin{enumerate}[label=(\Alph*)]
\item $e^{-X}$ follows $U\brak{-1,0}$ distribution
\item $1-e^{-X}$follows $U\brak{0,2}$ distribution
\item $2e^{-X}-1$follows $U\brak{-1,1}$ distribution
\item The probability mass function of $Y=\sbrak{X}$ is
$\pr{Y=k}=e^{-k}\brak{1-e^{-1}}$ for k= 0, 1, 2, …,
where $\sbrak{X}$denotes the largest integer not exceeding $x$
\end{enumerate}
\hfill(GATE ST 2023)\\
\solution
\iffalse
\documentclass[book,11pt]{IEEEtran}
\usepackage{setspace}
\usepackage{gensymb}
\singlespacing
\usepackage[cmex10]{amsmath}
\usepackage{amsthm}
\usepackage{mathrsfs}
\usepackage{txfonts}
\usepackage{stfloats}
\usepackage{bm}
\usepackage{cite}
\usepackage{cases}
\usepackage{subfig}
\usepackage{longtable}
\usepackage{multirow}
\usepackage{enumitem}
\usepackage{mathtools}
\usepackage{tikz}
\usepackage{circuitikz}
\usepackage{verbatim}
\usepackage[breaklinks=true]{hyperref}
\usepackage{tkz-euclide} % loads  TikZ and tkz-base
\usepackage{listings}
\usepackage{color}    
\usepackage{array}    
\usepackage{longtable}
\usepackage{calc}     
\usepackage{multirow} 
\usepackage{hhline}   
\usepackage{ifthen}   
\usepackage{lscape}     
\usepackage{chngcntr}
\usepackage{float}
\DeclareMathOperator*{\Res}{Res}
\renewcommand\thesection{\arabic{section}}
\renewcommand\thesubsection{\thesection.\arabic{subsection}}
\renewcommand\thesubsubsection{\thesubsection.\arabic{subsubsection}}

\renewcommand\thesectiondis{\arabic{section}}
\renewcommand\thesubsectiondis{\thesectiondis.\arabic{subsection}}
\renewcommand\thesubsubsectiondis{\thesubsectiondis.\arabic{subsubsection}}
\renewcommand\thetable{\arabic{table}}
% correct bad hyphenation here
\hyphenation{op-tical net-works semi-conduc-tor}
\def\inputGnumericTable{}                                 %%

\lstset{
%language=C,
frame=single, 
breaklines=true,
columns=fullflexible
}
%\lstset{
%language=tex,
%frame=single, 
%breaklines=true
%}

\begin{document}
\newtheorem{theorem}{Theorem}[section]
\newtheorem{problem}{Problem}
\newtheorem{proposition}{Proposition}[section]
\newtheorem{lemma}{Lemma}[section]
\newtheorem{corollary}[theorem]{Corollary}
\newtheorem{example}{Example}[section]
\newtheorem{definition}[problem]{Definition}
\newcommand{\BEQA}{\begin{eqnarray}}
\newcommand{\EEQA}{\end{eqnarray}}
\newcommand{\define}{\stackrel{\triangle}{=}}
\bibliographystyle{IEEEtran}
\providecommand{\mbf}{\mathbf}
\providecommand{\pr}[1]{\ensuremath{\Pr\left(#1\right)}}
\providecommand{\qfunc}[1]{\ensuremath{Q\left(#1\right)}}
\providecommand{\sbrak}[1]{\ensuremath{{}\left[#1\right]}}
\providecommand{\lsbrak}[1]{\ensuremath{{}\left[#1\right.}}
\providecommand{\rsbrak}[1]{\ensuremath{{}\left.#1\right]}}
\providecommand{\brak}[1]{\ensuremath{\left(#1\right)}}
\providecommand{\lbrak}[1]{\ensuremath{\left(#1\right.}}
\providecommand{\rbrak}[1]{\ensuremath{\left.#1\right)}}
\providecommand{\cbrak}[1]{\ensuremath{\left\{#1\right\}}}
\providecommand{\lcbrak}[1]{\ensuremath{\left\{#1\right.}}
\providecommand{\rcbrak}[1]{\ensuremath{\left.#1\right\}}}
\theoremstyle{remark}
\newtheorem{rem}{Remark}
\newcommand{\sgn}{\mathop{\mathrm{sgn}}}
\providecommand{\abs}[1]{\left\vert#1\right\vert}
\providecommand{\res}[1]{\Res\displaylimits_{#1}} 
\providecommand{\norm}[1]{\left\lVert#1\right\rVert}
\providecommand{\mtx}[1]{\mathbf{#1}}
\providecommand{\mean}[1]{E\left[ #1 \right]}
\providecommand{\fourier}{\overset{\mathcal{F}}{ \rightleftharpoons}}
\providecommand{\system}[1]{\overset{\mathcal{#1}}{ \longleftrightarrow}}
\newcommand{\solution}{\noindent \textbf{Solution: }}
\newcommand{\cosec}{\,\text{cosec}\,}
\providecommand{\dec}[2]{\ensuremath{\overset{#1}{\underset{#2}{\gtrless}}}}
\newcommand{\myvec}[1]{\ensuremath{\begin{pmatrix}#1\end{pmatrix}}}
\newcommand{\mydet}[1]{\ensuremath{\begin{vmatrix}#1\end{vmatrix}}}
\let\vec\mathbf
\def\putbox#1#2#3{\makebox[0in][l]{\makebox[#1][l]{}\raisebox{\baselineskip}[0in][0in]{\raisebox{#2}[0in][0in]{#3}}}}
     \def\rightbox#1{\makebox[0in][r]{#1}}
     \def\centbox#1{\makebox[0in]{#1}}
     \def\topbox#1{\raisebox{-\baselineskip}[0in][0in]{#1}}
     \def\midbox#1{\raisebox{-0.5\baselineskip}[0in][0in]{#1}}

\vspace{3cm}
\title{
%	\logo{
Probability and Random Processes
%	}
}
\author{ Sarvesh K\\EE22BTECH11046$^{*}$% <-this % stops a space
%	\thanks{*The author is with the Department
%		of Electrical Engineering, Indian Institute of Technology, Hyderabad
%		502285 India e-mail:  gadepall@iith.ac.in. All content in this manual is released under GNU GPL.  Free and open source.}
	
}	
%\title{
%	\logo{Matrix Analysis through Octave}{\begin{center}\includegraphics[scale=.24]{tlc}\end{center}}{}{HAMDSP}
%}


% paper title
% can use linebreaks \\ within to get better formatting as desired
%\title{Matrix Analysis through Octave}
%
%
% author names and IEEE memberships
% note positions of commas and nonbreaking spaces ( ~ ) LaTeX will not break
% a structure at a ~ so this keeps an author's name from being broken across
% two lines.
% use \thanks{} to gain access to the first footnote area
% a separate \thanks must be used for each paragraph as LaTeX2e's \thanks
% was not built to handle multiple paragraphs
%

%\author{<-this % stops a space
%\thanks{}}
%}
% note the % following the last \IEEEmembership and also \thanks - 
% these prevent an unwanted space from occurring between the last author name
% and the end of the author line. i.e., if you had this:
% 
% \author{....lastname \thanks{...} \thanks{...} }
%                     ^------------^------------^----Do not want these spaces!
%
% a space would be appended to the last name and could cause every name on that
% line to be shifted left slightly. This is one of those "LaTeX things". For
% instance, "\textbf{A} \textbf{B}" will typeset as "A B" not "AB". To get
% "AB" then you have to do: "\textbf{A}\textbf{B}"
% \thanks is no different in this regard, so shield the last } of each \thanks
% that ends a line with a % and do not let a space in before the next \thanks.
% Spaces after \IEEEmembership other than the last one are OK (and needed) as
% you are supposed to have spaces between the names. For what it is worth,
% this is a minor point as most people would not even notice if the said evil
% space somehow managed to creep in.



% The paper headers
%\markboth{Journal of \LaTeX\ Class Files,~Vol.~6, No.~1, January~2007}%
%{Shell \MakeLowercase{\textit{et al.}}: Bare Demo of IEEEtran.cls for Journals}
% The only time the second header will appear is for the odd numbered pages
% after the title page when using the twoside option.
% 
% *** Note that you probably will NOT want to include the author's ***
% *** name in the headers of peer review papers.                   ***
% You can use \ifCLASSOPTIONpeerreview for conditional compilation here if
% you desire.




% If you want to put a publisher's ID mark on the page you can do it like
% this:
%\IEEEpubid{0000--0000/00\$00.00~\copyright~2007 IEEE}
% Remember, if you use this you must call \IEEEpubidadjcol in the second
% column for its text to clear the IEEEpubid mark.



% make the title area
\maketitle

\newpage

%\tableofcontents

\bigskip

\renewcommand{\thefigure}{\theenumi}
\renewcommand{\thetable}{\theenumi}
%\renewcommand{\theequation}{\theenumi}

%\begin{abstract}
%%\boldmath
%In this letter, an algorithm for evaluating the exact analytical bit error rate  (BER)  for the piecewise linear (PL) combiner for  multiple relays is presented. Previous results were available only for upto three relays. The algorithm is unique in the sense that  the actual mathematical expressions, that are prohibitively large, need not be explicitly obtained. The diversity gain due to multiple relays is shown through plots of the analytical BER, well supported by simulations. 
%
%\end{abstract}
% IEEEtran.cls defaults to using nonbold math in the Abstract.
% This preserves the distinction between vectors and scalars. However,
% if the journal you are submitting to favors bold math in the abstract,
% then you can use LaTeX's standard command \boldmath at the very start
% of the abstract to achieve this. Many IEEE journals frown on math
% in the abstract anyway.

% Note that keywords are not normally used for peerreview papers.
%\begin{IEEEkeywords}
%Cooperative diversity, decode and forward, piecewise linear
%\end{IEEEkeywords}



% For peer review papers, you can put extra information on the cover
% page as needed:
% \ifCLASSOPTIONpeerreview
% \begin{center} \bfseries EDICS Category: 3-BBND \end{center}
% \fi
%
% For peerreview papers, this IEEEtran command inserts a page break and
% creates the second title. It will be ignored for other modes.
%\IEEEpeerreviewmaketitle

\textbf{Question:}\\
Let $X$ be a random variable with probability density function
\begin{align}
p_{X}\brak{x}=\begin{cases}
		e^{-x} & if x\ge 0\\ 
		0 & otherwise
	\end{cases}
\end{align}
For $a<b$, if $U\brak{a,b}$ denotes the uniform distribution over the interval $\brak{a,b}$,
then which of the following statements is/are true?
\begin{enumerate}[label=(\Alph*)]
\item $e^{-X}$ follows $U\brak{-1,0}$ distribution
\item $1-e^{-X}$follows $U\brak{0,2}$ distribution
\item $2e^{-X}-1$follows $U\brak{-1,1}$ distribution
\item The probability mass function of $Y=\sbrak{X}$ is
$\pr{Y=k}=e^{-k}\brak{1-e^{-1}}$ for k= 0, 1, 2, …,
where $\sbrak{X}$denotes the largest integer not exceeding $x$
\end{enumerate}
\hfill(GATE ST 2023)\\
\solution
\fi
Let $Y\sim U(a,b)$, then
\begin{align}
p_Y\brak{y}=\begin{cases}
		\frac{1}{b-a} & a<y<b\\ 
		0 & \text{otherwise}
	\end{cases}
\end{align}
and for $a<y<b$ 
\begin{align}
F_Y\brak{y}&=\pr{Y \le y}\\
&=\int_{a}^{y}\frac{1}{b-a}dy\\
&=\frac{y-a}{b-a}
\end{align}
Similarly,for $x\ge 0$
\begin{align}
F_X\brak{x}&=\pr{X \le x}\\
&=\int_{0}^{x}e^{-x}dx\\
&=1-e^{-x}
\end{align}
\begin{enumerate}[label=(\Alph*)]
\item $Y=e^{-X}=U(a,b)$\\
for $a<y<b$ 
\begin{align}
F_Y\brak{y}&=\pr{e^{-X}\le y}\\
&=\pr{X\ge -\ln y}\\
&=1-F_X\brak{-\ln y}\\
&=1-\brak{1-y}\\
&=y
\end{align}
Comparing this with CDF of Uniform distribution, we obtain
\begin{align}
a=0,b=1\\
\therefore Y\sim U(0,1)
\end{align}
\item $Y=1-e^{-X}=U(a,b)$\\
for $a<y<b$ 
\begin{align}
F_Y\brak{y}&=\pr{1-e^{-X}\le y}\\
&=\pr{e^{-X}\ge 1-y}\\
&=\pr{X \le -\ln\brak{1-y}}\\
&=F_X\brak{-\ln\brak{1-y}}\\
&=1-\brak{1-y}\\
&=y\\
\implies Y&\sim U(0,1)
\end{align}
\item $Y=2e^{-X}-1=U(a,b)$\\
for $a<y<b$ 
\begin{align}
F_Y\brak{y}&=\pr{2e^{-X}-1\le y}\\
&=\pr{X\ge -\ln\brak{\frac{y+1}{2}}}\\
&=1-F_X\brak{-\ln\brak{\frac{y+1}{2}}}
&=1-\brak{1-\frac{y+1}{2}}\\
&=\frac{y+1}{2}
\end{align}
Comparing this with CDF of Uniform distribution, we obtain
\begin{align}
a=-1,b=1\\
\therefore Y\sim U(-1,1)
\end{align}
\item $Y=\sbrak{X}$
\begin{align}
\pr{Y=k}&=\pr{\sbrak{X}=k}\\
&=\pr{k\le X<k+1}\\
&=\int_{k}^{k+1}e^{-x}dx\\
&=e^{-k}\brak{1-e^{-1}} \text{for k=0,1,2..}
\end{align}
\item Generation of Random Variable $X$ in C language
\begin{enumerate}[label=(\roman*)]
\item rand() / (double)RAND\_MAX:\\ This generates a random variable between 0 and RAND\_MAX and divides it by RAND\_MAX to obtain a uniform distribution between 0 and 1.
\item -log(rand() / (double)RAND\_MAX) :\\ This transforms the uniform distribution between 0 and 1 into an exponential distribution by making the values vary from 0 to $\infty$.
\item Alternatively the Uniform distribution can be converted into Gaussian distribution using the Central Limit Theorem.
\item Gaussian is then converted into chi-square distribution with degree of freedom 2 which is similar to an exponential distribution.
\end{enumerate}
\end{enumerate} 
\begin{figure}[H]
	\centering
	\includegraphics[width=\columnwidth]{2023/ST/53/figs/cdf_comp_1.png}
	\label{fig:i_2023/st/53/1}
	\caption{$e^{-X}$ vs. $U\brak{-1,0}$\\Graphs don't match, $\therefore$ wrong option}
\end{figure}
\begin{figure}[H]
	\centering
	\includegraphics[width=\columnwidth]{2023/ST/53/figs/cdf_comp_2.png}
	\label{fig:i_2023/st/53/2}
	\caption{$1-e^{-X}$ vs. $U\brak{0,2}$\\Graphs don't match, $\therefore$ wrong option}
\end{figure}
\begin{figure}[H]
	\centering
	\includegraphics[width=\columnwidth]{2023/ST/53/figs/cdf_comp_3.png}
	\label{fig:i_2023/st/53/3}
	\caption{$2e^{-X}-1$ vs. $U\brak{-1,1}$\\Graphs match, $\therefore$ correct option}
\end{figure}

\item Question: Let $X$ be a positive valued continuous random variable with finite mean $\mu$.
If $Y=[X]$, the largest integer less than or equal to $X$, then which of the
following statements is/are true?
\begin{enumerate}[label=(\Alph*)]
\item $\pr{Y \leq \mu} \leq \pr{X \leq \mu}$ for all $\mu \geq 0$
\item $\pr{Y \geq \mu} \leq \pr{X \geq \mu}$ for all $\mu \geq 0$
\item E(X) $<$ E(Y)
\item E(X) $>$ E(Y)
\end{enumerate}
\hfill(GATE ST 2023)\\
\iffalse
\let\negmedspace\undefined
\let\negthickspace\undefined
\documentclass[journal,12pt,twocolumn]{IEEEtran}
\usepackage{cite}
\usepackage{amsmath,amssymb,amsfonts,amsthm}
\usepackage{algorithmic}
\usepackage{graphicx}
\usepackage{textcomp}
\usepackage{xcolor}
\usepackage{txfonts}
\usepackage{listings}
\usepackage{enumitem}
\usepackage{mathtools}
\usepackage{gensymb}
\usepackage{comment}
\usepackage[breaklinks=true]{hyperref}
\usepackage{tkz-euclide} 
\usepackage{listings}
\usepackage{gvv}                                        
\def\inputGnumericTable{}                                 
\usepackage[latin1]{inputenc}                                
\usepackage{color}                                            
\usepackage{array}                                            
\usepackage{longtable}                                       
\usepackage{calc}                                             
\usepackage{multirow}                                         
\usepackage{hhline}                                           
\usepackage{ifthen}                                           
\usepackage{lscape}

\newtheorem{theorem}{Theorem}[section]
\newtheorem{problem}{Problem}
\newtheorem{proposition}{Proposition}[section]
\newtheorem{lemma}{Lemma}[section]
\newtheorem{corollary}[theorem]{Corollary}
\newtheorem{example}{Example}[section]
\newtheorem{definition}[problem]{Definition}
\newcommand{\BEQA}{\begin{eqnarray}}
\newcommand{\EEQA}{\end{eqnarray}}
\newcommand{\define}{\stackrel{\triangle}{=}}
\theoremstyle{remark}
\newtheorem{rem}{Remark}
\begin{document}

\bibliographystyle{IEEEtran}
\vspace{3cm}

\title{Probability Assignment}
\author{EE22BTECH11022-G.SAI HARSHITH$^{*}$% <-this % stops a space
}
\maketitle
\newpage
\bigskip
\renewcommand{\thefigure}{\theenumi}
\renewcommand{\thetable}{\theenumi}

Question: Let $X$ be a positive valued continuous random variable with finite mean $\mu$.
If $Y=[X]$, the largest integer less than or equal to $X$, then which of the
following statements is/are true?
\begin{enumerate}[label=(\Alph*)]
\item $\pr{Y \leq \mu} \leq \pr{X \leq \mu}$ for all $\mu \geq 0$
\item $\pr{Y \geq \mu} \leq \pr{X \geq \mu}$ for all $\mu \geq 0$
\item E(X) $<$ E(Y)
\item E(X) $>$ E(Y)
\end{enumerate}
\fi
\solution Given that $X$ is a positive valued random variable and $Y=[X]$.So,
\begin{align}
X&=Y+Z
\end{align}
Here, $Z$ is an uniform distrubtion.
\begin{align}
Z &\sim U[0,1)\\
F_Z(x)&=x\\
E(Z)&=\frac{1}{2}
\end{align}
Consider
\begin{enumerate}
\item 
\begin{align}
\pr{Y \leq \mu}&=\pr{X-Z \leq \mu}\\
&=\pr{Z \geq X-\mu}\\
&=E(1-F_Z(X-\mu))\\
&=E(1-X+\mu)\\
&=1-E(X)+\mu\\
&=1
\end{align}
From option (A), we have $1 \leq \pr{X \leq \mu}$. Option (A) is wrong since probability can't be greater than 1.
\item
\begin{align}
\pr{Y \geq \mu}&=\pr{X-Z \geq \mu}\\
&=\pr{Z \leq X-\mu}\\
&=E(F_Z(X-\mu))\\
&=E(X-\mu)\\
&=E(X)-\mu\\
&=0
\end{align}
 From option B, we have $\pr{X \leq \mu} \geq 0$. Option (B) is correct.
 \item
\begin{align}
E(Y)&=E(X-Z)\\
&=E(X)-E(Z)\\
&=\mu-\frac{1}{2}\\
&=E(X)-\frac{1}{2}
\end{align}
$E(X) > E(Y)$. Option (D) is correct and (C) is wrong.
\end{enumerate}
\textbf{Steps for Simulation:}
\begin{enumerate}
\item Taking $n$ samples, Generate n exponential random variable($X$) samples.
\item Generate $n$ samples of $Y=[X]$ by floor to every sample of $X$.
\item Find number of samples of $X$ where $X \leq \mu$ and $X \geq \mu$ and divide with $n$ to get $\pr{X \leq \mu}$ and $\pr{X \geq \mu}$ respectively.
\item Find number of samples of $Y$ where $Y \leq \mu$ and $Y \geq \mu$ and divide with $n$ to get $\pr{Y \leq \mu}$ and $\pr{Y \geq \mu}$ respectively.
\item Sum the $n$ samples of $X$ and $Y$ and divide with $n$ to get $E(X)$ and $E(Y)$.
\end{enumerate}
\begin{figure}[!ht]
\centering
\includegraphics[width=\columnwidth]{2023/ST/52/figs/figure.png}
\caption{CDF'S of X and Y for varying $\mu$ at x=1.5}
\end{figure}
\textbf{Note:}
At x $\in$ integers, $Y=X$ , so, CDF curves of $Y$ and $X$ are same. At non-integers we can see some difference in CDF curves in $X$ and $Y$.

\item In a diploid angiosperm species, flower colour is regulated by the R gene.
RR and Rr genotypes produce red flowers, whereas the rr genotype produces
white flowers. If two individual plants are randomly selected from a large
segregating population of a genetic cross between RR and rr parents, the
probability of both the plants producing red flowers is\\
\hfill(GATE XL 2023)\\
\iffalse
\let\negmedspace\undefined
\let\negthickspace\undefined
\documentclass[journal,12pt,twocolumn]{IEEEtran}
\usepackage{cite}
\usepackage{amsmath,amssymb,amsfonts,amsthm}
\usepackage{algorithmic}
\usepackage{graphicx}
\usepackage{textcomp}
\usepackage{xcolor}
\usepackage{txfonts}
\usepackage{listings}
%\usepackage{enumitem}
\usepackage{mathtools}
\usepackage{gensymb}
\usepackage[breaklinks=true]{hyperref}
\usepackage{tkz-euclide} % loads  TikZ and tkz-base
\usepackage{listings}
\usepackage[inline]{enumitem}
\DeclareMathOperator*{\Res}{Res}
\renewcommand\thesection{\arabic{section}}
\renewcommand\thesubsection{\thesection.\arabic{subsection}}
\renewcommand\thesubsubsection{\thesubsection.\arabic{subsubsection}}


\def\inputGnumericTable{}

\usepackage[latin1]{inputenc}                                 
\usepackage{color}                                            
\usepackage{array}                                            
\usepackage{longtable}                                        
\usepackage{calc}                                             
\usepackage{multirow}                                         
\usepackage{hhline}                                           
\usepackage{ifthen}
\usepackage{caption} 
\captionsetup[table]{skip=3pt}  
\providecommand{\pr}[1]{\ensuremath{\Pr\left(#1\right)}}
\providecommand{\cbrak}[1]{\ensuremath{\left\{#1\right\}}}

\renewcommand\thesectiondis{\arabic{section}}
\renewcommand\thesubsectiondis{\thesectiondis.\arabic{subsection}}
\renewcommand\thesubsubsectiondis{\thesubsectiondis.\arabic{subsubsection}}

\def\inputGnumericTable{}                                 %%

\lstset{
frame=single, 
breaklines=true,
columns=fullflexible
}

\begin{document}

\newtheorem{theorem}{Theorem}[section]
\newtheorem{problem}{Problem}
\newtheorem{proposition}{Proposition}[section]
\newtheorem{lemma}{Lemma}[section]
\newtheorem{corollary}[theorem]{Corollary}
\newtheorem{example}{Example}[section]
\newtheorem{definition}[problem]{Definition}
\newcommand{\BEQA}{\begin{eqnarray}}
\newcommand{\EEQA}{\end{eqnarray}}
\newcommand{\define}{\stackrel{\triangle}{=}}
\newcommand{\xor}{\oplus}
\bibliographystyle{IEEEtran}

\providecommand{\mbf}{\mathbf}
\providecommand{\pr}[1]{\ensuremath{\Pr\left(#1\right)}}
\providecommand{\qfunc}[1]{\ensuremath{Q\left(#1\right)}}
\providecommand{\sbrak}[1]{\ensuremath{{}\left[#1\right]}}
\providecommand{\lsbrak}[1]{\ensuremath{{}\left[#1\right.}}
\providecommand{\rsbrak}[1]{\ensuremath{{}\left.#1\right]}}
\providecommand{\brak}[1]{\ensuremath{\left(#1\right)}}
\providecommand{\lbrak}[1]{\ensuremath{\left(#1\right.}}
\providecommand{\rbrak}[1]{\ensuremath{\left.#1\right)}}
\providecommand{\cbrak}[1]{\ensuremath{\left\{#1\right\}}}
\providecommand{\lcbrak}[1]{\ensuremath{\left\{#1\right.}}
\providecommand{\rcbrak}[1]{\ensuremath{\left.#1\right\}}}
\theoremstyle{remark}
\newtheorem{rem}{Remark}
\newcommand{\sgn}{\mathop{\mathrm{sgn}}}

\newcommand{\solution}{\noindent \textbf{Solution: }}
\newcommand{\cosec}{\,\text{cosec}\,}
\providecommand{\dec}[2]{\ensuremath{\overset{#1}{\underset{#2}{\gtrless}}}}
\newcommand{\myvec}[1]{\ensuremath{\begin{pmatrix}#1\end{pmatrix}}}
\newcommand{\mydet}[1]{\ensuremath{\begin{vmatrix}#1\end{vmatrix}}}

\let\vec\mathbf


\vspace{3cm}

\title{
  

  Assignment -5 in \LaTeX
    
  }
  \author{ Muzaan Mohammed Faizel A P\\
  EE22BTECH11036
  }	
% make the title area
\maketitle
\newpage
\bigskip
\renewcommand{\thefigure}{\theenumi}
\renewcommand{\thetable}{\theenumi}
\renewcommand{\thetable}{\arabic{table}} 
\textbf{GATE 2023 BM QN.12}
For a Binomial random variable $X$, E($X$) and Var($X$) are the expectation and
variance, respectively. Which one of the following statements CANNOT be true?
\begin{table}[ht!]
		\centering
		\input{2023/BM/12/tables/table1.tex}
		\caption{}
		\label{table:table1}	
\end{table}
\fi
\solution
\begin{align}
X \sim \text{Bin}\brak{n,p} \nonumber
\end{align}

We know ,
\begin{align} 
	E\brak{X}=np\\
	Var\brak{X}=np\brak{1-p}\\
	0\leq p\leq1 \\
	\implies -1\leq -p \leq0\\
	\implies
	0 \leq 1-p \leq1\\
	\implies np\brak{1-p}\leq np
\end{align}
Therefore,
\begin{align} 
	Var\brak{X}\leq E\brak{X}
\end{align}
From the four options,the statement that cannot be true is option \brak{3}\\

\textbf{Simulation steps}\\
\textbf{Step 1: Generate a Range of Probabilities}\\
The program generates a range of probabilities (\(p\)) in increments of 0.0098, ranging from 0.01 to 0.99. This range is used to create different binomial distributions for subsequent calculations.
\\
% It then calculates the variance of the generated binomial random variable.\\
\textbf{Step 2: Generating binomial r.v from Uniform distribution}\\
\( U \sim \text{Uniform}(0, 1) \).

Defining a random variable X as:
\[
X = 
\begin{cases}
    1 & \text{if } U > p \\
    0 & \text{otherwise}
\end{cases}
\]
This becomes a Bernoulli rv.
The count variable evaluates the Binomial r.v by the summation of Bernoulli r.v

\[
count= \sum_{i=1}^{n} X
\]
\textbf{Step 3: Calculate Variances for Each Probability}\\
For each probability (\(p\)) in the generated range, the program generates a binomial random variable using the given mean and the inverse of the probability (\(mean/p\)) from uniform distribution.\\
\textbf{Step 4: Find Maximum and Minimum Variances}\\
After calculating variances for each probability, the program identifies the maximum and minimum variances in the generated set.\\
\textbf{Step 5: User Input}\\
The program prompts the user to input a variance value for validation.
\\
\textbf{Step 6: Check Validity}\\
The user-input variance is compared against the computed maximum and minimum variances. If the input variance falls within this range (inclusive of the minimum and exclusive of the maximum), the program outputs "Valid." Otherwise, it outputs "Invalid."
\begin{figure}[ht!]
    \centering
    \includegraphics[width=\columnwidth]{2023/BM/12/codes/ss10.png}
    \caption{Variance for mean=10}
    \label{fig:ss10}
\end{figure}

	%\includegraphics[width=\columnwidth]{./codes/ss10.png}
	

	

\item The frequencies for autosomal alleles $A$ and $a$ are $p = 0.5$ and $q = 0.5$,
respectively, where $A$ is dominant over $a$. Under the assumption of random
mating, the mating frequency among dominant parents is.\\
\hfill(GATE XL 2023)\\
\iffalse
\let\negmedspace\undefined
\let\negthickspace\undefined
\documentclass[journal,12pt,onecolumn]{IEEEtran}
\usepackage{cite}
\usepackage{amsmath,amssymb,amsfonts,amsthm}
\usepackage{algorithmic}
\usepackage{graphicx}
\usepackage{textcomp}
\usepackage{xcolor}
\usepackage{txfonts}
\usepackage{listings}
\usepackage{enumitem}
\usepackage{mathtools}
\usepackage{gensymb}
\usepackage[breaklinks=true]{hyperref}
\usepackage{tkz-euclide} % loads  TikZ and tkz-base
\usepackage{listings}
\usepackage{gvv}
\usepackage[latin1]{inputenc}                                 
\usepackage{color}                                            
\usepackage{array}                                            
\usepackage{longtable}                                        
\usepackage{calc}                                             
\usepackage{multirow}                                         
\usepackage{hhline}                                           
\usepackage{ifthen}                                           
%
%\usepackage{setspace}
%\usepackage{gensymb}
%\doublespacing
%\singlespacing

%\usepackage{graphicx}
%\usepackage{amssymb}
%\usepackage{relsize}
%\usepackage[cmex10]{amsmath}
%\usepackage{amsthm}
%\interdisplaylinepenalty=2500
%\savesymbol{iint}
%\usepackage{txfonts}
%\restoresymbol{TXF}{iint}
%\usepackage{wasysym}
%\usepackage{amsthm}
%\usepackage{iithtlc}
%\usepackage{mathrsfs}
%\usepackage{txfonts}
%\usepackage{stfloats}
%\usepackage{bm}
%\usepackage{cite}
%\usepackage{cases}
%\usepackage{subfig}
%\usepackage{xtab}
%\usepackage{longtable}
%\usepackage{multirow}
%\usepackage{algorithm}
%\usepackage{algpseudocode}
%\usepackage{enumitem}
%\usepackage{mathtools}
%\usepackage{tikz}
%\usepackage{circuitikz}
%\usepackage{verbatim}
%\usepackage{tfrupee}
%\usepackage{stmaryrd}
%\usetkzobj{all}
%    \usepackage{color}                                            %%
%    \usepackage{array}                                            %%
%    \usepackage{longtable}                                        %%
%    \usepackage{calc}                                             %%
%    \usepackage{multirow}                                         %%
%    \usepackage{hhline}                                           %%
%    \usepackage{ifthen}                                           %%
  %optionally (for landscape tables embedded in another document): %%
%    \usepackage{lscape}     
%\usepackage{multicol}
%\usepackage{chngcntr}
%\usepackage{enumerate}

%\usepackage{wasysym}
%\documentclass[conference]{IEEEtran}
%\IEEEoverridecommandlockouts
% The preceding line is only needed to identify funding in the first footnote. If that is unneeded, please comment it out.

\newtheorem{theorem}{Theorem}[section]
\newtheorem{problem}{Problem}
\newtheorem{proposition}{Proposition}[section]
\newtheorem{lemma}{Lemma}[section]
\newtheorem{corollary}[theorem]{Corollary}
\newtheorem{example}{Example}[section]
\newtheorem{definition}[problem]{Definition}
%\newtheorem{thm}{Theorem}[section] 
%\newtheorem{defn}[thm]{Definition}
%\newtheorem{algorithm}{Algorithm}[section]
%\newtheorem{cor}{Corollary}
\newcommand{\BEQA}{\begin{eqnarray}}
\newcommand{\EEQA}{\end{eqnarray}}
\newcommand{\define}{\stackrel{\triangle}{=}}
\theoremstyle{remark}
\newtheorem{rem}{Remark}

%\bibliographystyle{ieeetr}
\begin{document}
%

\bibliographystyle{IEEEtran}


\vspace{3cm}

\title{
%	\logo{
Solution of question xl-65.2023
%	}
}
\author{ Sameer kendal - EE22BTECH11044
	
}	
%\title{
%	\logo{Matrix Analysis through Octave}{\begin{center}\includegraphics[scale=.24]{tlc}\end{center}}{}{HAMDSP}
%}


% paper title
% can use linebreaks \\ within to get better formatting as desired
%\title{Matrix Analysis through Octave}
%
%
% author names and IEEE memberships
% note positions of commas and nonbreaking spaces ( ~ ) LaTeX will not break
% a structure at a ~ so this keeps an author's name from being broken across
% two lines.
% use \thanks{} to gain access to the first footnote area
% a separate \thanks must be used for each paragraph as LaTeX2e's \thanks
% was not built to handle multiple paragraphs
%

%\author{<-this % stops a space
%\thanks{}}
%}
% note the % following the last \IEEEmembership and also \thanks - 
% these prevent an unwanted space from occurring between the last author name
% and the end of the author line. i.e., if you had this:
% 
% \author{....lastname \thanks{...} \thanks{...} }
%                     ^------------^------------^----Do not want these spaces!
%
% a space would be appended to the last name and could cause every name on that
% line to be shifted left slightly. This is one of those "LaTeX things". For
% instance, "\textbf{A} \textbf{B}" will typeset as "A B" not "AB". To get
% "AB" then you have to do: "\textbf{A}\textbf{B}"
% \thanks is no different in this regard, so shield the last } of each \thanks
% that ends a line with a % and do not let a space in before the next \thanks.
% Spaces after \IEEEmembership other than the last one are OK (and needed) as
% you are supposed to have spaces between the names. For what it is worth,
% this is a minor point as most people would not even notice if the said evil
% space somehow managed to creep in.



% The paper headers
%\markboth{Journal of \LaTeX\ Class Files,~Vol.~6, No.~1, January~2007}%
%{Shell \MakeLowercase{\textit{et al.}}: Bare Demo of IEEEtran.cls for Journals}
% The only time the second header will appear is for the odd numbered pages
% after the title page when using the twoside option.
% 
% *** Note that you probably will NOT want to include the author's ***
% *** name in the headers of peer review papers.                   ***
% You can use \ifCLASSOPTIONpeerreview for conditional compilation here if
% you desire.




% If you want to put a publisher's ID mark on the page you can do it like
% this:
%\IEEEpubid{0000--0000/00\$00.00~\copyright~2007 IEEE}
% Remember, if you use this you must call \IEEEpubidadjcol in the second
% column for its text to clear the IEEEpubid mark.



% make the title area
\maketitle


%\tableofcontents

\bigskip

\renewcommand{\thefigure}{\theenumi}
\renewcommand{\thetable}{\theenumi}
%\renewcommand{\theequation}{\theenumi}

%\begin{abstract}
%%\boldmath
%In this letter, an algorithm for evaluating the exact analytical bit error rate  (BER)  for the piecewise linear (PL) combiner for  multiple relays is presented. Previous results were available only for upto three relays. The algorithm is unique in the sense that  the actual mathematical expressions, that are prohibitively large, need not be explicitly obtained. The diversity gain due to multiple relays is shown through plots of the analytical BER, well supported by simulations. 
%
%\end{abstract}
% IEEEtran.cls defaults to using nonbold math in the Abstract.
% This preserves the distinction between vectors and scalars. However,
% if the journal you are submitting to favors bold math in the abstract,
% then you can use LaTeX's standard command \boldmath at the very start
% of the abstract to achieve this. Many IEEE journals frown on math
% in the abstract anyway.

% Note that keywords are not normally used for peerreview papers.
%\begin{IEEEkeywords}
%Cooperative diversity, decode and forward, piecewise linear
%\end{IEEEkeywords}

Question: The frequencies for autosomal alleles $A$ and $a$ are $p = 0.5$ and $q = 0.5$,
respectively, where $A$ is dominant over $a$. Under the assumption of random
mating, the mating frequency among dominant parents is.\\
\fi
\solution
Given:
$A$ and $a$ are two alleles where $A$ is dominant one.\\
Let $Y$ be a random variable depicting the number of dominant alleles in zygote($AA,Aa,aA,aa$).
\begin{table}[!ht]
\input{2023/XL/65/tables/table1.tex}
\end{table}\\
\begin{enumerate}
\item Theory:\\ 
Using binomial which states that
\begin{align}
\pr{Y=i}= {}^{n}C_{i} (p)^i (q)^{(n-i)}
\end{align}

For the mating frquency among the dominant parents, both parents must have atleast one dominant allele.
The probability of getting atleast 1 dominant allele in parent zygote :\\
\begin{align}
\pr{Y \geq 1} &= \pr{Y=1}+ \pr{Y=2}\\
&= {}^{2}C_{1} (p) (q) + {}^{2}C_{2} (p)^2 (1)\\
&= 2pq + p^2\\
&= 0.5 + 0.25\\
&= 0.75
\end{align}
\item Step for Simulation of Random variable $Y$:\\
\begin{enumerate}
\item Define the simulation size for simulation data set.\\
\item Generate two different random distribution to get two different bernoulli showing $A$ as $1$ and $a$ as $0$ each having frequency of $0.5$.
\item The two bernoulli data for parents will mate with each other to form a zygote containig $1$'s and $0$'s.
\item Add up two bernouli to generate binomial distrubition for random variable $Y$ showing the number of $1$'s or dominant allele in zygote.
\item Count up the cases where there is atleast one $1$ to generate the simulated probability.
\end{enumerate}
\end{enumerate}
\begin{figure}[!ht]
\centering
\includegraphics[width=\columnwidth]{2023/XL/65/figs/fig.png}
\end{figure}





\item Let $X$ be a random variable having poisson distribution with mean $\lambda>0$. Then E\brak{\cond{\frac{1}{1+X}}{X>0}} equals
\begin{enumerate}
	\item $\frac{1-e^{-\lambda}-\lambda e^{-\lambda}}{\lambda\brak{1-e^{-\lambda}}}$
	\item $\frac{1-e^{-\lambda}}{\lambda}$
	\item $\frac{1-e^{-\lambda}-\lambda e^{-\lambda}}{\lambda}$
	\item $\frac{1-e^{-\lambda}}{\lambda +1}$
\end{enumerate}
\hfill(GATE ST 2023)\\
\solution\\
\iffalse
\documentclass[journal,11pt,twocolumn]{IEEEtran}
\usepackage{setspace}
\usepackage{gensymb}
\singlespacing
\usepackage[cmex10]{amsmath}
\usepackage{amsthm}
\usepackage{mathrsfs}
\usepackage{txfonts}
\usepackage{stfloats}
\usepackage{bm}
\usepackage{cite}
\usepackage{cases}
\usepackage{subfig}
\usepackage{longtable}
\usepackage{multirow}
\usepackage{enumitem}
\usepackage{mathtools}
\usepackage{tikz}
\usepackage{circuitikz}
\usepackage{verbatim}
\usepackage[breaklinks=true]{hyperref}
\usepackage{tkz-euclide} % loads  TikZ and tkz-base
\usepackage{listings}
\usepackage{color}    
\usepackage{array}    
\usepackage{longtable}
\usepackage{calc}     
\usepackage{multirow} 
\usepackage{hhline}   
\usepackage{ifthen}   
\usepackage{lscape}     
\usepackage{chngcntr}
\usepackage{float}
\DeclareMathOperator*{\Res}{Res}
\renewcommand\thesection{\arabic{section}}
\renewcommand\thesubsection{\thesection.\arabic{subsection}}
\renewcommand\thesubsubsection{\thesubsection.\arabic{subsubsection}}

\renewcommand\thesectiondis{\arabic{section}}
\renewcommand\thesubsectiondis{\thesectiondis.\arabic{subsection}}
\renewcommand\thesubsubsectiondis{\thesubsectiondis.\arabic{subsubsection}}
\renewcommand\thetable{\arabic{table}}
% correct bad hyphenation here
\hyphenation{op-tical net-works semi-conduc-tor}
\def\inputGnumericTable{}                                 %%

\lstset{
%language=C,
frame=single, 
breaklines=true,
columns=fullflexible
}
%\lstset{
%language=tex,
%frame=single, 
%breaklines=true
%}
\providecommand{\pr}[1]{\ensuremath{\Pr\left(#1\right)}}
\providecommand{\prt}[2]{\ensuremath{p_{#1}^{\left(#2\right)} }}        % own macro for this question
\providecommand{\qfunc}[1]{\ensuremath{Q\left(#1\right)}}
\providecommand{\sbrak}[1]{\ensuremath{{}\left[#1\right]}}
\providecommand{\lsbrak}[1]{\ensuremath{{}\left[#1\right.}}
\providecommand{\rsbrak}[1]{\ensuremath{{}\left.#1\right]}}
\providecommand{\brak}[1]{\ensuremath{\left(#1\right)}}
\providecommand{\lbrak}[1]{\ensuremath{\left(#1\right.}}
\providecommand{\rbrak}[1]{\ensuremath{\left.#1\right)}}
\providecommand{\cbrak}[1]{\ensuremath{\left\{#1\right\}}}
\providecommand{\lcbrak}[1]{\ensuremath{\left\{#1\right.}}
\providecommand{\rcbrak}[1]{\ensuremath{\left.#1\right\}}}
\newcommand{\sgn}{\mathop{\mathrm{sgn}}}
\providecommand{\abs}[1]{\left\vert#1\right\vert}
\providecommand{\res}[1]{\Res\displaylimits_{#1}} 
\providecommand{\norm}[1]{\left\lVert#1\right\rVert}
%\providecommand{\norm}[1]{\lVert#1\rVert}
\providecommand{\mtx}[1]{\mathbf{#1}}
\providecommand{\mean}[1]{E\left[ #1 \right]}
\providecommand{\cond}[2]{#1\middle|#2}
\providecommand{\fourier}{\overset{\mathcal{F}}{ \rightleftharpoons}}
%\providecommand{\hilbert}{\overset{\mathcal{H}}{ \rightleftharpoons}}
%\providecommand{\system}{\overset{\mathcal{H}}{ \longleftrightarrow}}
	%\newcommand{\solution}[2]{\textbf{Solution:}{#1}}
\newcommand{\solution}{\noindent \textbf{Solution: }}
\newcommand{\cosec}{\,\text{cosec}\,}
\providecommand{\dec}[2]{\ensuremath{\overset{#1}{\underset{#2}{\gtrless}}}}
\newcommand{\myvec}[1]{\ensuremath{\begin{pmatrix}#1\end{pmatrix}}}
\newcommand{\mydet}[1]{\ensuremath{\begin{vmatrix}#1\end{vmatrix}}}
\providecommand{\rank}{\text{rank}}
\providecommand{\pr}[1]{\ensuremath{\Pr\left(#1\right)}}
\providecommand{\qfunc}[1]{\ensuremath{Q\left(#1\right)}}
	\newcommand*{\permcomb}[4][0mu]{{{}^{#3}\mkern#1#2_{#4}}}
\newcommand*{\perm}[1][-3mu]{\permcomb[#1]{P}}
\newcommand*{\comb}[1][-1mu]{\permcomb[#1]{C}}
\providecommand{\qfunc}[1]{\ensuremath{Q\left(#1\right)}}
\providecommand{\gauss}[2]{\mathcal{N}\ensuremath{\left(#1,#2\right)}}
\providecommand{\diff}[2]{\ensuremath{\frac{d{#1}}{d{#2}}}}
\providecommand{\myceil}[1]{\left \lceil #1 \right \rceil }
\newcommand\figref{Fig.~\ref}
\newcommand\tabref{Table~\ref}
\newcommand{\sinc}{\,\text{sinc}\,}
\newcommand{\rect}{\,\text{rect}\,}
\title{Assignment}
\author{Barath surya M | EE22BTECH11014}
\begin{document}
\newtheorem{theorem}{Theorem}[section]
\newtheorem{problem}{Problem}
\newtheorem{proposition}{Proposition}[section]
\newtheorem{lemma}{Lemma}[section]
\newtheorem{corollary}[theorem]{Corollary}
\newtheorem{example}{Example}[section]
\newtheorem{definition}[problem]{Definition}
\newcommand{\BEQA}{\begin{eqnarray}}
\newcommand{\EEQA}{\end{eqnarray}}
\newcommand{\define}{\stackrel{\triangle}{=}}
\bibliographystyle{IEEEtran}
\providecommand{\mbf}{\mathbf}
\providecommand{\pr}[1]{\ensuremath{\Pr\left(#1\right)}}
\providecommand{\qfunc}[1]{\ensuremath{Q\left(#1\right)}}
\providecommand{\sbrak}[1]{\ensuremath{{}\left[#1\right]}}
\providecommand{\lsbrak}[1]{\ensuremath{{}\left[#1\right.}}
\providecommand{\rsbrak}[1]{\ensuremath{{}\left.#1\right]}}
\providecommand{\brak}[1]{\ensuremath{\left(#1\right)}}
\providecommand{\lbrak}[1]{\ensuremath{\left(#1\right.}}
\providecommand{\rbrak}[1]{\ensuremath{\left.#1\right)}}
\providecommand{\cbrak}[1]{\ensuremath{\left\{#1\right\}}}
\providecommand{\lcbrak}[1]{\ensuremath{\left\{#1\right.}}
\providecommand{\rcbrak}[1]{\ensuremath{\left.#1\right\}}}
\theoremstyle{remark}
\newtheorem{rem}{Remark}
\providecommand{\abs}[1]{\left\vert#1\right\vert}
\providecommand{\res}[1]{\Res\displaylimits_{#1}} 
\providecommand{\norm}[1]{\left\lVert#1\right\rVert}
\providecommand{\mtx}[1]{\mathbf{#1}}
\providecommand{\mean}[1]{E\left[ #1 \right]}
\providecommand{\fourier}{\overset{\mathcal{F}}{ \rightleftharpoons}}
\providecommand{\system}[1]{\overset{\mathcal{#1}}{ \longleftrightarrow}}
\providecommand{\dec}[2]{\ensuremath{\overset{#1}{\underset{#2}{\gtrless}}}}
\let\vec\mathbf
\def\putbox#1#2#3{\makebox[0in][l]{\makebox[#1][l]{}\raisebox{\baselineskip}[0in][0in]{\raisebox{#2}[0in][0in]{#3}}}}
     \def\rightbox#1{\makebox[0in][r]{#1}}
     \def\centbox#1{\makebox[0in]{#1}}
     \def\topbox#1{\raisebox{-\baselineskip}[0in][0in]{#1}}
     \def\midbox#1{\raisebox{-0.5\baselineskip}[0in][0in]{#1}}
\maketitle
Question Let $X$ be a random variable having poisson distribution with mean $\lambda>0$. Then E\brak{\cond{\frac{1}{1+X}}{X>0}} equals
\begin{enumerate}
	\item $\frac{1-e^{-\lambda}-\lambda e^{-\lambda}}{\lambda\brak{1-e^{-\lambda}}}$
	\item $\frac{1-e^{-\lambda}}{\lambda}$
	\item $\frac{1-e^{-\lambda}-\lambda e^{-\lambda}}{\lambda}$
	\item $\frac{1-e^{-\lambda}}{\lambda +1}$
\end{enumerate}

\solution
\\
\fi
\begin{enumerate}[label=(\Alph*)]
	\item Theory
\begin{align}
	X&\sim Pois\brak{\lambda}\\
	\pr{X=k}&=e^{-\lambda} \frac{\lambda^k}{k!}; k \geq 0 \label{eq:poissondistributionpmf}
\end{align}
we know that 
\begin{align}
	E\brak{A|B}&=\frac{E\brak{A,B}}{\pr{B}}\\
	\implies E\brak{\cond{\frac{1}{1+X}}{X>0}} &=\frac{\sum_{k=1}^{\infty}\frac{1}{k+1}\pr{X=k}}{\pr{X>0}}\\
	&=\frac{\sum_{k=1}^{\infty}\frac{1}{k+1} e^{-\lambda}\frac{\lambda^k}{k!}}{1-\pr{X\leq 0}}\\
	&=\frac{e^{-\lambda}\sum_{k=1}^{\infty}\frac{\lambda^k}{\brak{k+1}!}}{1-\pr{X=0}}
\end{align}
from equation \eqref{eq:poissondistributionpmf}
\begin{align}
	&=\frac{e^{-\lambda}\sum_{k=1}^{\infty}\frac{\lambda^k}{\brak{k+1}!}}{1-e^{\lambda}}
\end{align}
Now simplifying Just the Summation
\begin{align}
	\implies & \sum_{k=1}^{\infty} \frac{\lambda^k}{\brak{k+1}!}\\
	&=\frac{1}{\lambda} \sum_{k=1}^{\infty} \frac{\lambda^{k+1}}{k+1}
\end{align}
Letting $k+1 =m $,
\begin{align}
	\implies & \frac{1}{\lambda} \sum_{m=2}^{\infty} \frac{\lambda^m}{m!}
\end{align}
We Know from Taylor series
\begin{align}
	e^x&=\sum_{k=0}^{\infty} \frac{x^k}{k!}\\
	\implies \frac{1}{\lambda} \sum_{m=2}^{\infty} \frac{\lambda^m}{m!}&= \frac{1}{\lambda}\brak{e^{\lambda}-1-\lambda}
\end{align}
Substituting back we get,
\begin{align}
	&=\frac{e^{-\lambda}}{ \brak{1-e^{\lambda}}}\brak{\frac{1}{\lambda} \brak{e^{\lambda}-1-\lambda}}\\
	&=\frac{e^{-\lambda}}{\lambda \brak{1-e^{\lambda}}}\brak{e^{\lambda}-1 -\lambda}\\
	&=\frac{1-e^{-\lambda}-\lambda e^{-\lambda}}{\lambda \brak{1-e^{\lambda}}}
\end{align}
\item Simulation\\
\begin{enumerate}[label=(\roman*)]
	\item In the code, it simulates the generation of RV $X$ using the CDF of Poisson distribution.\\
	\begin{align}
		F\brak{x}&=\sum_{k=0}^{x}e^{-\lambda}\frac{\lambda^k}{k!}
	\end{align}
	\item initialize $X=0$
	\item initialize $F=e^{-\lambda}$ which is $F\brak{0}$ of a poisson distribution
	\item generate a uniform random variable between 0 and 1
	\item enter the loop that continues as long as $u>F$\\
	Inside the loop,
	\item Increment $X$ by 1
	\item Add next term of Poisson pmf $\brak{\pr{X=k+1}}$ to $F\brak{x}$
	\item the loop continues until $u$ is no longer greater than $F$. At this point, X represents the generated value of the poisson random variable that follows the desired poisson distribution with mean parameter $\lambda$.
	\item Save all the values of poisson random variable $X$ in pois.dat so to open it in python an plot the cdf graph 
	\item Then for the second part Check if the generated value of $X$ is greater than 0. If $X$ is greater than 0, calculate the value $Y$ as $\frac{1}{X + 1}$ and add it to the sumY.Increment the validCount to keep track of valid $X$ values.
	\item If validCount is greater than 0, calculate the  estimate of the conditional expectation by dividing sumY by validCount.

\end{enumerate}
\end{enumerate}
\begin{figure}[ht!]
	\includegraphics[width=\columnwidth]{2023/ST/17/figs/poisson.png}
	\label{gate17_2023_poissoncdf}
	\caption{Simulation Vs theoretical cdf of poisson distribution with $\lambda=9$}
\end{figure}


\item Consider the probability space $(\Omega, \mathcal{G}, P)$, where 
   $ \Omega = \{1, 2, 3, 4\}$, 
    $\mathcal{G} = \{\emptyset, \Omega, \{1\}, \{4\}, \{2, 3\}, \{1, 4\}, \{1, 2, 3\}, \{2, 3, 4\}\}$, 
    $P(\{1\}) = \frac{1}{4}$.
Let $X$ be the random variable defined on the above probability space as
   $ X(1) = 1$, 
    $X(2) = X(3) = 2$, 
    $X(4) = 3$.
If $P(X \leq 2) = \frac{3}{4}$, then find $P(\{1, 4\})$ (rounded off to two decimal places).\\\hfill (GATE ST 2023)\\
\iffalse
\documentclass[journal,11pt,onecolumn]{IEEEtran}
\usepackage{setspace}
\usepackage{gensymb}
\singlespacing
\usepackage[cmex10]{amsmath}
\usepackage{amsthm}
\usepackage{mathrsfs}
\usepackage{txfonts}
\usepackage{stfloats}
\usepackage{bm}
\usepackage{cite}
\usepackage{cases}
\usepackage{subfig}
\usepackage{longtable}
\usepackage{multirow}
\usepackage{enumitem}
\usepackage{mathtools}
\usepackage{tikz}
\usepackage{circuitikz}
\usepackage{verbatim}
\usepackage[breaklinks=true]{hyperref}
\usepackage{tkz-euclide} % loads  TikZ and tkz-base
\usepackage{listings}
\usepackage{color}    
\usepackage{array}    
\usepackage{longtable}
\usepackage{calc}     
\usepackage{multirow} 
\usepackage{hhline}   
\usepackage{ifthen}   
\usepackage{lscape}     
\usepackage{chngcntr}
\usepackage{float}
\DeclareMathOperator*{\Res}{Res}
\renewcommand\thesection{\arabic{section}}
\renewcommand\thesubsection{\thesection.\arabic{subsection}}
\renewcommand\thesubsubsection{\thesubsection.\arabic{subsubsection}}

\renewcommand\thesectiondis{\arabic{section}}
\renewcommand\thesubsectiondis{\thesectiondis.\arabic{subsection}}
\renewcommand\thesubsubsectiondis{\thesubsectiondis.\arabic{subsubsection}}
\renewcommand\thetable{\arabic{table}}
% correct bad hyphenation here
\hyphenation{op-tical net-works semi-conduc-tor}
\def\inputGnumericTable{}                                 %%

\lstset{
%language=C,
frame=single, 
breaklines=true,
columns=fullflexible
}
%\lstset{
%language=tex,
%frame=single, 
%breaklines=true
%}
\providecommand{\pr}[1]{\ensuremath{\Pr\left(#1\right)}}
\providecommand{\prt}[2]{\ensuremath{p_{#1}^{\left(#2\right)} }}        % own macro for this question
\providecommand{\qfunc}[1]{\ensuremath{Q\left(#1\right)}}
\providecommand{\sbrak}[1]{\ensuremath{{}\left[#1\right]}}
\providecommand{\lsbrak}[1]{\ensuremath{{}\left[#1\right.}}
\providecommand{\rsbrak}[1]{\ensuremath{{}\left.#1\right]}}
\providecommand{\brak}[1]{\ensuremath{\left(#1\right)}}
\providecommand{\lbrak}[1]{\ensuremath{\left(#1\right.}}
\providecommand{\rbrak}[1]{\ensuremath{\left.#1\right)}}
\providecommand{\cbrak}[1]{\ensuremath{\left\{#1\right\}}}
\providecommand{\lcbrak}[1]{\ensuremath{\left\{#1\right.}}
\providecommand{\rcbrak}[1]{\ensuremath{\left.#1\right\}}}
\newcommand{\sgn}{\mathop{\mathrm{sgn}}}
\providecommand{\abs}[1]{\left\vert#1\right\vert}
\providecommand{\res}[1]{\Res\displaylimits_{#1}} 
\providecommand{\norm}[1]{\left\lVert#1\right\rVert}
%\providecommand{\norm}[1]{\lVert#1\rVert}
\providecommand{\mtx}[1]{\mathbf{#1}}
\providecommand{\mean}[1]{E\left[ #1 \right]}
\providecommand{\cond}[2]{#1\middle|#2}
\providecommand{\fourier}{\overset{\mathcal{F}}{ \rightleftharpoons}}
%\providecommand{\hilbert}{\overset{\mathcal{H}}{ \rightleftharpoons}}
%\providecommand{\system}{\overset{\mathcal{H}}{ \longleftrightarrow}}
	%\newcommand{\solution}[2]{\textbf{Solution:}{#1}}
\newcommand{\solution}{\noindent \textbf{Solution: }}
\newcommand{\cosec}{\,\text{cosec}\,}
\providecommand{\dec}[2]{\ensuremath{\overset{#1}{\underset{#2}{\gtrless}}}}
\newcommand{\myvec}[1]{\ensuremath{\begin{pmatrix}#1\end{pmatrix}}}
\newcommand{\mydet}[1]{\ensuremath{\begin{vmatrix}#1\end{vmatrix}}}
\providecommand{\rank}{\text{rank}}
\providecommand{\pr}[1]{\ensuremath{\Pr\left(#1\right)}}
\providecommand{\qfunc}[1]{\ensuremath{Q\left(#1\right)}}
	\newcommand*{\permcomb}[4][0mu]{{{}^{#3}\mkern#1#2_{#4}}}
\newcommand*{\perm}[1][-3mu]{\permcomb[#1]{P}}
\newcommand*{\comb}[1][-1mu]{\permcomb[#1]{C}}
\providecommand{\qfunc}[1]{\ensuremath{Q\left(#1\right)}}
\providecommand{\gauss}[2]{\mathcal{N}\ensuremath{\left(#1,#2\right)}}
\providecommand{\diff}[2]{\ensuremath{\frac{d{#1}}{d{#2}}}}
\providecommand{\myceil}[1]{\left \lceil #1 \right \rceil }
\newcommand\figref{Fig.~\ref}
\newcommand\tabref{Table~\ref}
\newcommand{\sinc}{\,\text{sinc}\,}
\newcommand{\rect}{\,\text{rect}\,}
\title{Assignment}
\author{dushyant | EE22BTECH11031}
\begin{document}
\newtheorem{theorem}{Theorem}[section]
\newtheorem{problem}{Problem}
\newtheorem{proposition}{Proposition}[section]
\newtheorem{lemma}{Lemma}[section]
\newtheorem{corollary}[theorem]{Corollary}
\newtheorem{example}{Example}[section]
\newtheorem{definition}[problem]{Definition}
\newcommand{\BEQA}{\begin{eqnarray}}
\newcommand{\EEQA}{\end{eqnarray}}
\newcommand{\define}{\stackrel{\triangle}{=}}
\bibliographystyle{IEEEtran}
\providecommand{\mbf}{\mathbf}
\providecommand{\pr}[1]{\ensuremath{\Pr\left(#1\right)}}
\providecommand{\qfunc}[1]{\ensuremath{Q\left(#1\right)}}
\providecommand{\sbrak}[1]{\ensuremath{{}\left[#1\right]}}
\providecommand{\lsbrak}[1]{\ensuremath{{}\left[#1\right.}}
\providecommand{\rsbrak}[1]{\ensuremath{{}\left.#1\right]}}
\providecommand{\brak}[1]{\ensuremath{\left(#1\right)}}
\providecommand{\lbrak}[1]{\ensuremath{\left(#1\right.}}
\providecommand{\rbrak}[1]{\ensuremath{\left.#1\right)}}
\providecommand{\cbrak}[1]{\ensuremath{\left\{#1\right\}}}
\providecommand{\lcbrak}[1]{\ensuremath{\left\{#1\right.}}
\providecommand{\rcbrak}[1]{\ensuremath{\left.#1\right\}}}
\theoremstyle{remark}
\newtheorem{rem}{Remark}
\providecommand{\abs}[1]{\left\vert#1\right\vert}
\providecommand{\res}[1]{\Res\displaylimits_{#1}} 
\providecommand{\norm}[1]{\left\lVert#1\right\rVert}
\providecommand{\mtx}[1]{\mathbf{#1}}
\providecommand{\mean}[1]{E\left[ #1 \right]}
\providecommand{\fourier}{\overset{\mathcal{F}}{ \rightleftharpoons}}
\providecommand{\system}[1]{\overset{\mathcal{#1}}{ \longleftrightarrow}}
\providecommand{\dec}[2]{\ensuremath{\overset{#1}{\underset{#2}{\gtrless}}}}
\let\vec\mathbf
\def\putbox#1#2#3{\makebox[0in][l]{\makebox[#1][l]{}\raisebox{\baselineskip}[0in][0in]{\raisebox{#2}[0in][0in]{#3}}}}
     \def\rightbox#1{\makebox[0in][r]{#1}}
     \def\centbox#1{\makebox[0in]{#1}}
     \def\topbox#1{\raisebox{-\baselineskip}[0in][0in]{#1}}
     \def\midbox#1{\raisebox{-0.5\baselineskip}[0in][0in]{#1}}
\maketitle
\textbf{Question:}Consider the probability space $(\Omega, \mathcal{G}, P)$, where 
   $ \Omega = \{1, 2, 3, 4\}$, 
    $\mathcal{G} = \{\emptyset, \Omega, \{1\}, \{4\}, \{2, 3\}, \{1, 4\}, \{1, 2, 3\}, \{2, 3, 4\}\}$, 
    $P(\{1\}) = \frac{1}{4}$.
Let $X$ be the random variable defined on the above probability space as
   $ X(1) = 1$, 
    $X(2) = X(3) = 2$, 
    $X(4) = 3$.
If $P(X \leq 2) = \frac{3}{4}$, then find $P(\{1, 4\})$ (rounded off to two decimal places).\\\hfill (GATE ST 2023)\\
\fi
\solution
\begin{table}[ht]
\centering
\caption{Probablity space}
\label{tab:2023/ST/60_1}
\begin{tabular}{|c|c|c}
\hline
Probablity space &Value \\ \hline
$\Omega$ & $\{1, 2, 3, 4\}$\\\hline
$\mathcal{G}$ &$\{\emptyset, \Omega, \{1\}, \{4\}, \{2, 3\}, \{1, 4\}, \{1, 2, 3\}, \{2, 3, 4\}\}$\\\hline
$P(\{1\})$ &$\frac{1}{4}$\\\hline
$P(X \leq 2)$ & $\frac{3}{4}$\\\hline
\end{tabular}
\end{table}
\\
\begin{table}[ht]
\centering
\caption{Random variable}
    \label{tab:2023/ST/60_2}
\begin{tabular}{|c|c|c}
\hline
$X\brak{\Omega}$ & $\Omega$\\\hline
$\{1\}$ & 1\\\hline
$\{2,3\}$ &2\\\hline
$\{4\}$ & 3\\\hline
\end{tabular}
\end{table}
\\
Pmf is defined as\\
\begin{align}
p_x\brak{k} &= \begin{cases}
P(\{1\}) & ,k=1\\
P(\{2,3\}) &, k=2\\
P(\{4\}) &, k=3\\
\end{cases}
\end{align}
Values of P(\{2,3\}), P(\{4\}) are unknown, so let p, q be their respective values
\begin{align}
p_X\brak{k} &= \begin{cases}
\frac{1}{4} & ,k=1\\
p &, k=2\\
q &, k=3\\
\end{cases}
\end{align}
\begin{align}
\Pr{(\{1, 4\})} = p_X(1) +p_X(3)
\end{align}
We know\\
\begin{align}
p_X(1) +  p + q  &= 1
\end{align}
We can express Pr($X \leq 2$)as:
\begin{align}
\Pr{(X \leq 2)} &= p_X(1) + p\\
\end{align}
We can expres above equations as:
\begin{align}
\myvec{1 & 1
        \\1 & 0}\myvec{p \\q} = \myvec{\frac{3}{4}\\ \frac{1}{2}}
\end{align}
\begin{align}
p &= \frac{1}{2}, q =\frac{1}{4}
\end{align}
Finally
\begin{align}
\Pr{(\{1, 4\})} &= P(\{1\}) +q\\
\Pr{(\{1, 4\})} &= \frac{1}{4} + \frac{1}{4}\\
\Pr{(\{1, 4\})} &= 0.5
\end{align}
\begin{figure}[h]
\includegraphics[width=\columnwidth]{2023/ST/60/figs/main2.png}
\caption{Analytical vs simulated}
\label{fig:2023/ST/60_3}
\end{figure}\\
Steps for simulating random variable.\\
\begin{enumerate}
\item Define the simulation size for datast (samples).
\item Assign calculated probablity for each probablity space p1, p2, p3, p4.
\item Define Random to generate a random number between 0 and 1.
\item Define the loop such that it generated number 1, 2, 3 for defined probablity space.
\item Store the simulated data in a .dat file.
\item Using matplotlib lib of python generate a V-line graph from the data in .dat file by counting the number of 1, 2, 3 .
\end{enumerate}

 

Let ${N(t)}_{t\ge 0}$ be a Poisson process with rate 1. Consider the following statements. 
\begin{enumerate}[label=(\alph*)]
\item $\pr{N(3)=3|N(5)=5}=\comb{5}{3}\left(\frac{3}{5}\right)^3 \left(\frac{2}{5}\right)^2$
\item If $S_5$ denotes the time of occurrence of the $5^{th}$ event for the above Poisson process,then $E(S_5|N(5)=3)=7$ \\
Which of the above statements is/are true?\\
\end{enumerate}
\begin{enumerate}[label=(\roman*)]
\item only (a)
\item only (b)
\item Both (a) and (b)
\item Neither (a) and (b)
\end{enumerate}
\hfill ( GATE ST 2023)\\
\solution \\
\iffalse
\let\negmedspace\undefined
\let\negthickspace\undefined
\documentclass[journal,12pt,twocolumn]{IEEEtran}
\usepackage{cite}
\usepackage{amsmath,amssymb,amsfonts,amsthm}
\usepackage{algorithmic}
\usepackage{graphicx}
\usepackage{textcomp}
\usepackage{xcolor}
\usepackage{txfonts}
\usepackage{listings}
%\usepackage{enumitem}
\usepackage{mathtools}
\usepackage{gensymb}
\usepackage[breaklinks=true]{hyperref}
\usepackage{tkz-euclide} % loads  TikZ and tkz-base
\usepackage{listings}
\usepackage[inline]{enumitem}
\DeclareMathOperator*{\Res}{Res}
\renewcommand\thesection{\arabic{section}}
\renewcommand\thesubsection{\thesection.\arabic{subsection}}
\renewcommand\thesubsubsection{\thesubsection.\arabic{subsubsection}}


\def\inputGnumericTable{}

\usepackage[latin1]{inputenc}                                 
\usepackage{color}                                            
\usepackage{array}                                            
\usepackage{longtable}                                        
\usepackage{calc}                                             
\usepackage{multirow}                                         
\usepackage{hhline}                                           
\usepackage{ifthen}
\usepackage{caption} 
\captionsetup[table]{skip=3pt}  
\providecommand{\pr}[1]{\ensuremath{\Pr\left(#1\right)}}
\providecommand{\cbrak}[1]{\ensuremath{\left\{#1\right\}}}

\renewcommand\thesectiondis{\arabic{section}}
\renewcommand\thesubsectiondis{\thesectiondis.\arabic{subsection}}
\renewcommand\thesubsubsectiondis{\thesubsectiondis.\arabic{subsubsection}}

\def\inputGnumericTable{}                                 %%

\lstset{
frame=single, 
breaklines=true,
columns=fullflexible
}

\begin{document}

\newtheorem{theorem}{Theorem}[section]
\newtheorem{problem}{Problem}
\newtheorem{proposition}{Proposition}[section]
\newtheorem{lemma}{Lemma}[section]
\newtheorem{corollary}[theorem]{Corollary}
\newtheorem{example}{Example}[section]
\newtheorem{definition}[problem]{Definition}
\newcommand{\BEQA}{\begin{eqnarray}}
\newcommand{\EEQA}{\end{eqnarray}}
\newcommand{\define}{\stackrel{\triangle}{=}}
\newcommand{\xor}{\oplus}
\bibliographystyle{IEEEtran}

\providecommand{\mbf}{\mathbf}
\providecommand{\pr}[1]{\ensuremath{\Pr\left(#1\right)}}
\providecommand{\qfunc}[1]{\ensuremath{Q\left(#1\right)}}
\providecommand{\sbrak}[1]{\ensuremath{{}\left[#1\right]}}
\providecommand{\lsbrak}[1]{\ensuremath{{}\left[#1\right.}}
\providecommand{\rsbrak}[1]{\ensuremath{{}\left.#1\right]}}
\providecommand{\brak}[1]{\ensuremath{\left(#1\right)}}
\providecommand{\lbrak}[1]{\ensuremath{\left(#1\right.}}
\providecommand{\rbrak}[1]{\ensuremath{\left.#1\right)}}
\providecommand{\cbrak}[1]{\ensuremath{\left\{#1\right\}}}
\providecommand{\lcbrak}[1]{\ensuremath{\left\{#1\right.}}
\providecommand{\rcbrak}[1]{\ensuremath{\left.#1\right\}}}
\theoremstyle{remark}
\newtheorem{rem}{Remark}
\newcommand{\sgn}{\mathop{\mathrm{sgn}}}

\newcommand{\solution}{\noindent \textbf{Solution: }}
\newcommand{\cosec}{\,\text{cosec}\,}
\providecommand{\dec}[2]{\ensuremath{\overset{#1}{\underset{#2}{\gtrless}}}}
\newcommand{\myvec}[1]{\ensuremath{\begin{pmatrix}#1\end{pmatrix}}}
\newcommand{\mydet}[1]{\ensuremath{\begin{vmatrix}#1\end{vmatrix}}}

\let\vec\mathbf


\vspace{3cm}

\title{
  

  Assignment -5 in \LaTeX
    
  }
  \author{ Muzaan Mohammed Faizel A P\\
  EE22BTECH11036
  }	
% make the title area
\maketitle
\newpage
\bigskip
\renewcommand{\thefigure}{\theenumi}
\renewcommand{\thetable}{\theenumi}
\renewcommand{\thetable}{\arabic{table}} 
\textbf{GATE 2023 BM QN.12}
For a Binomial random variable $X$, E($X$) and Var($X$) are the expectation and
variance, respectively. Which one of the following statements CANNOT be true?
\begin{table}[ht!]
		\centering
		\input{2023/BM/12/tables/table1.tex}
		\caption{}
		\label{table:table1}	
\end{table}
\fi
\solution
\begin{align}
X \sim \text{Bin}\brak{n,p} \nonumber
\end{align}

We know ,
\begin{align} 
	E\brak{X}=np\\
	Var\brak{X}=np\brak{1-p}\\
	0\leq p\leq1 \\
	\implies -1\leq -p \leq0\\
	\implies
	0 \leq 1-p \leq1\\
	\implies np\brak{1-p}\leq np
\end{align}
Therefore,
\begin{align} 
	Var\brak{X}\leq E\brak{X}
\end{align}
From the four options,the statement that cannot be true is option \brak{3}\\

\textbf{Simulation steps}\\
\textbf{Step 1: Generate a Range of Probabilities}\\
The program generates a range of probabilities (\(p\)) in increments of 0.0098, ranging from 0.01 to 0.99. This range is used to create different binomial distributions for subsequent calculations.
\\
% It then calculates the variance of the generated binomial random variable.\\
\textbf{Step 2: Generating binomial r.v from Uniform distribution}\\
\( U \sim \text{Uniform}(0, 1) \).

Defining a random variable X as:
\[
X = 
\begin{cases}
    1 & \text{if } U > p \\
    0 & \text{otherwise}
\end{cases}
\]
This becomes a Bernoulli rv.
The count variable evaluates the Binomial r.v by the summation of Bernoulli r.v

\[
count= \sum_{i=1}^{n} X
\]
\textbf{Step 3: Calculate Variances for Each Probability}\\
For each probability (\(p\)) in the generated range, the program generates a binomial random variable using the given mean and the inverse of the probability (\(mean/p\)) from uniform distribution.\\
\textbf{Step 4: Find Maximum and Minimum Variances}\\
After calculating variances for each probability, the program identifies the maximum and minimum variances in the generated set.\\
\textbf{Step 5: User Input}\\
The program prompts the user to input a variance value for validation.
\\
\textbf{Step 6: Check Validity}\\
The user-input variance is compared against the computed maximum and minimum variances. If the input variance falls within this range (inclusive of the minimum and exclusive of the maximum), the program outputs "Valid." Otherwise, it outputs "Invalid."
\begin{figure}[ht!]
    \centering
    \includegraphics[width=\columnwidth]{2023/BM/12/codes/ss10.png}
    \caption{Variance for mean=10}
    \label{fig:ss10}
\end{figure}

	%\includegraphics[width=\columnwidth]{./codes/ss10.png}
	

	

\end{enumerate}
