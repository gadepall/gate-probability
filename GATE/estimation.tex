\begin{enumerate}[label=\thechapter.\arabic*,ref=\thechapter.\theenumi]
\item Let $\{-1, -\frac{1}{2}, 1, \frac{5}{2}, 3\}$ be a realization of a random sample of size $5$ from a population having $N\left(\frac{1}{2}, \sigma^2\right)$ distribution, where $\sigma > 0$ is an unknown parameter. Let $T$ be an unbiased estimator of $\sigma^2$ whose variance attains the Cramer-Rao lower bound. Then, based on the above data, the realized value of $T$ (rounded off to two decimal places) equals
\hfill (GATE ST 2023)
\iffalse
\let\negmedspace\undefined
\let\negthickspace\undefined
\documentclass[journal,12pt,twocolumn]{IEEEtran}
\usepackage{cite}
\usepackage{amsmath,amssymb,amsfonts,amsthm}
\usepackage{algorithmic}
\usepackage{graphicx}
\usepackage{textcomp}
\usepackage{xcolor}
\usepackage{txfonts}
\usepackage{listings}
%\usepackage{enumitem}
\usepackage{mathtools}
\usepackage{gensymb}
\usepackage[breaklinks=true]{hyperref}
\usepackage{tkz-euclide} % loads  TikZ and tkz-base
\usepackage{listings}
\usepackage[inline]{enumitem}
\DeclareMathOperator*{\Res}{Res}
\renewcommand\thesection{\arabic{section}}
\renewcommand\thesubsection{\thesection.\arabic{subsection}}
\renewcommand\thesubsubsection{\thesubsection.\arabic{subsubsection}}


\def\inputGnumericTable{}

\usepackage[latin1]{inputenc}                                 
\usepackage{color}                                            
\usepackage{array}                                            
\usepackage{longtable}                                        
\usepackage{calc}                                             
\usepackage{multirow}                                         
\usepackage{hhline}                                           
\usepackage{ifthen}
\usepackage{caption} 
\captionsetup[table]{skip=3pt}  
\providecommand{\pr}[1]{\ensuremath{\Pr\left(#1\right)}}
\providecommand{\cbrak}[1]{\ensuremath{\left\{#1\right\}}}

\renewcommand\thesectiondis{\arabic{section}}
\renewcommand\thesubsectiondis{\thesectiondis.\arabic{subsection}}
\renewcommand\thesubsubsectiondis{\thesubsectiondis.\arabic{subsubsection}}

\def\inputGnumericTable{}                                 %%

\lstset{
frame=single, 
breaklines=true,
columns=fullflexible
}

\begin{document}

\newtheorem{theorem}{Theorem}[section]
\newtheorem{problem}{Problem}
\newtheorem{proposition}{Proposition}[section]
\newtheorem{lemma}{Lemma}[section]
\newtheorem{corollary}[theorem]{Corollary}
\newtheorem{example}{Example}[section]
\newtheorem{definition}[problem]{Definition}
\newcommand{\BEQA}{\begin{eqnarray}}
\newcommand{\EEQA}{\end{eqnarray}}
\newcommand{\define}{\stackrel{\triangle}{=}}
\newcommand{\xor}{\oplus}
\bibliographystyle{IEEEtran}

\providecommand{\mbf}{\mathbf}
\providecommand{\pr}[1]{\ensuremath{\Pr\left(#1\right)}}
\providecommand{\qfunc}[1]{\ensuremath{Q\left(#1\right)}}
\providecommand{\sbrak}[1]{\ensuremath{{}\left[#1\right]}}
\providecommand{\lsbrak}[1]{\ensuremath{{}\left[#1\right.}}
\providecommand{\rsbrak}[1]{\ensuremath{{}\left.#1\right]}}
\providecommand{\brak}[1]{\ensuremath{\left(#1\right)}}
\providecommand{\lbrak}[1]{\ensuremath{\left(#1\right.}}
\providecommand{\rbrak}[1]{\ensuremath{\left.#1\right)}}
\providecommand{\cbrak}[1]{\ensuremath{\left\{#1\right\}}}
\providecommand{\lcbrak}[1]{\ensuremath{\left\{#1\right.}}
\providecommand{\rcbrak}[1]{\ensuremath{\left.#1\right\}}}
\theoremstyle{remark}
\newtheorem{rem}{Remark}
\newcommand{\sgn}{\mathop{\mathrm{sgn}}}

\newcommand{\solution}{\noindent \textbf{Solution: }}
\newcommand{\cosec}{\,\text{cosec}\,}
\providecommand{\dec}[2]{\ensuremath{\overset{#1}{\underset{#2}{\gtrless}}}}
\newcommand{\myvec}[1]{\ensuremath{\begin{pmatrix}#1\end{pmatrix}}}
\newcommand{\mydet}[1]{\ensuremath{\begin{vmatrix}#1\end{vmatrix}}}

\let\vec\mathbf


\vspace{3cm}

\title{
  

  Assignment -5 in \LaTeX
    
  }
  \author{ Muzaan Mohammed Faizel A P\\
  EE22BTECH11036
  }	
% make the title area
\maketitle
\newpage
\bigskip
\renewcommand{\thefigure}{\theenumi}
\renewcommand{\thetable}{\theenumi}
\renewcommand{\thetable}{\arabic{table}} 
\textbf{GATE 2023 BM QN.12}
For a Binomial random variable $X$, E($X$) and Var($X$) are the expectation and
variance, respectively. Which one of the following statements CANNOT be true?
\begin{table}[ht!]
		\centering
		\input{2023/BM/12/tables/table1.tex}
		\caption{}
		\label{table:table1}	
\end{table}
\fi
\solution
\begin{align}
X \sim \text{Bin}\brak{n,p} \nonumber
\end{align}

We know ,
\begin{align} 
	E\brak{X}=np\\
	Var\brak{X}=np\brak{1-p}\\
	0\leq p\leq1 \\
	\implies -1\leq -p \leq0\\
	\implies
	0 \leq 1-p \leq1\\
	\implies np\brak{1-p}\leq np
\end{align}
Therefore,
\begin{align} 
	Var\brak{X}\leq E\brak{X}
\end{align}
From the four options,the statement that cannot be true is option \brak{3}\\

\textbf{Simulation steps}\\
\textbf{Step 1: Generate a Range of Probabilities}\\
The program generates a range of probabilities (\(p\)) in increments of 0.0098, ranging from 0.01 to 0.99. This range is used to create different binomial distributions for subsequent calculations.
\\
% It then calculates the variance of the generated binomial random variable.\\
\textbf{Step 2: Generating binomial r.v from Uniform distribution}\\
\( U \sim \text{Uniform}(0, 1) \).

Defining a random variable X as:
\[
X = 
\begin{cases}
    1 & \text{if } U > p \\
    0 & \text{otherwise}
\end{cases}
\]
This becomes a Bernoulli rv.
The count variable evaluates the Binomial r.v by the summation of Bernoulli r.v

\[
count= \sum_{i=1}^{n} X
\]
\textbf{Step 3: Calculate Variances for Each Probability}\\
For each probability (\(p\)) in the generated range, the program generates a binomial random variable using the given mean and the inverse of the probability (\(mean/p\)) from uniform distribution.\\
\textbf{Step 4: Find Maximum and Minimum Variances}\\
After calculating variances for each probability, the program identifies the maximum and minimum variances in the generated set.\\
\textbf{Step 5: User Input}\\
The program prompts the user to input a variance value for validation.
\\
\textbf{Step 6: Check Validity}\\
The user-input variance is compared against the computed maximum and minimum variances. If the input variance falls within this range (inclusive of the minimum and exclusive of the maximum), the program outputs "Valid." Otherwise, it outputs "Invalid."
\begin{figure}[ht!]
    \centering
    \includegraphics[width=\columnwidth]{2023/BM/12/codes/ss10.png}
    \caption{Variance for mean=10}
    \label{fig:ss10}
\end{figure}

	%\includegraphics[width=\columnwidth]{./codes/ss10.png}
	

	

\item Let $X$ be a random variable with probability density function
\begin{align}
\label{eq:22/2023/1}f(x;\lambda)&=
\begin{cases}
\frac{1}{\lambda}e^{-\frac{x}{\lambda}} & \text{if} x>0\\
0 & \text{otherwise}
\end{cases}
\end{align}
where $\lambda > 0$ is an unknown parameter. Let $Y_1, Y_2,...,Y_n$ be a random sample of
size $n$ from a population having the same distribution as $X^2$.If
\begin{align}
\label{eq:22/2023/2}\bar{Y} &= \frac{1}{n}\sum_{i=1}^n Y_i
\end{align}
then which of the following statements is true?
\begin{enumerate}
\item \label{eq:22/2023/3}$\sqrt{\frac{\bar{Y}}{2}} \text{is a method of moments estimator of}         \lambda$
\item $\sqrt{\bar{Y}} \text{is a method of moments estimator of}\lambda$
\item ${\frac{1}{2}\sqrt{\bar{Y}}} \text{is a method of moments estimator of }\lambda$
\item $2\sqrt{\bar{Y}} \text{is a method of moments estimator of} \lambda$
\hfill(GATE ST 2023)\\
\end{enumerate}
\iffalse
\let\negmedspace\undefined
\let\negthickspace\undefined
\documentclass[journal,12pt,twocolumn]{IEEEtran}
\usepackage{cite}
\usepackage{amsmath,amssymb,amsfonts,amsthm}
\usepackage{algorithmic}
\usepackage{graphicx}
\usepackage{textcomp}
\usepackage{xcolor}
\usepackage{txfonts}
\usepackage{listings}
%\usepackage{enumitem}
\usepackage{mathtools}
\usepackage{gensymb}
\usepackage[breaklinks=true]{hyperref}
\usepackage{tkz-euclide} % loads  TikZ and tkz-base
\usepackage{listings}
\usepackage[inline]{enumitem}
\DeclareMathOperator*{\Res}{Res}
\renewcommand\thesection{\arabic{section}}
\renewcommand\thesubsection{\thesection.\arabic{subsection}}
\renewcommand\thesubsubsection{\thesubsection.\arabic{subsubsection}}


\def\inputGnumericTable{}

\usepackage[latin1]{inputenc}                                 
\usepackage{color}                                            
\usepackage{array}                                            
\usepackage{longtable}                                        
\usepackage{calc}                                             
\usepackage{multirow}                                         
\usepackage{hhline}                                           
\usepackage{ifthen}
\usepackage{caption} 
\captionsetup[table]{skip=3pt}  
\providecommand{\pr}[1]{\ensuremath{\Pr\left(#1\right)}}
\providecommand{\cbrak}[1]{\ensuremath{\left\{#1\right\}}}

\renewcommand\thesectiondis{\arabic{section}}
\renewcommand\thesubsectiondis{\thesectiondis.\arabic{subsection}}
\renewcommand\thesubsubsectiondis{\thesubsectiondis.\arabic{subsubsection}}

\def\inputGnumericTable{}                                 %%

\lstset{
frame=single, 
breaklines=true,
columns=fullflexible
}

\begin{document}

\newtheorem{theorem}{Theorem}[section]
\newtheorem{problem}{Problem}
\newtheorem{proposition}{Proposition}[section]
\newtheorem{lemma}{Lemma}[section]
\newtheorem{corollary}[theorem]{Corollary}
\newtheorem{example}{Example}[section]
\newtheorem{definition}[problem]{Definition}
\newcommand{\BEQA}{\begin{eqnarray}}
\newcommand{\EEQA}{\end{eqnarray}}
\newcommand{\define}{\stackrel{\triangle}{=}}
\newcommand{\xor}{\oplus}
\bibliographystyle{IEEEtran}

\providecommand{\mbf}{\mathbf}
\providecommand{\pr}[1]{\ensuremath{\Pr\left(#1\right)}}
\providecommand{\qfunc}[1]{\ensuremath{Q\left(#1\right)}}
\providecommand{\sbrak}[1]{\ensuremath{{}\left[#1\right]}}
\providecommand{\lsbrak}[1]{\ensuremath{{}\left[#1\right.}}
\providecommand{\rsbrak}[1]{\ensuremath{{}\left.#1\right]}}
\providecommand{\brak}[1]{\ensuremath{\left(#1\right)}}
\providecommand{\lbrak}[1]{\ensuremath{\left(#1\right.}}
\providecommand{\rbrak}[1]{\ensuremath{\left.#1\right)}}
\providecommand{\cbrak}[1]{\ensuremath{\left\{#1\right\}}}
\providecommand{\lcbrak}[1]{\ensuremath{\left\{#1\right.}}
\providecommand{\rcbrak}[1]{\ensuremath{\left.#1\right\}}}
\theoremstyle{remark}
\newtheorem{rem}{Remark}
\newcommand{\sgn}{\mathop{\mathrm{sgn}}}

\newcommand{\solution}{\noindent \textbf{Solution: }}
\newcommand{\cosec}{\,\text{cosec}\,}
\providecommand{\dec}[2]{\ensuremath{\overset{#1}{\underset{#2}{\gtrless}}}}
\newcommand{\myvec}[1]{\ensuremath{\begin{pmatrix}#1\end{pmatrix}}}
\newcommand{\mydet}[1]{\ensuremath{\begin{vmatrix}#1\end{vmatrix}}}

\let\vec\mathbf


\vspace{3cm}

\title{
  

  Assignment -5 in \LaTeX
    
  }
  \author{ Muzaan Mohammed Faizel A P\\
  EE22BTECH11036
  }	
% make the title area
\maketitle
\newpage
\bigskip
\renewcommand{\thefigure}{\theenumi}
\renewcommand{\thetable}{\theenumi}
\renewcommand{\thetable}{\arabic{table}} 
\textbf{GATE 2023 BM QN.12}
For a Binomial random variable $X$, E($X$) and Var($X$) are the expectation and
variance, respectively. Which one of the following statements CANNOT be true?
\begin{table}[ht!]
		\centering
		\input{2023/BM/12/tables/table1.tex}
		\caption{}
		\label{table:table1}	
\end{table}
\fi
\solution
\begin{align}
X \sim \text{Bin}\brak{n,p} \nonumber
\end{align}

We know ,
\begin{align} 
	E\brak{X}=np\\
	Var\brak{X}=np\brak{1-p}\\
	0\leq p\leq1 \\
	\implies -1\leq -p \leq0\\
	\implies
	0 \leq 1-p \leq1\\
	\implies np\brak{1-p}\leq np
\end{align}
Therefore,
\begin{align} 
	Var\brak{X}\leq E\brak{X}
\end{align}
From the four options,the statement that cannot be true is option \brak{3}\\

\textbf{Simulation steps}\\
\textbf{Step 1: Generate a Range of Probabilities}\\
The program generates a range of probabilities (\(p\)) in increments of 0.0098, ranging from 0.01 to 0.99. This range is used to create different binomial distributions for subsequent calculations.
\\
% It then calculates the variance of the generated binomial random variable.\\
\textbf{Step 2: Generating binomial r.v from Uniform distribution}\\
\( U \sim \text{Uniform}(0, 1) \).

Defining a random variable X as:
\[
X = 
\begin{cases}
    1 & \text{if } U > p \\
    0 & \text{otherwise}
\end{cases}
\]
This becomes a Bernoulli rv.
The count variable evaluates the Binomial r.v by the summation of Bernoulli r.v

\[
count= \sum_{i=1}^{n} X
\]
\textbf{Step 3: Calculate Variances for Each Probability}\\
For each probability (\(p\)) in the generated range, the program generates a binomial random variable using the given mean and the inverse of the probability (\(mean/p\)) from uniform distribution.\\
\textbf{Step 4: Find Maximum and Minimum Variances}\\
After calculating variances for each probability, the program identifies the maximum and minimum variances in the generated set.\\
\textbf{Step 5: User Input}\\
The program prompts the user to input a variance value for validation.
\\
\textbf{Step 6: Check Validity}\\
The user-input variance is compared against the computed maximum and minimum variances. If the input variance falls within this range (inclusive of the minimum and exclusive of the maximum), the program outputs "Valid." Otherwise, it outputs "Invalid."
\begin{figure}[ht!]
    \centering
    \includegraphics[width=\columnwidth]{2023/BM/12/codes/ss10.png}
    \caption{Variance for mean=10}
    \label{fig:ss10}
\end{figure}

	%\includegraphics[width=\columnwidth]{./codes/ss10.png}
	

	

\item Suppose from the estimation of a linear regression model
$$Y_i=\beta_0+\beta_1X_i+e_i$$
the residual sum of squares and the total sum of squares are obtained as 44 and 80, respectively. The value of coefficient of determination is \\ (round off to two decimal places).
\hfill (GATE XH 2023)
\iffalse
\let\negmedspace\undefined
\let\negthickspace\undefined
\documentclass[journal,12pt,twocolumn]{IEEEtran}
\usepackage{cite}
\usepackage{amsmath,amssymb,amsfonts,amsthm}
\usepackage{algorithmic}
\usepackage{graphicx}
\usepackage{textcomp}
\usepackage{xcolor}
\usepackage{txfonts}
\usepackage{listings}
\usepackage{enumitem}
\usepackage{mathtools}
\usepackage{gensymb}
\usepackage[breaklinks=true]{hyperref}
\usepackage{tkz-euclide} 
\usepackage{listings}
\newtheorem{theorem}{Theorem}[section]
\newtheorem{problem}{Problem}
\newtheorem{proposition}{Proposition}[section]
\newtheorem{lemma}{Lemma}[section]
\newtheorem{corollary}[theorem]{Corollary}
\newtheorem{example}{Example}[section]
\newtheorem{definition}[problem]{Definition}
\newcommand{\BEQA}{\begin{eqnarray}}
\newcommand{\EEQA}{\end{eqnarray}}
\newcommand{\define}{\stackrel{\triangle}{=}}
\theoremstyle{remark}
\newtheorem{rem}{Remark}

%\bibliographystyle{ieeetr}
\begin{document}
%

\bibliographystyle{IEEEtran}


\vspace{3cm}

\title{
%	\logo{
Assignment-7
%	}
}
\author{EE22BTECH11012-A.Chhatrapati}


\maketitle

\newpage

%\tableofcontents

\bigskip

\renewcommand{\thefigure}{\theenumi}
\renewcommand{\thetable}{\theenumi}


\providecommand{\pr}[1]{\ensuremath{\Pr\left(#1\right)}}
\providecommand{\prt}[2]{\ensuremath{p_{#1}^{\left(#2\right)} }}        % own macro for this question
\providecommand{\qfunc}[1]{\ensuremath{Q\left(#1\right)}}
\providecommand{\sbrak}[1]{\ensuremath{{}\left[#1\right]}}
\providecommand{\lsbrak}[1]{\ensuremath{{}\left[#1\right.}}
\providecommand{\rsbrak}[1]{\ensuremath{{}\left.#1\right]}}
\providecommand{\brak}[1]{\ensuremath{\left(#1\right)}}
\providecommand{\lbrak}[1]{\ensuremath{\left(#1\right.}}
\providecommand{\rbrak}[1]{\ensuremath{\left.#1\right)}}
\providecommand{\cbrak}[1]{\ensuremath{\left\{#1\right\}}}
\providecommand{\lcbrak}[1]{\ensuremath{\left\{#1\right.}}
\providecommand{\rcbrak}[1]{\ensuremath{\left.#1\right\}}}
\newcommand{\sgn}{\mathop{\mathrm{sgn}}}
\providecommand{\abs}[1]{\left\vert#1\right\vert}
\providecommand{\res}[1]{\Res\displaylimits_{#1}} 
\providecommand{\norm}[1]{\left\lVert#1\right\rVert}
%\providecommand{\norm}[1]{\lVert#1\rVert}
\providecommand{\mtx}[1]{\mathbf{#1}}
\providecommand{\mean}[1]{E\left[ #1 \right]}
\providecommand{\cond}[2]{#1\middle|#2}
\providecommand{\fourier}{\overset{\mathcal{F}}{ \rightleftharpoons}}
\newenvironment{amatrix}[1]{%
  \left(\begin{array}{@{}*{#1}{c}|c@{}}
}{%
  \end{array}\right)
}

\newcommand{\solution}{\noindent \textbf{Solution: }}
\newcommand{\cosec}{\,\text{cosec}\,}
\providecommand{\dec}[2]{\ensuremath{\overset{#1}{\underset{#2}{\gtrless}}}}
\newcommand{\myvec}[1]{\ensuremath{\begin{pmatrix}#1\end{pmatrix}}}
\newcommand{\mydet}[1]{\ensuremath{\begin{vmatrix}#1\end{vmatrix}}}
\newcommand{\myaugvec}[2]{\ensuremath{\begin{amatrix}{#1}#2\end{amatrix}}}
\providecommand{\rank}{\text{rank}}
\providecommand{\pr}[1]{\ensuremath{\Pr\left(#1\right)}}
\providecommand{\qfunc}[1]{\ensuremath{Q\left(#1\right)}}
	\newcommand*{\permcomb}[4][0mu]{{{}^{#3}\mkern#1#2_{#4}}}
\newcommand*{\perm}[1][-3mu]{\permcomb[#1]{P}}
\newcommand*{\comb}[1][-1mu]{\permcomb[#1]{C}}
\providecommand{\qfunc}[1]{\ensuremath{Q\left(#1\right)}}
\providecommand{\gauss}[2]{\mathcal{N}\ensuremath{\left(#1,#2\right)}}
\providecommand{\diff}[2]{\ensuremath{\frac{d{#1}}{d{#2}}}}
\providecommand{\myceil}[1]{\left \lceil #1 \right \rceil }
\newcommand\figref{Fig.~\ref}
\newcommand\tabref{Table~\ref}
\newcommand{\sinc}{\,\text{sinc}\,}
\newcommand{\rect}{\,\text{rect}\,}
\let\vec\mathbf

\textbf{Question 63.2023)}Suppose from the estimation of a linear regression model
$$Y_i=\beta_0+\beta_1X_i+e_i$$
the residual sum of squares and the total sum of squares are obtained as 44 and 80, respectively. The value of coefficient of determination is \\ (round off to two decimal places).\\
\solution
\fi
\begin{align}
Y_i &= \beta_0+\beta_1X_i+e_i
\end{align}
Here 
\begin{table}[ht]
    \centering
    \caption{Parameters}
    \label{table:xh_63.2023}
\begin{tabular}{|c|c|}
\hline
Parameters & Description \\
\hline
$Y_i$ & Dependent variable \\
\hline
$X_i$ & Independant variables \\
\hline
$\beta_0 , \beta_1$ & Constant variables \\ 
\hline
$e_i$ & Error term \\
\hline
\end{tabular} 
\end{table}
\begin{definition}
Residual sum of squares(RSS): \\
It measures the extent of variability of observed data not predicted by the regression model. That is it estimates the variance in residual or error term's. 
\end{definition}
\begin{align}
RSS &= \sum \brak{Y_i-\hat{Y}} \\
    &= \sum e_i^2 \\
    &= \sum \brak{Y_i-\beta_0-\beta_1X_i}^2\
\end{align}
Here $\hat{Y}$ = the value of Y on the line of regression.
\begin{definition}
Total sum of squares(TSS): \\
It measures the amount of variation measures in observed data. It is a measure of deviation from the mean. A low total sum of squares indicates little variation between data sets while a higher one indicates more variation. 
\end{definition}
\begin{align}
TSS &= \sum \brak{Y_i-\bar{Y}}^2
\end{align}
where $\bar{Y}$ = Mean of data
\begin{definition}
Coeffiecient of determination(\(R^2\)):\\
It is the proportion of the variance in the dependent variable that is predicted from the independent variable. It indicates the level of variation in the given data set.
\end{definition}
\begin{align}
R^2 &= 1-\frac{RSS}{TSS}\\
&= 1-\frac{44}{80}\\
&=0.45
\end{align}
45 percent of the variance in the Y variable is predicted from the X variable.
\begin{figure}[!ht]
\centering
\includegraphics[width=\columnwidth]{2023/XH/63/figs/figs/RSS.png}
\label{fig:xh_63.2023}
\end{figure}


\item  Consider the following regression model
\begin{center}
	$y_{k} = \alpha_{0} + \alpha_{1} \log_{e}k + \epsilon_{k}, \qquad k = 1,2,…,n,$\\
\end{center}
where $\epsilon_{k}$'s are independent and identically distributed random variables each
having probability density function $ f\brak{x} = \frac{1}{2} e^{-|x|}, x \in \mathbb{R}$. Then which one of
the following statements is true? 
\begin{enumerate}[label=(\Alph*)]
	\item The maximum likelihood estimator of $\alpha_{0}$ does not exist
	\item The maximum likelihood estimator of $\alpha_{1}$ does not exist
	\item The least squares estimator of $\alpha_{0}$ exists and is unique
	\item The least squares estimator of $\alpha_{1}$ exists, but it is not unique
\end{enumerate}
\hfill(GATE ST 2023)\\
\iffalse
\let\negmedspace\undefined
\let\negthickspace\undefined
\documentclass[journal,12pt,onecolumn]{IEEEtran}
\usepackage{cite}
\usepackage{amsmath,amssymb,amsfonts,amsthm}
\usepackage{algorithmic}
\usepackage{textcomp}
\usepackage{xcolor}
\usepackage{txfonts}
\usepackage{listings}
\usepackage{enumitem}
\usepackage{mathtools}
\usepackage{gensymb}
\usepackage[breaklinks=true]{hyperref}
\usepackage{tkz-euclide} % loads  TikZ and tkz-base
\usepackage{listings}
\usepackage{graphicx}
\usepackage{optidef}
\usepackage{gvv}
\usepackage{placeins}
\newcommand\myeq{\mathrel{\stackrel{\makebox[0pt]{\mbox{\normalfont\tiny iid}}}{\sim}}}
%
%\usepackage{setspace}
%\usepackage{gensymb}
%\doublespacing
%\singlespacing

%\usepackage{graphicx}
%\usepackage{amssymb}
%\usepackage{relsize}
%\usepackage[cmex10]{amsmath}
%\usepackage{amsthm}
%\interdisplaylinepenalty=2500
%\savesymbol{iint}
%\usepackage{txfonts}
%\restoresymbol{TXF}{iint}
%\usepackage{wasysym}
%\usepackage{amsthm}
%\usepackage{iithtlc}
%\usepackage{mathrsfs}
%\usepackage{txfonts}
%\usepackage{stfloats}
%\usepackage{bm}
%\usepackage{cite}
%\usepackage{cases}
%\usepackage{subfig}
%\usepackage{xtab}
%\usepackage{longtable}
%\usepackage{multirow}
%\usepackage{algorithm}
%\usepackage{algpseudocode}
%\usepackage{enumitem}
%\usepackage{mathtools}
%\usepackage{tikz}
%\usepackage{circuitikz}
%\usepackage{verbatim}
%\usepackage{tfrupee}
%\usepackage{stmaryrd}
%\usetkzobj{all}
    \usepackage{color}                                            %%
    \usepackage{array}                                            %%
    \usepackage{longtable}                                        %%
    \usepackage{calc}                                             %%
    \usepackage{multirow}                                         %%
    \usepackage{hhline}                                           %%
    \usepackage{ifthen}                                           %%
  %optionally (for landscape tables embedded in another document): %%
    \usepackage{lscape}     
%\usepackage{multicol}
%\usepackage{chngcntr}
%\usepackage{enumerate}

%\usepackage{wasysym}
%\documentclass[conference]{IEEEtran}
%\IEEEoverridecommandlockouts
% The preceding line is only needed to identify funding in the first footnote. If that is unneeded, please comment it out.

\newtheorem{theorem}{Theorem}[section]
\newtheorem{problem}{Problem}
\newtheorem{proposition}{Proposition}[section]
\newtheorem{lemma}{Lemma}[section]
\newtheorem{corollary}[theorem]{Corollary}
\newtheorem{example}{Example}[section]
\newtheorem{definition}[problem]{Definition}
%\newtheorem{thm}{Theorem}[section] 
%\newtheorem{defn}[thm]{Definition}
%\newtheorem{algorithm}{Algorithm}[section]
%\newtheorem{cor}{Corollary}
\newcommand{\BEQA}{\begin{eqnarray}}
\newcommand{\EEQA}{\end{eqnarray}}
\newcommand{\define}{\stackrel{\triangle}{=}}
\theoremstyle{remark}
\newtheorem{rem}{Remark}

%\bibliographystyle{ieeetr}
\begin{document}
%

\bibliographystyle{IEEEtran}


\vspace{3cm}

\title{
%	\logo{
	Solution to Gate ST 2023 Q 26
%	}
}
\author{ Mayank Gupta %$^{*}$% <-this % stops a space
%	\thanks{*The author is with the Department
%		of Electrical Engineering, Indian Institute of Technology, Hyderabad
%		502285 India e-mail:  gadepall@iith.ac.in. All content in this manual is released under GNU GPL.  Free and open source.}		
}
%\title{
%	\logo{Matrix Analysis through Octave}{\begin{center}\includegraphics[scale=.24]{tlc}\end{center}}{}{HAMDSP}
%}


% paper title
% can use linebreaks \\ within to get better formatting as desired
%\title{Matrix Analysis through Octave}
%
%
% author names and IEEE memberships
% note positions of commas and nonbreaking spaces ( ~ ) LaTeX will not break
% a structure at a ~ so this keeps an author's name from being broken across
% two lines.
% use \thanks{} to gain access to the first footnote area
% a separate \thanks must be used for each paragraph as LaTeX2e's \thanks
% was not built to handle multiple paragraphs
%

%\author{<-this % stops a space
%\thanks{}}
%}
% note the % following the last \IEEEmembership and also \thanks - 
% these prevent an unwanted space from occurring between the last author name
% and the end of the author line. i.e., if you ths:
% 
% \author{....lastname \thanks{...} \thanks{...} }
%                     ^------------^------------^----Do not want these spaces!
%
% a space would be appended to the last name and could cause every name on that
% line to be shifted left slightly. This is one of those "LaTeX things". For
% instance, "\textbf{A} \textbf{B}" will typeset as "A B" not "AB". To get
% "AB" then you have to do: "\textbf{A}\textbf{B}"
% \thanks is no different in this regard, so shield the last } of each \thanks
% that ends a line with a % and do not let a space in before the next \thanks.
% Spaces after \IEEEmembership other than the last one are OK (and needed) as
% you are supposed to have spaces between the names. For what it is worth,
% this is a minor point as most people would not even notice if the said evil
% space somehow managed to creep in.



% The paper headers
%\markboth{Journal of \LaTeX\ Class Files,~Vol.~6, No.~1, January~2007}%
%{Shell \MakeLowercase{\textit{et al.}}: Bare Demo of IEEEtran.cls for Journals}
% The only time the second header will appear is for the odd numbered pages
% after the title page when using the twoside option.
% 
% *** Note that you probably will NOT want to include the author's ***
% *** name in the headers of peer review papers.                   ***
% You can use \ifCLASSOPTIONpeerreview for conditional compilation here if
% you desire.




% If you want to put a publisher's ID mark on the page you can do it like
% this:
%\IEEEpubid{0000--0000/00\$00.00~\copyright~2007 IEEE}
% Remember, if you use this you must call \IEEEpubidadjcol in the second
% column for its text to clear the IEEEpubid mark.



% make the title area
\maketitle

%\tableofcontents

\bigskip

\renewcommand{\thefigure}{\theenumi}
\renewcommand{\thetable}{\theenumi}
%\renewcommand{\theequation}{\theenumi}

%\begin{abstract}
%%\boldmath
%In this letter, an algorithm for evaluating the exact analytical bit error rate  (BER)  for the piecewise linear (PL) combiner for  multiple relays is presented. Previous results were available only for upto three relays. The algorithm is unique in the sense that  the actual mathematical expressions, that are prohibitively large, need not be explicitly obtained. The diversity gain due to multiple relays is shown through plots of the analytical BER, well supported by simulations. 
%
%\end{abstract}
% IEEEtran.cls defaults to using nonbold math in the Abstract.
% This preserves the distinction between vectors and scalars. However,
% if the journal you are submitting to favors bold math in the abstract,
% then you can use LaTeX's standard command \boldmath at the very start
% of the abstract to achieve this. Many IEEE journals frown on math
% in the abstract anyway.

% Note that keywords are not normally used for peerreview papers.
%\begin{IEEEkeywords}
%Cooperative diversity, decode and forward, piecewise linear
%\end{IEEEkeywords}



% For peer review papers, you can put extra information on the cover
% page as needed:
% \ifCLASSOPTIONpeerreview
% \begin{center} \bfseries EDICS Category: 3-BBND \end{center}
% \fi
%
% For peerreview papers, this IEEEtran command inserts a page break and
% creates the second title. It will be ignored for other modes.
%\IEEEpeerreviewmaketitlei
Question : Consider the following regression model
\begin{center}
	$y_{k} = \alpha_{0} + \alpha_{1} \log_{e}k + \epsilon_{k}, \qquad k = 1,2,…,n,$\\
\end{center}
where $\epsilon_{k}$'s are independent and identically distributed random variables each
having probability density function $ f\brak{x} = \frac{1}{2} e^{-|x|}, x \in \mathbb{R}$. Then which one of
the following statements is true? 
\begin{enumerate}[label=(\Alph*)]
	\item The maximum likelihood estimator of $\alpha_{0}$ does not exist
	\item The maximum likelihood estimator of $\alpha_{1}$ does not exist
	\item The least squares estimator of $\alpha_{0}$ exists and is unique
	\item The least squares estimator of $\alpha_{1}$ exists, but it is not unique
\end{enumerate}
\fi
\solution
\begin{align}
	f(\epsilon_{k}) &= \frac{1}{2} e^{-|\epsilon_{k}|}\\
	\text{Likelihood function}: f(\epsilon_{1}\epsilon_{2}....\epsilon_{n}) &= \prod_{k = 1}^{n} f(\epsilon_{k})\\
	L     &= \prod_{k = 1}^{n}\frac{1}{2}e^{-|\epsilon_{k}|}\\
L_{1} = \ln L &= \ln\brak{\prod_{k = 1}^{n}\frac{1}{2}e^{-|\epsilon_{k}|}}\\
	      &= \sum_{k = 1}^{n} \ln\brak{\frac{1}{2}e^{-|\epsilon_{k}|}}\\
	      &= \sum_{k = 1}^{n} \brak{-\ln 2 -|y_{k}-\alpha_{0}-\alpha_{1} \log_{e}k|}\\
	      &= -n\ln 2 -\sum_{k = 1}^{n} \brak{|y_{k}-\alpha_{0}-\alpha_{1} \log_{e}k|}\\  
	L_{1} &= \text{function of }\alpha_{0}, \alpha_{1}
\end{align}
\begin{enumerate}
	\item Maximum likelihood estimator\\
	We need to find the value of $\alpha_{0}$ and $\alpha_{1}$ which will maximise the value of $L_{1}$ i.e. the value of 
	$\alpha_{0}$ and $\alpha_{1}$ which will minimise the value of $\sum_{k = 1}^{n}|y_{k}-\alpha_{0}-\alpha_{1} \log_{e}k|$
		\begin{enumerate}
\item With respect to $\alpha_{0}$
\begin{enumerate}
	\item For $y_{k}-\alpha_{0}-\alpha_{1} \log_{e}k > 0$
\begin{mini*}|s|
{\alpha_{0}}{y_{k}-\alpha_{0}-\alpha_{1} \log_{e}k}
{}{}
\addConstraint{\alpha_{0}\leq y_{k}-\alpha_{1}\log_{e}k}{}
\end{mini*}
Using Lagrange multiplier method
\begin{align}
	L\brak{\lambda} &= y_{k}-\alpha_{0}-\alpha_{1} \log_{e}k - \lambda(\alpha_{0}-y_{k}+\alpha_{1} \log_{e}k)\\
	\frac{\partial L}{\partial \alpha_{0}} &= -1-\lambda = 0\\
	\frac{\partial L}{\partial \lambda} &= y_{k}-\alpha_{0}-\alpha_{1} \log_{e}k = 0\\
	\lambda &= -1\\
	\alpha_{0} &= y_{k} - \alpha_{1} \log_{e}k
\end{align}
	\item For $y_{k}-\alpha_{0}-\alpha_{1} \log_{e}k < 0$
\begin{mini*}|s|
{\alpha_{0}}{-\brak{y_{k}-\alpha_{0}-\alpha_{1} \log_{e}k}}
{}{}
\addConstraint{\alpha_{0}\geq y_{k}-\alpha_{1}\log_{e}k}{}
\end{mini*}
Using Lagrange multiplier method
\begin{align}
	L\brak{\lambda} &= -\brak{y_{k}-\alpha_{0}-\alpha_{1} \log_{e}k} - \lambda(\alpha_{0}-y_{k}+\alpha_{1} \log_{e}k)\\
        \frac{\partial L}{\partial \alpha_{0}} &= 1-\lambda = 0\\
        \frac{\partial L}{\partial \lambda} &= y_{k}-\alpha_{0}-\alpha_{1} \log_{e}k = 0\\
        \lambda &= 1\\
        \alpha_{0} &= y_{k} - \alpha_{1} \log_{e}k
\end{align}
As value of $\alpha_{0}$ matches for both cases of modulus\\
$\therefore$ The maximum likelihood estimator of $\alpha_{0}$ exist
\end{enumerate}
\item With respect to $\alpha_{1}$
	\begin{enumerate}
	\item For $y_{k}-\alpha_{0}-\alpha_{1} \log_{e}k > 0$
\begin{mini*}|s|
{\alpha_{1}}{y_{k}-\alpha_{0}-\alpha_{1} \log_{e}k}
{}{}
	\addConstraint{\alpha_{1}\leq \frac{y_{k}-\alpha_{0}}{\log_{e}k}}{}
\end{mini*}
Using Lagrange multiplier method
\begin{align}
	L\brak{\lambda} &= y_{k}-\alpha_{0}-\alpha_{1} \log_{e}k - \lambda \brak{\alpha_{1}-\frac{y_{k}-\alpha_{0}}{\log_{e}k}}\\
        \frac{\partial L}{\partial \alpha_{1}} &= -\log_{e}k -\lambda = 0\\
	\frac{\partial L}{\partial \lambda} &= -\brak{\alpha_{1}-\frac{y_{k}-\alpha_{0}}{\log_{e}k}} = 0\\
        \lambda &= -\log_{e}k\\
	\alpha_{1} &= \frac{y_{k} - \alpha_{0}}{\log_{e}k}
\end{align}
	\item For $y_{k}-\alpha_{0}-\alpha_{1} \log_{e}k < 0$
\begin{mini*}|s|
{\alpha_{1}}{-\brak{y_{k}-\alpha_{0}-\alpha_{1} \log_{e}k}}
{}{}
        \addConstraint{\alpha_{1}\geq \frac{y_{k}-\alpha_{0}}{\log_{e}k}}{}
\end{mini*}
Using Lagrange multiplier method
\begin{align}
	L\brak{\lambda} &= -\brak{y_{k}-\alpha_{0}-\alpha_{1} \log_{e}k} - \lambda \brak{\alpha_{1}-\frac{y_{k}-\alpha_{0}}{\log_{e}k}}\\
        \frac{\partial L}{\partial \alpha_{1}} &= \log_{e}k -\lambda = 0\\
        \frac{\partial L}{\partial \lambda} &= -\brak{\alpha_{1}-\frac{y_{k}-\alpha_{0}}{\log_{e}k}} = 0\\
        \lambda &= \log_{e}k\\
        \alpha_{1} &= \frac{y_{k} - \alpha_{0}}{\log_{e}k}
\end{align}
As value of $\alpha_{1}$ matches for both cases of modulus\\
$\therefore$ The maximum likelihood estimator of $\alpha_{1}$ exist\\
\end{enumerate}
$\therefore$ Option \brak{A} and \brak{B} are incorrect\\
\item Least square estimator\\
The least square estimator of $\alpha_{0}$ and $\alpha_{1}$ is $\tilde{\alpha_{0}}$ and $\tilde{\alpha_{1}}$ which will minimise
\begin{table}[!htb]
	\input{2023/ST/26/tables/table.tex}
	\caption{Variables used}
	\label{table_gate23_st_26}
\end{table}
\begin{align}
	Q\brak{\alpha_{0},\alpha_{1}} &=  \sum_{k = 1}^{n} \brak{y_{k}-\alpha_{0}-\alpha_{1} \log_{e}k}^2\\
\frac{\partial Q}{\partial \alpha_{0}} &= -2 \sum_{k = 1}^{n} \brak{y_{k}-\alpha_{0}-\alpha_{1} \log_{e}k} = 0\\
	\sum_{k = 1}^{n} \brak{y_{k}-\alpha_{0}-\alpha_{1} \log_{e}k} &= 0\\
	n\bar{y} - n\alpha_{0} - \alpha_{1}n\bar{x} &= 0 \\%,\quad \text{where } \bar{x} = \frac{1}{n} \sum_{k = 1}^{n}\log_{e}k,\quad \bar{y} = \frac{1}{n} \sum_{k = 1}^{n}y_{k}\\
	\implies \tilde{\alpha_{0}} &= \bar{y} - \tilde{\alpha_{1}}\bar{x}\\
	\frac{\partial Q}{\partial \alpha_{1}} &= -2 \sum_{k = 1}^{n} \brak{y_{k}-\alpha_{0}-\alpha_{1} \log_{e}k}\log_{e}k = 0\\
	\implies \tilde{\alpha_{1}} &= \frac{\sum_{k = 1}^{n}\brak{\log_{e}k-\bar{x}}\brak{y_{k}-\bar{y}}}{\sum_{k = 1}^{n}\brak{\log_{e}k-\bar{x}}^2}
\end{align}
$\therefore$ Least square estimator of $\alpha_{0}$ and $\alpha_{1}$ exists and are unique\\
$\therefore$ Option \brak{C} is correct and \brak{D} is incorrect\\
\item Steps for simulation the given distribution whose 
	probability density function is $ f\brak{x} = \frac{1}{2} e^{-|x|}$
\begin{enumerate}
\item Write a function cdf for calculating the cdf of any random variable\\
	 \begin{align}
  p_X(x) &= 
  \begin{cases}
          \frac{1}{2} e^{x} & x \le 0
  \\
          \frac{1}{2} e^{-x} &  x > 0
  \end{cases}
  \end{align}

  \begin{align}
  F_X(x) &= 
  \begin{cases}
	  \int_{-\infty}^{x} \brak{\frac{1}{2} e^{x}} dx  & x \le 0
  \\
	  \int_{-\infty}^{0} \brak{\frac{1}{2} e^{x}} dx + \int_{0}^{x} \brak{\frac{1}{2} e^{-x}} dx &  x > 0
  \end{cases}
  \end{align}

  \begin{align}
  F_X(x) &= 
  \begin{cases}
	  \frac{1}{2} e^{x} & x \le 0
  \\
	  \frac{1}{2} \brak{2-e^{-x}} &  x > 0
  \end{cases}
  \end{align}
\item Declare a function inverse cdf \brak{I\brak{u}} such that its input is any random number 
	and output is random variable whose cdf equals that of the given distribution\\
	For x $\le$ 0
		\begin{align}
			u &= \frac{1}{2} e^{x}\\
			e^{x} &= 2u\\
			x &= \ln{2u}\\
			\because x \le 0\\
			u \le 0.5
		\end{align}
	For x $>$ 0
		\begin{align}
			u &= \frac{1}{2} \brak{2-e^{-x}}\\
			2-e^{-x} &= 2u\\
			e^{-x} &= 2-2u\\
			x &= -\ln \brak{2-2u}\\
			\because x > 0\\
			u > 0.5
		\end{align}
  \begin{align}
  I\brak{u} &=
  \begin{cases}
	  \ln\brak{2u} & u \le 0.5
  \\
	  -\ln \brak{2-2u} &  u > 0.5
  \end{cases}
  \end{align}
\item Define three arrays random\textunderscore{vars} , cdf\textunderscore{values}
	, theoretical\textunderscore{cdf}\textunderscore{values}
	to store random variables, simulated cdf values and theoretical cdf values
\item Generate random numbers using rand() and calling inverse cdf funtion to generate our random variable
\item Calling cdf function to calculate the cdf of the generated random variable
\item Storing the random variable,theoretical cdf and generated cdf into their respective arrays
\item Storing the data of these three array into a .dat file
\item Plotting these .dat file in python
\end{enumerate}
\end{enumerate}
\end{enumerate}

\item Consider the following regression model
\begin{align}
y_t={\alpha}_0+{\alpha}_1t+{\alpha}_2t^2+\epsilon_{t}, \qquad t = 1,2,…,n
\end{align}
where ${\alpha}_0$ , ${\alpha}_1$ and ${\alpha}_2$ are unknown parameters and $\epsilon_{t}$’s are independent and identically distributed random variables each having $\gauss{\mu}{1}$ distribution with $\mu$ unknown. Then which of the following statements is/are true?
\begin{enumerate}
\item{There exists an unbiased estimator of ${\alpha}_1$}
\item{There exists an unbiased estimator of ${\alpha}_2$}
\item{There exists an unbiased estimator of ${\alpha}_0$}
\item{There exists an unbiased estimator of ${\mu}$}
\end{enumerate}
\hfill(GATE ST 2023)\\
\iffalse
\let\negmedspace\undefined
\let\negthickspace\undefined
\documentclass[journal,12pt,twocolumn]{IEEEtran}
\usepackage{cite}
\usepackage{amsmath,amssymb,amsfonts,amsthm}
\usepackage{algorithmic}
\usepackage{graphicx}
\usepackage{textcomp}
\usepackage{xcolor}
\usepackage{txfonts}
\usepackage{listings}
%\usepackage{enumitem}
\usepackage{mathtools}
\usepackage{gensymb}
\usepackage[breaklinks=true]{hyperref}
\usepackage{tkz-euclide} % loads  TikZ and tkz-base
\usepackage{listings}
\usepackage[inline]{enumitem}
\DeclareMathOperator*{\Res}{Res}
\renewcommand\thesection{\arabic{section}}
\renewcommand\thesubsection{\thesection.\arabic{subsection}}
\renewcommand\thesubsubsection{\thesubsection.\arabic{subsubsection}}


\def\inputGnumericTable{}

\usepackage[latin1]{inputenc}                                 
\usepackage{color}                                            
\usepackage{array}                                            
\usepackage{longtable}                                        
\usepackage{calc}                                             
\usepackage{multirow}                                         
\usepackage{hhline}                                           
\usepackage{ifthen}
\usepackage{caption} 
\captionsetup[table]{skip=3pt}  
\providecommand{\pr}[1]{\ensuremath{\Pr\left(#1\right)}}
\providecommand{\cbrak}[1]{\ensuremath{\left\{#1\right\}}}

\renewcommand\thesectiondis{\arabic{section}}
\renewcommand\thesubsectiondis{\thesectiondis.\arabic{subsection}}
\renewcommand\thesubsubsectiondis{\thesubsectiondis.\arabic{subsubsection}}

\def\inputGnumericTable{}                                 %%

\lstset{
frame=single, 
breaklines=true,
columns=fullflexible
}

\begin{document}

\newtheorem{theorem}{Theorem}[section]
\newtheorem{problem}{Problem}
\newtheorem{proposition}{Proposition}[section]
\newtheorem{lemma}{Lemma}[section]
\newtheorem{corollary}[theorem]{Corollary}
\newtheorem{example}{Example}[section]
\newtheorem{definition}[problem]{Definition}
\newcommand{\BEQA}{\begin{eqnarray}}
\newcommand{\EEQA}{\end{eqnarray}}
\newcommand{\define}{\stackrel{\triangle}{=}}
\newcommand{\xor}{\oplus}
\bibliographystyle{IEEEtran}

\providecommand{\mbf}{\mathbf}
\providecommand{\pr}[1]{\ensuremath{\Pr\left(#1\right)}}
\providecommand{\qfunc}[1]{\ensuremath{Q\left(#1\right)}}
\providecommand{\sbrak}[1]{\ensuremath{{}\left[#1\right]}}
\providecommand{\lsbrak}[1]{\ensuremath{{}\left[#1\right.}}
\providecommand{\rsbrak}[1]{\ensuremath{{}\left.#1\right]}}
\providecommand{\brak}[1]{\ensuremath{\left(#1\right)}}
\providecommand{\lbrak}[1]{\ensuremath{\left(#1\right.}}
\providecommand{\rbrak}[1]{\ensuremath{\left.#1\right)}}
\providecommand{\cbrak}[1]{\ensuremath{\left\{#1\right\}}}
\providecommand{\lcbrak}[1]{\ensuremath{\left\{#1\right.}}
\providecommand{\rcbrak}[1]{\ensuremath{\left.#1\right\}}}
\theoremstyle{remark}
\newtheorem{rem}{Remark}
\newcommand{\sgn}{\mathop{\mathrm{sgn}}}

\newcommand{\solution}{\noindent \textbf{Solution: }}
\newcommand{\cosec}{\,\text{cosec}\,}
\providecommand{\dec}[2]{\ensuremath{\overset{#1}{\underset{#2}{\gtrless}}}}
\newcommand{\myvec}[1]{\ensuremath{\begin{pmatrix}#1\end{pmatrix}}}
\newcommand{\mydet}[1]{\ensuremath{\begin{vmatrix}#1\end{vmatrix}}}

\let\vec\mathbf


\vspace{3cm}

\title{
  

  Assignment -5 in \LaTeX
    
  }
  \author{ Muzaan Mohammed Faizel A P\\
  EE22BTECH11036
  }	
% make the title area
\maketitle
\newpage
\bigskip
\renewcommand{\thefigure}{\theenumi}
\renewcommand{\thetable}{\theenumi}
\renewcommand{\thetable}{\arabic{table}} 
\textbf{GATE 2023 BM QN.12}
For a Binomial random variable $X$, E($X$) and Var($X$) are the expectation and
variance, respectively. Which one of the following statements CANNOT be true?
\begin{table}[ht!]
		\centering
		\input{2023/BM/12/tables/table1.tex}
		\caption{}
		\label{table:table1}	
\end{table}
\fi
\solution
\begin{align}
X \sim \text{Bin}\brak{n,p} \nonumber
\end{align}

We know ,
\begin{align} 
	E\brak{X}=np\\
	Var\brak{X}=np\brak{1-p}\\
	0\leq p\leq1 \\
	\implies -1\leq -p \leq0\\
	\implies
	0 \leq 1-p \leq1\\
	\implies np\brak{1-p}\leq np
\end{align}
Therefore,
\begin{align} 
	Var\brak{X}\leq E\brak{X}
\end{align}
From the four options,the statement that cannot be true is option \brak{3}\\

\textbf{Simulation steps}\\
\textbf{Step 1: Generate a Range of Probabilities}\\
The program generates a range of probabilities (\(p\)) in increments of 0.0098, ranging from 0.01 to 0.99. This range is used to create different binomial distributions for subsequent calculations.
\\
% It then calculates the variance of the generated binomial random variable.\\
\textbf{Step 2: Generating binomial r.v from Uniform distribution}\\
\( U \sim \text{Uniform}(0, 1) \).

Defining a random variable X as:
\[
X = 
\begin{cases}
    1 & \text{if } U > p \\
    0 & \text{otherwise}
\end{cases}
\]
This becomes a Bernoulli rv.
The count variable evaluates the Binomial r.v by the summation of Bernoulli r.v

\[
count= \sum_{i=1}^{n} X
\]
\textbf{Step 3: Calculate Variances for Each Probability}\\
For each probability (\(p\)) in the generated range, the program generates a binomial random variable using the given mean and the inverse of the probability (\(mean/p\)) from uniform distribution.\\
\textbf{Step 4: Find Maximum and Minimum Variances}\\
After calculating variances for each probability, the program identifies the maximum and minimum variances in the generated set.\\
\textbf{Step 5: User Input}\\
The program prompts the user to input a variance value for validation.
\\
\textbf{Step 6: Check Validity}\\
The user-input variance is compared against the computed maximum and minimum variances. If the input variance falls within this range (inclusive of the minimum and exclusive of the maximum), the program outputs "Valid." Otherwise, it outputs "Invalid."
\begin{figure}[ht!]
    \centering
    \includegraphics[width=\columnwidth]{2023/BM/12/codes/ss10.png}
    \caption{Variance for mean=10}
    \label{fig:ss10}
\end{figure}

	%\includegraphics[width=\columnwidth]{./codes/ss10.png}
	

	

\item Let $X_1, X_2,...,X_n$ be a random sample of size $n$ from a population having uniform distribution over the interval $\brak{\frac{1}{3},\theta}$, where $\theta>\frac{1}{3}$ is an unknown parameter. If $Y = $ max \{$X_1, X_2,...,X_n$\}, then which one of the following statements is true?
\begin{enumerate}
\item $\brak{\frac{n+1}{n}}\brak{Y-\frac{1}{3}} + \frac{1}{3}$ is an unbiased estimator of $\theta$
\item $\brak{\frac{n}{n+1}}\brak{Y-\frac{1}{3}} + \frac{1}{3}$ is an unbiased estimator of $\theta$
\item $\brak{\frac{n+1}{n}}\brak{Y+\frac{1}{3}} - \frac{1}{3}$ is an unbiased estimator of $\theta$
\item $Y$ is an unbiased estimator of $\theta$
\end{enumerate} 
\hfill(GATE ST 2023)\\
\iffalse
\let\negmedspace\undefined
\let\negthickspace\undefined
\documentclass[journal,12pt,twocolumn]{IEEEtran}
\usepackage{setspace}
\singlespacing
\usepackage[cmex10]{amsmath}
\usepackage{amsthm}
\usepackage{mathrsfs}
\usepackage{txfonts}
\usepackage{stfloats}
\usepackage{bm}
\usepackage{cite}
\usepackage{cases}
\usepackage{subfig}
\usepackage{longtable}
\usepackage{multirow}
\usepackage{enumitem}
\usepackage{mathtools}
\usepackage{tikz}
\usepackage{circuitikz}
\usepackage{verbatim}
\usepackage[breaklinks=true]{hyperref}
\usepackage{tkz-euclide} % loads  TikZ and tkz-base
\usepackage{listings}
\usepackage{color}    
\usepackage{array}    
\usepackage{longtable}
\usepackage{calc}     
\usepackage{multirow} 
\usepackage{hhline}   
\usepackage{ifthen}   
\usepackage{lscape}     
\usepackage{chngcntr}
\DeclareMathOperator*{\Res}{Res}
\renewcommand\thesection{\arabic{section}}
\renewcommand\thesubsection{\thesection.\arabic{subsection}}
\renewcommand\thesubsubsection{\thesubsection.\arabic{subsubsection}}
\renewcommand\thesectiondis{\arabic{section}}
\renewcommand\thesubsectiondis{\thesectiondis.\arabic{subsection}}
\renewcommand\thesubsubsectiondis{\thesubsectiondis.\arabic{subsubsection}}
\renewcommand{\thefigure}{\theenumi}
\renewcommand{\thetable}{\theenumi}
\providecommand{\gauss}[2]{\mathcal{N}\ensuremath{\left(#1,#2\right)}}
% correct bad hyphenation here
\hyphenation{op-tical net-works semi-conduc-tor}
\def\inputGnumericTable{}                                 %%

\lstset{
%language=C,
frame=single, 
breaklines=true,
columns=fullflexible
}
%\lstset{
%language=tex,
%frame=single, 
%breaklines=true
%}
\begin{document}
\newtheorem{theorem}{Theorem}[section]
\newtheorem{problem}{Problem}
\newtheorem{proposition}{Proposition}[section]
\newtheorem{lemma}{Lemma}[section]
\newtheorem{corollary}[theorem]{Corollary}
\newtheorem{example}{Example}[section]
\newtheorem{definition}[problem]{Definition}
\newcommand{\BEQA}{\begin{eqnarray}}
\newcommand{\EEQA}{\end{eqnarray}}
\newcommand{\define}{\stackrel{\triangle}{=}}
\bibliographystyle{IEEEtran}
\providecommand{\mbf}{\mathbf}
\providecommand{\pr}[1]{\ensuremath{\Pr\left(#1\right)}}
\providecommand{\qfunc}[1]{\ensuremath{Q\left(#1\right)}}
\providecommand{\sbrak}[1]{\ensuremath{{}\left[#1\right]}}
\providecommand{\lsbrak}[1]{\ensuremath{{}\left[#1\right.}}
\providecommand{\rsbrak}[1]{\ensuremath{{}\left.#1\right]}}
\providecommand{\brak}[1]{\ensuremath{\left(#1\right)}}
\providecommand{\lbrak}[1]{\ensuremath{\left(#1\right.}}
\providecommand{\rbrak}[1]{\ensuremath{\left.#1\right)}}
\providecommand{\cbrak}[1]{\ensuremath{\left\{#1\right\}}}
\providecommand{\lcbrak}[1]{\ensuremath{\left\{#1\right.}}
\providecommand{\rcbrak}[1]{\ensuremath{\left.#1\right\}}}
\theoremstyle{remark}
\newtheorem{rem}{Remark}
\newcommand{\sgn}{\mathop{\mathrm{sgn}}}
\providecommand{\abs}[1]{\left\vert#1\right\vert}
\providecommand{\res}[1]{\Res\displaylimits_{#1}} 
\providecommand{\norm}[1]{\left\lVert#1\right\rVert}
\providecommand{\mtx}[1]{\mathbf{#1}}
\providecommand{\mean}[1]{E\left[ #1 \right]}
\providecommand{\fourier}{\overset{\mathcal{F}}{ \rightleftharpoons}}
\providecommand{\system}[1]{\overset{\mathcal{#1}}{ \longleftrightarrow}}
\newcommand{\solution}{\noindent \textbf{Solution: }}
\newcommand{\cosec}{\,\text{cosec}\,}
\providecommand{\dec}[2]{\ensuremath{\overset{#1}{\underset{#2}{\gtrless}}}}
\newcommand{\myvec}[1]{\ensuremath{\begin{pmatrix}#1\end{pmatrix}}}
\newcommand{\mydet}[1]{\ensuremath{\begin{vmatrix}#1\end{vmatrix}}}
\let\vec\mathbf
\def\putbox#1#2#3{\makebox[0in][l]{\makebox[#1][l]{}\raisebox{\baselineskip}[0in][0in]{\raisebox{#2}[0in][0in]{#3}}}}
     \def\rightbox#1{\makebox[0in][r]{#1}}
     \def\centbox#1{\makebox[0in]{#1}}
     \def\topbox#1{\raisebox{-\baselineskip}[0in][0in]{#1}}
     \def\midbox#1{\raisebox{-0.5\baselineskip}[0in][0in]{#1}}
\setlength{\parindent}{0pt}
\bibliographystyle{IEEEtran}
\newenvironment{amatrix}[1]{%
  \left(\begin{array}{@{}*{#1}{c}|c@{}}
}{%
  \end{array}\right)
}
\title{
%	\logo{
Probability and Random Processes
%	}
}
\author{ Gude Pravarsh EE22BTECH11023$^{*}$% <-this % stops a space
	%
}
	
	
%\title{
%	\logo{Matrix Analysis through Octave}{\begin{center}\includegraphics[scale=.24]{tlc}\end{center}}{}{HAMDSP}
%}


% paper title
% can use linebreaks \\ within to get better formatting as desired
%\title{Matrix Analysis through Octave}
%
%
% author names and IEEE memberships
% note positions of commas and nonbreaking spaces ( ~ ) LaTeX will not break
% a structure at a ~ so this keeps an author's name from being broken across
% two lines.
% use \thanks{} to gain access to the first footnote area
% a separate \thanks must be used for each paragraph as LaTeX2e's \thanks
% was not built to handle multiple paragraphs
%
%\author{<-this % stops a space
%\thanks{}}
%}
% note the % following the last \IEEEmembership and also \thanks - 
% these prevent an unwanted space from occurring between the last author name
% and the end of the author line. i.e., if you had this:
% 
% \author{....lastname \thanks{...} \thanks{...} }
%                     ^------------^------------^----Do not want these spaces!
%
% a space would be appended to the last name and could cause every name on that
% line to be shifted left slightly. This is one of those "LaTeX things". For
% instance, "\textbf{A} \textbf{B}" will typeset as "A B" not "AB". To get
% "AB" then you have to do: "\textbf{A}\textbf{B}"
% \thanks is no different in this regard, so shield the last } of each \thanks
% that ends a line with a % and do not let a space in before the next \thanks.
% Spaces after \IEEEmembership other than the last one are OK (and needed) as
% you are supposed to have spaces between the names. For what it is worth,
% this is a minor point as most people would not even notice if the said evil
% space somehow managed to creep in.



% The paper headers
%\markboth{Journal of \LaTeX\ Class Files,~Vol.~6, No.~1, January~2007}%
%{Shell \MakeLowercase{\textit{et al.}}: Bare Demo of IEEEtran.cls for Journals}
% The only time the second header will appear is for the odd numbered pages
% after the title page when using the twoside option.
% 
% *** Note that you probably will NOT want to include the author's ***
% *** name in the headers of peer review papers.                   ***
% You can use \ifCLASSOPTIONpeerreview for conditional compilation here if
% you desire.




% If you want to put a publisher's ID mark on the page you can do it like
% this:
%\IEEEpubid{0000--0000/00\$00.00~\copyright~2007 IEEE}
% Remember, if you use this you must call \IEEEpubidadjcol in the second
% column for its text to clear the IEEEpubid mark.
% make the title area 
\maketitle

\newpage

%\tableofcontents

\bigskip

\renewcommand{\thefigure}{\arabic{figure}}
\renewcommand{\thetable}{\theenumi}
%\renewcommand{\theequation}{\theenumi}

%\begin{abstract}
%%\boldmath
%In this letter, an algorithm for evaluating the exact analytical bit error rate  (BER)  for the piecewise linear (PL) combiner for  multiple relays is presented. Previous results were available only for upto three relays. The algorithm is unique in the sense that  the actual mathematical expressions, that are prohibitively large, need not be explicitly obtained. The diversity gain due to multiple relays is shown through plots of the analytical BER, well supported by simulations. 
%
%\end{abstract}
% IEEEtran.cls defaults to using nonbold math in the Abstract.
% This preserves the distinction between vectors and scalars. However,
% if the journal you are submitting to favors bold math in the abstract,
% then you can use LaTeX's standard command \boldmath at the very start
% of the abstract to achieve this. Many IEEE journals frown on math
% in the abstract anyway.

% Note that keywords are not normally used for peerreview papers.
%\begin{IEEEkeywords}
%Cooperative diversity, decode and forward, piecewise linear
%\end{IEEEkeywords} 
Q)Let $\{X_n\}_{n \geq 1}$ be a sequence of independent and identically distributed random variables each having probability density function
\[
f(x) = 
\begin{cases}
  e^{-x} & \text{if } x > 0 \\
  0 & \text{otherwise}.
\end{cases}
\]
For $n \geq 1$, let $Y_n = |X_{2n} - X_{2n-1}|$. If $\overline{Y}_n = \frac{1}{n} \sum_{i=1}^{n} Y_i$ for $n \geq 1$ and $\{\sqrt{n} (e^{-\overline{Y}_n} - e^{-1})\}_{n \geq 1}$ converges in distribution to a normal random variable with mean $0$ and variance $\sigma^2$, then $\sigma^2$ (rounded off to two decimal places) equals \hfill (GATE ST 2023) \\
   \fi
   \solution
\begin{enumerate}
\item Let $X, Y \sim \exp(1)$ and $Z = X - Y$ 
\begin{align}
p_X(x) &= e^{-x}u(x) \\
M_X(s) &= E\brak{e^{-sX}} \\
&= \int_{0}^{\infty}e^{-sx} e^{-x} \,dx \\
&= \frac{1}{s + 1}
\end{align} 
ROC for $M_X(s):\text{Re}(s)>-1$ \\ 
Similarly,
\begin{align}
M_Y(s) &= \frac{1}{s + 1} \\
M_Y(-s) &= \frac{1}{-s + 1} 
\end{align}
ROC for $M_Y(-s):\text{Re}(s)<1$
\begin{align}
M_Z(s) &= E\brak{e^{-sZ}} 
\end{align}
Using,
\begin{align}
Z &= X - Y \label{eq:ST/2023/61/1} \\
\implies M_Z(s) &= E\brak{e^{-s(X-Y)}} \\
&= E\brak{e^{-sX}}E\brak{e^{sY}} \\
&= M_X(s)M_Y(-s) \\
&= \frac{1}{s+1}\times\frac{1}{-s+1} \\ 
M_Z(s) &= \frac{1}{1-s^2} 
\end{align}
The ROC for the laplace transform : $ |\text{Re}(s)|<1 $\\
\begin{align}
M_Z(s) &= \frac{1}{2}\brak{\frac{1}{1-s} + \frac{1}{1+s}} 
\end{align}
Using Inverse Laplace transform,
\begin{align}
P_Z(x) &= \frac{1}{2}\brak{ e^x u\brak{-x} + e^{-x} u\brak{x}} \\
p_Z(x) &=  \frac{1}{2} e^{-|x|} \\
\implies Z &\sim \mathrm{Lap}\brak{0, 1} 
\end{align}
\item Let $T = |Z|$
\begin{align}
p_Z(x) &= \frac{1}{2} e^{-|x|} \\
F_Z(x) &= \int_{-\infty}^{x} \frac{1}{2} e^{-|t|} \,dt \\
&= \frac{1}{2} + \frac{1}{2} e^{-x} \\
F_T(x) &= \pr{T \leq x} \\
&=\pr{|Z| \leq x} \\ 
&= \pr{-x \leq Z \leq x} \\
F_T(x) &= \frac{1}{2} - \frac{1}{2} e^{-x} - \brak{- \frac{1}{2} + \frac{1}{2} e^{-x}} \\
F_T(x) &= 1 - e^{-x} \text{ for $x>0$} \\
T &\sim \exp\brak{1} \\
\implies |Z| &\sim \exp\brak{1}  \label{eq:ST/2023/61/3}
\end{align}
Using equations \eqref{eq:ST/2023/61/1} and \eqref{eq:ST/2023/61/3}, we get:
\begin{align}
|X_{2n} - X_{2n-1}| &\sim \exp(1) \\
\implies Y_n &\sim \exp(1)  \\
M_{Y_n}(s) &= \frac{1}{1+s} \\
\text{E}\brak{Y_n} &= \mu_1 \\
\mu_1 &= -\frac{dM_{Y_n}(s)}{ds} \\
&= -\left. \frac{d}{ds} \brak{\frac{1}{s + 1}} \right|_{s=0} 
\end{align}
\begin{align}
&= \left. \frac{1}{(s + 1)^2} \right|_{s=0} \\ 
\text{E}\brak{Y_n} &= 1 \\
\text{E}\brak{{Y_n}^2} &= \mu_2 \\
\mu_2 &= \frac{d^2{Y_n}(s)}{ds^2} \\
&= \left. \frac{d^2}{ds^2} \brak{\frac{-1}{(s + 1)^2}} \right|_{s=0} \\
&= \left. \frac{2}{(s + 1)^3} \right|_{s=0} \\
\text{E}\brak{{Y_n}^2} &= 2 
\end{align}
\begin{align}
\mathrm{Var}(Y_n) &= \mathrm{E}((Y_n -\mathrm{E}(Y_n))^2) \\
&= \mathrm{E}((Y_n - 1)^2) \\
&= \mathrm{E}({Y_n}^2) -2 \mathrm{E}(Y_n) + 1 = 1  
\end{align}
\item We know, \begin{align}
\overline{Y}_n &= \frac{1}{n} \sum_{i=1}^{n} Y_i \\
\mathrm{E}\brak{\overline{Y}_n} &= \frac{1}{n} \sum_{i=1}^{n} \mathrm{E}\brak{Y_i} \\
&= \frac{1}{n} \cdot \brak{n} = 1 \\
\mathrm{E}\brak{\overline{Y}_n}&= 1 
\end{align}
\begin{align}
    \text{var}\brak{\overline{Y}_n} &= \text{E}\sbrak{\brak{\frac{1}{n} \sum_{i=1}^{n} Y_i}^2} - \brak{\text{E}\sbrak{\frac{1}{n} \sum_{i=1}^{n} Y_i}}^2\\
    &= \frac{1}{n^2} \cbrak{\text{E}\sbrak{\brak{\sum_{i=1}^{n} Y_i}^2} - \brak{\text{E}\sbrak{\sum_{i=1}^{n} Y_i}}^2}\label{eq:ST/61/2023/1}
\end{align}
But
\begin{align}
    \text{E}\sbrak{\brak{\sum_{i=1}^{n} Y_i}^2} &= \text{E}\sbrak{\sum_{i=1}^{n} \sum_{j=1}^{n} Y_iY_j}\\
    &= \sum_{i=1}^{n} \sum_{j=1}^{n} \text{E}\sbrak{Y_iY_j} \label{eq:ST/61/2023/2}
\end{align}
and 
\begin{align}
    \brak{\text{E}\sbrak{\sum_{i=1}^{n} Y_i}}^2 &= \brak{\sum_{i=1}^{n} \text{E}\sbrak{Y_i}}^2\\
    &= \sum_{i=1}^{n} \sum_{j=1}^{n} \text{E}\sbrak{Y_i} \text{E}\sbrak{Y_j} \label{eq:ST/61/2023/3}
\end{align}
Putting \eqref{eq:ST/61/2023/2} and \eqref{eq:ST/61/2023/3} in \eqref{eq:ST/61/2023/1}, and using the definition of covariance,
\begin{align}
    \text{var}\brak{\overline{Y}_n} &= \frac{1}{n^2} \cbrak{\sum_{i=1}^{n} \sum_{j=1}^{n} \brak{\text{E}\sbrak{Y_iY_j} - \text{E}\sbrak{Y_i} \text{E}\sbrak{Y_j}}}\\
    &= \frac{1}{n^2} \cbrak{\sum_{i=1}^{n} \sum_{j=1}^{n} \text{cov}\brak{Y_i, Y_j}} \label{eq:ST/61/2023/4}
\end{align}
As all the variables are i.i.d's and are thus uncorrelated,
\begin{align}
    \text{cov}\brak{Y_i, Y_j} =
    \begin{cases}
        0 & \text{if } i \ne j\\
        \text{var}\brak{Y_i} & \text{if } i = j
    \end{cases}\label{eq:ST/61/2023/5}
\end{align}
Putting \eqref{eq:ST/61/2023/5} in \eqref{eq:ST/61/2023/4},
\begin{align}
    \text{var}\brak{\overline{Y}_n} &= \frac{1}{n^2} \brak{\sum_{i=1}^{n} \text{cov}\brak{Y_i, Y_i}}\\
     &= \frac{1}{n^2} \brak{\sum_{i=1}^{n} \text{var}\brak{Y_i}}\\
     &= \frac{1}{n^2} \cdot n = \frac{1}{n}\\
\text{var}\brak{\overline{Y}_n} &= \frac{1}{n} \\
\implies \mathrm{E}\brak{\overline{Y}_n} &= 1 \text{ and } \mathrm{Var}\brak{\overline{Y}_n} = \frac{1}{n}  
\end{align}
By the Central Limit Theorem, $n \to \infty \implies \sqrt{n} (Y_n - \mu) \xrightarrow{} \mathcal{N}(0, 1)$
\begin{align}
\frac{\overline{Y_n} - 1}{\frac{1}{\sqrt{n}}} &\sim \mathcal{N}(0, 1) \\
\sqrt{n} (\overline{Y_n}- 1) &\sim \mathcal{N}(0, 1) 
\end{align}
We know,
\begin{align}
\sqrt{n} (Y_n - k) \sim \mathcal{N}(0,\sigma^2) 
\end{align}
Let us write the taylor expansion of $g\brak{Y_n}$ around $k$
\begin{align}
g(Y_n) = g(k) + g'(k)(Y_n - k) + \frac{1}{2}g''(k)(Y_n - k)^2 + \ldots 
\end{align}
Apply the Central Limit Theorem (CLT) to the standardized variable $Z_n$ 
\begin{align}
Z_n = \frac{\sqrt{n}g'(k)(Y_n - k)}{\sigma \sqrt{n}} \\
n \to \infty \implies Z_n \xrightarrow{} \mathcal{N}(0, 1)
\end{align}
Compare with standardised variable we get,
\begin{align}
\implies \sqrt{n} (g\brak{Y_n} - g(k)) \sim \mathcal{N}(0, \sigma^2 [g'(k)]^2) \label{eq:ST/2023/61/2}
\end{align}
\end{enumerate}
\begin{align}
g(x) = e^{-x} \implies g'(x) = -e^{-x} 
\end{align}
Using equation \eqref{eq:ST/2023/61/2}, we get:
\begin{align}
\sqrt{n} (e^{-\overline{Y}_n} - e^{-1}) \sim \mathcal{N}(0,e^{-2}) \\
\implies \sigma^2 = e^{-2} = 0.14 
\end{align}
\textbf{Steps for Simulation:}
\begin{enumerate}
\item  rand() / (double)RAND MAX: \\
This generates a random variable between 0 and RAND MAX and divides
it by RAND MAX to obtain a uniform distribution between 0 and 1
\item -log(rand() / (double)RAND MAX) : \\
This transforms the uniform distribution between 0 and 1 into an exponential
distribution by making the values vary from 0 to  $\infty$.
\item Generate '2n' samples of Random Variable $X$ from the given probability density function.
\item Now generate 'n' samples of $Y = |X_{2n} - X_{2n-1}| $.
\item Now find $\overline{Y}$ which is mean of 'n' samples of $Y$.
\item Now calculate $\sqrt{n} (e^{-\overline{Y}_n} - e^{-1})$ as result.
\item Now repeat the process for 'm' simulations so that we get m results.
\item Calculate the mean and the variance of the 'm' results obtained earlier using basic mean and variance formula.
\end{enumerate}
\begin{figure}[ht]
\includegraphics[width=\columnwidth]{2023/ST/61/figs/Figure_1.png}
\caption{pdf of the laplacian}
\label{fig:2023/ST/61/1}
\end{figure}
\begin{figure}[ht]
\includegraphics[width=\columnwidth]{2023/ST/61/figs/Figure_2.png}
\caption{pdf of absolute of the laplacian}
\label{fig:2023/ST/61/2}
\end{figure}
\begin{figure}[ht]
\includegraphics[width=\columnwidth]{2023/ST/61/figs/Figure_3.png}
\caption{Gaussian pdf}
\label{fig:2023/ST/61/3}
\end{figure}
\begin{figure}[ht]
\includegraphics[width=\columnwidth]{2023/ST/61/figs/Figure_11.png}
\caption{Cdf of X}
\label{fig:2023/ST/61/4}
\end{figure}
\begin{figure}[ht]
\includegraphics[width=\columnwidth]{2023/ST/61/figs/Figure_12.png}
\caption{Cdf of Z}
\label{fig:2023/ST/61/5}
\end{figure}
\begin{figure}[ht]
\includegraphics[width=\columnwidth]{2023/ST/61/figs/Figure_13.png}
\caption{Cdf of T}
\label{fig:2023/ST/61/6}
\end{figure}
\begin{figure}[ht]
\includegraphics[width=\columnwidth]{2023/ST/61/figs/Figure_14.png}
\caption{Cdf of $Y_n$}
\label{fig:2023/ST/61/7}
\end{figure}
\begin{figure}[ht]
\includegraphics[width=\columnwidth]{2023/ST/61/figs/Figure_15.png}
\caption{Cdf of $Z_n$}
\label{fig:2023/ST/61/8}
\end{figure}

\item Let $X_1$, $X_2$, ... , $X_n$ be a random sample of size $n$ from a population having probability density function
\begin{align}
p_X(x; \mu) =
\begin{cases}
e^{-(x-\mu)}, & \text{if } \mu \leq x < \infty \\
0, & \text{otherwise,} 
\end{cases}
\end{align}
where $\mu \in \mathbb{R}$ is an unknown parameter. If $\hat{M}$ is the maximum likelihood estimator of the median of $X_1$, then which one of the following statements is true?
\begin{enumerate}[label=\Alph*)]
  \item $\pr{\hat{M} \leq 2}$ = $1 - e^{-n(1-\log_e 2)}$ if $\mu = 1$
  \item $\pr{\hat{M} \leq 1}$ = $1 - e^{-n \log_e 2}$ if $\mu = 1$
  \item $\pr{\hat{M} \leq 3}$ = $1 - e^{-n(1-\log_e 2)}$ if $\mu = 1$
  \item $\pr{\hat{M} \leq 4}$ = $1 - e^{-n(2\log_e 2-1)}$ if $\mu = 1$
\end{enumerate}
\hfill(GATE ST 2023)\\
\documentclass[]{article}
\usepackage{amsfonts, amssymb}
\usepackage{amsmath}
\usepackage{amsthm}
\usepackage{mathtools}
\usepackage{algorithmic}
\usepackage{float}
\usepackage{graphicx}
\usepackage{enumitem}

\newtheorem{theorem}{Theorem}[section]
\newtheorem{problem}{Problem}
\newtheorem{proposition}{Proposition}[section]
\newtheorem{lemma}{Lemma}[section]
\newtheorem{corollary}[theorem]{Corollary}
\newtheorem{example}{Example}[section]
\newtheorem{definition}[problem]{Definition}
%\newtheorem{thm}{Theorem}[section] 
%\newtheorem{defn}[thm]{Definition}
%\newtheorem{algorithm}{Algorithm}[section]
%\newtheorem{cor}{Corollary}
\newcommand{\BEQA}{\begin{eqnarray}}
\newcommand{\EEQA}{\end{eqnarray}}
\newcommand{\define}{\stackrel{\triangle}{=}}
\theoremstyle{remark}
\newtheorem{rem}{Remark}
%\bibliographystyle{ieeetr}

\title{Question 39.2023}
\author{Anupama Kulshreshtha \\ EE22BTECH11009}
\date{}
\begin{document}
\maketitle
\providecommand{\pr}[1]{\ensuremath{\Pr\left(#1\right)}}
\providecommand{\prt}[2]{\ensuremath{p_{#1}^{\left(#2\right)} }}        % own macro for this question
\providecommand{\qfunc}[1]{\ensuremath{Q\left(#1\right)}}
\providecommand{\sbrak}[1]{\ensuremath{{}\left[#1\right]}}
\providecommand{\lsbrak}[1]{\ensuremath{{}\left[#1\right.}}
\providecommand{\rsbrak}[1]{\ensuremath{{}\left.#1\right]}}
\providecommand{\brak}[1]{\ensuremath{\left(#1\right)}}
\providecommand{\lbrak}[1]{\ensuremath{\left(#1\right.}}
\providecommand{\rbrak}[1]{\ensuremath{\left.#1\right)}}
\providecommand{\cbrak}[1]{\ensuremath{\left\{#1\right\}}}
\providecommand{\lcbrak}[1]{\ensuremath{\left\{#1\right.}}
\providecommand{\rcbrak}[1]{\ensuremath{\left.#1\right\}}}
\newcommand{\sgn}{\mathop{\mathrm{sgn}}}
\providecommand{\abs}[1]{\left\vert#1\right\vert}
\providecommand{\res}[1]{\Res\displaylimits_{#1}} 
\providecommand{\norm}[1]{\left\lVert#1\right\rVert}
%\providecommand{\norm}[1]{\lVert#1\rVert}
\providecommand{\mtx}[1]{\mathbf{#1}}
\providecommand{\mean}[1]{E\left[ #1 \right]}
\providecommand{\cond}[2]{#1\middle|#2}
\providecommand{\fourier}{\overset{\mathcal{F}}{ \rightleftharpoons}}
\newenvironment{amatrix}[1]{%
  \left(\begin{array}{@{}*{#1}{c}|c@{}}
}{%
  \end{array}\right)
}
%\providecommand{\hilbert}{\overset{\mathcal{H}}{ \rightleftharpoons}}
%\providecommand{\system}{\overset{\mathcal{H}}{ \longleftrightarrow}}
	%\newcommand{\solution}[2]{\textbf{Solution:}{#1}}
\newcommand{\solution}{\noindent \textbf{Solution: }}
\newcommand{\cosec}{\,\text{cosec}\,}
\providecommand{\dec}[2]{\ensuremath{\overset{#1}{\underset{#2}{\gtrless}}}}
\newcommand{\myvec}[1]{\ensuremath{\begin{pmatrix}#1\end{pmatrix}}}
\newcommand{\mydet}[1]{\ensuremath{\begin{vmatrix}#1\end{vmatrix}}}
\newcommand{\myaugvec}[2]{\ensuremath{\begin{amatrix}{#1}#2\end{amatrix}}}
\providecommand{\rank}{\text{rank}}
\providecommand{\pr}[1]{\ensuremath{\Pr\left(#1\right)}}
\providecommand{\qfunc}[1]{\ensuremath{Q\left(#1\right)}}
	\newcommand*{\permcomb}[4][0mu]{{{}^{#3}\mkern#1#2_{#4}}}
\newcommand*{\perm}[1][-3mu]{\permcomb[#1]{P}}
\newcommand*{\comb}[1][-1mu]{\permcomb[#1]{C}}
\providecommand{\qfunc}[1]{\ensuremath{Q\left(#1\right)}}
\providecommand{\gauss}[2]{\mathcal{N}\ensuremath{\left(#1,#2\right)}}
\providecommand{\diff}[2]{\ensuremath{\frac{d{#1}}{d{#2}}}}
\providecommand{\myceil}[1]{\left \lceil #1 \right \rceil }
\newcommand\figref{Fig.~\ref}
\newcommand\tabref{Table~\ref}
\newcommand{\sinc}{\,\text{sinc}\,}
\newcommand{\rect}{\,\text{rect}\,}
%%
%	%\newcommand{\solution}[2]{\textbf{Solution:}{#1}}
%\newcommand{\solution}{\noindent \textbf{Solution: }}
%\newcommand{\cosec}{\,\text{cosec}\,}
%\numberwithin{equation}{section}
%\numberwithin{equation}{subsection}
%\numberwithin{problem}{section}
%\numberwithin{definition}{section}
%\makeatletter
%\@addtoreset{figure}{problem}
%\makeatother

%\let\StandardTheFigure\thefigure
\let\vec\mathbf

Let $X_1$, $X_2$, ... , $X_n$ be a random sample of size $n$ from a population having probability density function
\begin{align}
p_X(x; \mu) =
\begin{cases}
e^{-(x-\mu)}, & \text{if } \mu \leq x < \infty \\
0, & \text{otherwise,} 
\end{cases}
\end{align}
where $\mu \in \mathbb{R}$ is an unknown parameter. If $\hat{M}$ is the maximum likelihood estimator of the median of $X_1$, then which one of the following statements is true?
\begin{enumerate}[label=\Alph*)]
  \item $\pr{\hat{M} \leq 2}$ = $1 - e^{-n(1-\log_e 2)}$ if $\mu = 1$
  \item $\pr{\hat{M} \leq 1}$ = $1 - e^{-n \log_e 2}$ if $\mu = 1$
  \item $\pr{\hat{M} \leq 3}$ = $1 - e^{-n(1-\log_e 2)}$ if $\mu = 1$
  \item $\pr{\hat{M} \leq 4}$ = $1 - e^{-n(2\log_e 2-1)}$ if $\mu = 1$
\end{enumerate}
\solution
For continuous random variable X, median M is such that,
\begin{align}
\pr{X \leq M} &= 0.5
\end{align}
The pdf of X is given by,
\begin{align}
p_X(x) =
\begin{cases}
e^{-(x-\mu)}, & \text{if } \mu \leq x < \infty \\
0, & \text{otherwise,} 
\end{cases}
\label{eq:391}
\end{align}
Hence, cdf is given by
\begin{align}
F_X(x; \mu) &= \int_{\mu}^{x} e^{-(t-\mu)}dt\\ 
&= e^{\mu}[-e^{-x} + e^{-\mu}]\\
&= 1 - e^{-(x-\mu)}
\end{align}
Now,
\begin{align}
F_X(x; \mu) &= 0.5\\
\implies 1 - e^{-(M-\mu)} &= 0.5\\
\implies \hat{M} &= \mu + \ln(2)
\label{eq:39}
\end{align}
\begin{definition}
L, the Maximum Likelihood Estimator of the distribution is given by,
\begin{align}
L &= \prod e^{-(x-\mu)}\\
&= e^{-(\sum x_i-n\mu)}\\
\end{align}
\end{definition}
For the Likelihood function to be maximum,
$\sum x_i-n\mu$ should be minimum
Hence,
\begin{align}
X_i > \mu\\
\implies \sum x_i > n\mu\\
\sum x_i-n\mu > 0\\
\implies \mu &= \frac{\sum x_i}{n}
\end{align}
Given,
\begin{align}
p_X(x) &= e^{-(x-\mu)}\\
\end{align}
$X_i$ follows an exponential distribution.
\begin{definition}
We know that if,
\begin{align}
p_X(x) &= \lambda_ie^{-\lambda_ix}\\
S &= X_1 + X_2 +..... +X_n\\
p_S(n) &= \frac{\lambda^nx^{n-1}e^{-\lambda x}}{(n-1)!}
\end{align}
which is gamma distribution with parameters n and $\lambda$
\end{definition}
Hence, pdf of 
\begin{align}
Y &= \sum_{i=1}^{n} x_i\\
\text{where,}\\
p_X(x) &= e^{-(x-\mu)}
\end{align}
will follow gamma distribution with parameter n and $\lambda = 1$, given by,
\begin{align}
p_Y(x;n,1) &= \frac{x^{n-1}e^{-x}}{(n-1)!}
\end{align}
Hence, cdf is given by,
\begin{align}
F_Y(x;n) &= \int_{1}^{x} \frac{t^{n-1}e^{-t}}{(n-1)!} \,dt \\
&= 1 - \Gamma(n,x)
\end{align}
where $\Gamma(n,x)$ is incomplete gamma function.
Thus,
\begin{align}
\pr{\hat{M} \leq k} &= \pr{\frac{\sum x_i}{n} + \ln(2) \leq k}\\
&= \pr{Y/n + \ln(2) \leq k}\\
&= \pr{Y \leq n(k - \ln(2))}\\
&= F_Y(n(k-\ln(2)))\\
&= 1 - \Gamma(n,n(k-ln2))
\end{align}
It needs a value of n to be computed.
\begin{figure}[htbp]
    \centering
    \includegraphics[width=0.8\textwidth]{figs/fig.png} 
    \caption{Verifying gamma cdf through simulation}
    \label{fig:39/2023}
\end{figure}\\
Steps for simulation in C:
\begin{enumerate}
\item Import the necessary libraries, including 'stdio.h','stdlib.h' and 'math.h'.
\item Write functions for generating exponential distribution, gamma pdf and gamma cdf.
\item In the main function, the exponentials generated using the function are added and stored in variable 'gamma samples' for each sample, to get sum of exponentials.
\item Then the simulated gamma cdf values are calculated using 'gamma cdf' function and the calculated sum of exponentials.
\item The code is then compiled using GCC compiler in the terminal (gcc simulation.c -o simulation -lm), and the results are stored in an output.txt file. (./simulation$>$output.txt)
\item The output file is loaded into the python code with theoretical cdf values, and the final graph between theoretical and simulated cdf is plotted.
\item The simulated and theoretical cdf values match, which verifies the gamma cdf through simulation.
\end{enumerate}
\end{document}

\item Let $X_1, X_2, ..., X_{10}$ be a random sample of size 10 from a population having $\gauss{0}{\theta^2}$ distribution, where $\theta > 0$ is an unknown parameter.
Let $T = \frac{1}{10}\sum^{10}_{i=1}{X_i^2}$. If the mean square error of $cT \brak{c > 0}$ as an estimator of $\theta^2$, is minimized at $c = c_0$, then the value of $c_0$ equals
\begin{enumerate}[label =(\roman*)]
	\item $\frac{5}{6}$ \vspace{2pt}
	\item $\frac{2}{3}$ \vspace{2pt}
	\item $\frac{3}{5}$ \vspace{2pt}
	\item $\frac{1}{2}$ \vspace{2pt}
\end{enumerate}
\hfill{(GATE ST 2023)}\\
\fi
\let\negmedspace\undefined
\let\negthickspace\undefined
\documentclass[journal,12pt,onecolumn]{IEEEtran}
       \def\inputGnumericTable{}                                 %%
\usepackage{cite}
\usepackage{amsmath,amssymb,amsfonts,amsthm}
\usepackage{algorithmic}
\usepackage{graphicx}
\usepackage{textcomp}
\usepackage{xcolor}
\usepackage{txfonts}
\usepackage{listings}
\usepackage{enumitem}
\usepackage{mathtools}
\usepackage{gensymb}
\usepackage[breaklinks=true]{hyperref}
\usepackage{tkz-euclide} % loads  TikZ and tkz-base
\usepackage{listings}
\usepackage{gvv}
%
%\usepackage{setspace}
%\usepackage{gensymb}
%\doublespacing
%\singlespacing

%\usepackage{graphicx}
%\usepackage{amssymb}
%\usepackage{relsize}
%\usepackage[cmex10]{amsmath}
%\usepackage{amsthm}
%\interdisplaylinepenalty=2500
%\savesymbol{iint}
%\usepackage{txfonts}
%\restoresymbol{TXF}{iint}
%\usepackage{wasysym}
%\usepackage{amsthm}
%\usepackage{iithtlc}
%\usepackage{mathrsfs}
%\usepackage{txfonts}
%\usepackage{stfloats}
%\usepackage{bm}
%\usepackage{cite}
%\usepackage{cases}
%\usepackage{subfig}
%\usepackage{xtab}
%\usepackage{longtable}
%\usepackage{multirow}
%\usepackage{algorithm}
%\usepackage{algpseudocode}
%\usepackage{enumitem}
%\usepackage{mathtools}
%\usepackage{tikz}
%\usepackage{circuitikz}
%\usepackage{verbatim}
%\usepackage{tfrupee}
%\usepackage{stmaryrd}
%\usetkzobj{all}
    \usepackage{color}                                            %%
    \usepackage{array}                                            %%
    \usepackage{longtable}                                        %%
    \usepackage{calc}                                             %%
    \usepackage{multirow}                                         %%
    \usepackage{hhline}                                           %%
    \usepackage{ifthen}                                           %%
 %optionally (for landscape tables embedded in another document): %%
    \usepackage{lscape}     
%\usepackage{multicol}
%\usepackage{chngcntr}
%\usepackage{enumerate}

%\usepackage{wasysym}
%\documentclass[conference]{IEEEtran}
%\IEEEoverridecommandlockouts
% The preceding line is only needed to identify funding in the first footnote. If that is unneeded, please comment it out.

\newtheorem{theorem}{Theorem}[section]
\newtheorem{problem}{Problem}
\newtheorem{proposition}{Proposition}[section]
\newtheorem{lemma}{Lemma}[section]
\newtheorem{corollary}[theorem]{Corollary}
\newtheorem{example}{Example}[section]
\newtheorem{definition}[problem]{Definition}
%\newtheorem{thm}{Theorem}[section] 
%\newtheorem{defn}[thm]{Definition}
%\newtheorem{algorithm}{Algorithm}[section]
%\newtheorem{cor}{Corollary}
\newcommand{\BEQA}{\begin{eqnarray}}
\newcommand{\EEQA}{\end{eqnarray}}
\newcommand{\define}{\stackrel{\triangle}{=}}
\theoremstyle{remark}
\newtheorem{rem}{Remark}

%\bibliographystyle{ieeetr}
\begin{document}
%

\bibliographystyle{IEEEtran}


\vspace{3cm}

\title{Solution to GATE ST 2023.40}
\author{Devansh Jain - EE22BTECH11018}
%\title{
%	\logo{Matrix Analysis through Octave}{\begin{center}\includegraphics[scale=.24]{tlc}\end{center}}{}{HAMDSP}
%}


% paper title
% can use linebreaks \\ within to get better formatting as desired
%\title{Matrix Analysis through Octave}
%
%
% author names and IEEE memberships
% note positions of commas and nonbreaking spaces ( ~ ) LaTeX will not break
% a structure at a ~ so this keeps an author's name from being broken across
% two lines.
% use \thanks{} to gain access to the first footnote area
% a separate \thanks must be used for each paragraph as LaTeX2e's \thanks
% was not built to handle multiple paragraphs
%

%\author{<-this % stops a space
%\thanks{}}
%}
% note the % following the last \IEEEmembership and also \thanks - 
% these prevent an unwanted space from occurring between the last author name
% and the end of the author line. i.e., if you had this:
% 
% \author{....lastname \thanks{...} \thanks{...} }
%                     ^------------^------------^----Do not want these spaces!
%
% a space would be appended to the last name and could cause every name on that
% line to be shifted left slightly. This is one of those "LaTeX things". For
% instance, "\textbf{A} \textbf{B}" will typeset as "A B" not "AB". To get
% "AB" then you have to do: "\textbf{A}\textbf{B}"
% \thanks is no different in this regard, so shield the last } of each \thanks
% that ends a line with a % and do not let a space in before the next \thanks.
% Spaces after \IEEEmembership other than the last one are OK (and needed) as
% you are supposed to have spaces between the names. For what it is worth,
% this is a minor point as most people would not even notice if the said evil
% space somehow managed to creep in.



% The paper headers
%\markboth{Journal of \LaTeX\ Class Files,~Vol.~6, No.~1, January~2007}%
%{Shell \MakeLowercase{\textit{et al.}}: Bare Demo of IEEEtran.cls for Journals}
% The only time the second header will appear is for the odd numbered pages
% after the title page when using the twoside option.
% 
% *** Note that you probably will NOT want to include the author's ***
% *** name in the headers of peer review papers.                   ***
% You can use \ifCLASSOPTIONpeerreview for conditional compilation here if
% you desire.




% If you want to put a publisher's ID mark on the page you can do it like
% this:
%\IEEEpubid{0000--0000/00\$00.00~\copyright~2007 IEEE}
% Remember, if you use this you must call \IEEEpubidadjcol in the second
% column for its text to clear the IEEEpubid mark.



% make the title area
\maketitle



%\tableofcontents

\bigskip

\renewcommand{\thefigure}{\theenumi}
\renewcommand{\thetable}{\theenumi}
%\renewcommand{\theequation}{\theenumi}

%\begin{abstract}
%%\boldmath
%In this letter, an algorithm for evaluating the exact analytical bit error rate  (BER)  for the piecewise linear (PL) combiner for  multiple relays is presented. Previous results were available only for upto three relays. The algorithm is unique in the sense that  the actual mathematical expressions, that are prohibitively large, need not be explicitly obtained. The diversity gain due to multiple relays is shown through plots of the analytical BER, well supported by simulations. 
%
%\end{abstract}
% IEEEtran.cls defaults to using nonbold math in the Abstract.
% This preserves the distinction between vectors and scalars. However,
% if the journal you are submitting to favors bold math in the abstract,
% then you can use LaTeX's standard command \boldmath at the very start
% of the abstract to achieve this. Many IEEE journals frown on math
% in the abstract anyway.

% Note that keywords are not normally used for peerreview papers.
%\begin{IEEEkeywords}
%Cooperative diversity, decode and forward, piecewise linear
%\end{IEEEkeywords}



% For peer review papers, you can put extra information on the cover
% page as needed:
% \ifCLASSOPTIONpeerreview
% \begin{center} \bfseries EDICS Category: 3-BBND \end{center}
% \fi
%
% For peerreview papers, this IEEEtran command inserts a page break and
% creates the second title. It will be ignored for other modes.
%\IEEEpeerreviewmaketitle
Question:
Let $X_1, X_2, ..., X_{10}$ be a random sample of size 10 from a population having $\gauss{0}{\theta^2}$ distribution, where $\theta > 0$ is an unknown parameter.
Let $T = \frac{1}{10}\sum^{10}_{i=1}{X_i^2}$. If the mean square error of $cT \brak{c > 0}$ as an estimator of $\theta^2$, is minimized at $c = c_0$, then the value of $c_0$ equals
\begin{enumerate}[label =(\roman*)]
	\item $\frac{5}{6}$ \vspace{2pt}
	\item $\frac{2}{3}$ \vspace{2pt}
	\item $\frac{3}{5}$ \vspace{2pt}
	\item $\frac{1}{2}$ \vspace{2pt}
\end{enumerate}
\hfill{(GATE ST 2023)}
\\
\fi
\solution
The mean of $T$ is given by,
\begin{align}
	E\brak{T} &= E\brak{\frac{1}{10}\sum^{10}_{i=1}{X_i^2}}\\
	&= \frac{1}{10}\sum^{10}_{i=1}{E\brak{X_i^2}}
\end{align}
Since
\begin{align}
	E\brak{X^2} = V\brak{X} + \brak{E\brak{X}}^2 \label{eq:gate/2023/st/40/1}
\end{align}
Using \eqref{eq:gate/2023/st/40/1}
\begin{align}
	E\brak{T} &= \frac{1}{10}\sum^{10}_{i=1}{V\brak{X_i} + E\brak{X_i}^2}\\
	&= \frac{1}{10}\brak{10\theta^2}\\
	&= \theta^2
\end{align}
Similarly, using \eqref{eq:gate/2023/st/40/1}, the variance of $T$ is given by,
\begin{align}
	V\brak{T} &= E\brak{T^2} - E\brak{T}^2\\
	&= \frac{1}{100}\brak{E\brak{\brak{\sum^{10}_{i=1}{X_i^2}}^2} - \brak{E\brak{\sum^{10}_{i=1}{X_i^2}}}^2}
\end{align}
But
\begin{align}
    \text{E}\brak{\brak{\sum_{i=1}^{10} X_i^2}^2} &= \text{E}\brak{\sum_{i=1}^{10} \sum_{j=1}^{10} X_i^2X_j^2}\\
    &= \sum_{i=1}^{n} \sum_{j=1}^{n} \text{E}\brak{X_i^2X_j^2} \label{eq:gate/2023/st/40/2}
\end{align}
and 
\begin{align}
    \brak{\text{E}\brak{\sum_{i=1}^{10} X_i}}^2 &= \brak{\sum_{i=1}^{10} \text{E}\brak{X_i}}^2\\
    &= \sum_{i=1}^{10} \sum_{j=1}^{10} \text{E}\brak{X_i} \text{E}\brak{X_j} \label{eq:gate/2023/st/40/3}
\end{align}
Using \eqref{eq:gate/2023/st/40/2} , \eqref{eq:gate/2023/st/40/3}, and the definition of covariance,
\begin{align}
	V\brak{T} &= \frac{1}{100} \brak{\sum_{i=1}^{10} \sum_{j=1}^{10} \brak{E\sbrak{X_i^2X_j^2} - E\brak{X_i^2} E\brak{X_j^2}}}\\
    &= \frac{1}{100} \brak{\sum_{i=1}^{10} \sum_{j=1}^{10} \text{cov}\brak{X_i^2, X_j^2}}
\end{align}
As all the variables are i.i.d's and are thus uncorrelated,
\begin{align}
    \text{cov}\brak{X_i^2, X_j^2} = 
    \begin{cases}
        0 & \text{if } i \ne j\\
        V\brak{X_i^2} & \text{if } i = j
    \end{cases}
\end{align}
Now,
\begin{align}
	V\brak{T} &= \frac{1}{100}\sum^{10}_{i=1}{\text{cov}\brak{X_i^2,X_i^2}}\\
	&= \frac{1}{100}\sum^{10}_{i=1}{V\brak{X^2_i}}
\end{align}
Using \eqref{eq:gate/2023/st/40/1} to find $V\brak{X_i^2}$,
\begin{align}
	V\brak{X^2_i} &= E\brak{X^4_i} - E\brak{X^2_i}^2
\end{align}
Using moment generating function to find $E\brak{X_i^4}$ and $E\brak{X_i^2}$,
\begin{align}
	M_X\brak{t} &= E\brak{e^{tX}}\\
	&= e^{\frac{1}{2}\theta^2t^2}
\end{align}
Differentiating it with respect to $t$,
\begin{align}
	\frac{dM_X\brak{t}}{dt} &= E\brak{Xe^{tX}}\\
	\frac{d^nM_X\brak{t}}{dt^n} &= E\brak{X^ne^{tX}}\\
	\frac{dM_X\brak{t}}{dt} &= \theta^2te^{\frac{1}{2}\theta^2t^2}\\
	\frac{d^2M_X\brak{t}}{dt^2} &= \theta^2e^{\frac{1}{2}\theta^2t^2} + \theta^4t^2e^{\frac{1}{2}\theta^2t^2}\\
	\frac{d^3M_X\brak{t}}{dt^3} &= \theta^4te^{\frac{1}{2}\theta^2t^2} + 2\theta^4te^{\frac{1}{2}\theta^2t^2} + \theta^6t^3e^{\frac{1}{2}\theta^2t^2}\\
	&= 3\theta^4te^{\frac{1}{2}\theta^2t^2} + \theta^6t^3e^{\frac{1}{2}\theta^2t^2}\\
	\frac{d^4M_X\brak{t}}{dt^4} &= 3\theta^4e^{\frac{1}{2}\theta^2t^2} + 3\theta^6t^2e^{\frac{1}{2}\theta^2t^2} + 3\theta^6t^2e^{\frac{1}{2}\theta^2t^2} + \theta^8t^4e^{\frac{1}{2}\theta^2t^2}\\
	&= 3\theta^4e^{\frac{1}{2}\theta^2t^2} + 6\theta^6t^2e^{\frac{1}{2}\theta^2t^2} + \theta^8t^4e^{\frac{1}{2}\theta^2t^2}	
\end{align}
Now,
\begin{align}
	E\brak{X^4} &= \frac{d^4M_X\brak{t}}{dt^4}|_{t=0}\\
	&= 3\theta^4\\
	E\brak{X^2} &= \frac{d^2M_X\brak{t}}{dt^2}|_{t=0}\\
	&= \theta^2
\end{align}
\begin{align}
	V\brak{X^2_i} &= 3\theta^4 - \brak{\theta^2}^2\\
	&= 2\theta^4\\
	V\brak{T} &= \frac{1}{100}\sum^{10}_{i=1}{2\theta^4}\\
	&= \frac{1}{5}\theta^4
\end{align}

The mean square error of $cT$ as an estimator of $\theta^2$ is given by
\begin{align}
	f = E\brak{\brak{cT - \theta^2}^2}
\end{align}
Since
\begin{align}
	E\brak{X^2} = V\brak{X} + \brak{E\brak{X}}^2
\end{align}
Now,
\begin{align}
	f &= V\brak{cT - \theta^2} + \brak{E\brak{cT - \theta^2}^2}\\
	&= V\brak{cT} + \brak{E\brak{cT} - \theta^2}^2\\
	&= c^2V\brak{T} + \brak{cE\brak{T} - \theta^2}^2\\
	&= \frac{c^2}{5}\theta^4 + \theta^4\brak{c-1}^2
\end{align}
Minimizing the mean square error,
\begin{align}
	\frac{df}{dc} &= 0\\
	\frac{df}{dc} &= \frac{2c}{5}\theta^4 + 2\theta^4\brak{c-1}\\
	\frac{c}{5} + c - 1 &= 0\\
	6c &= 5\\
	c &= \frac{5}{6}
\end{align}
To verify that $cT$ is a good estimate for $\theta$, let
\begin{align}
	\theta &= 0.5\\
	E\brak{\brak{cT - \theta^2}^2} &= \frac{1}{5}\brak{\frac{25}{36}}\brak{0.5}^4 + \brak{\frac{-1}{6}}^2\brak{0.5}^4\\
	&= \frac{1}{96}\\
	&= 0.0104
\end{align}
The mean square error is small relative to the value of $\theta$, hence $cT$ is a good estimate.
\\
\begin{figure}[!ht]
	\centering
	\includegraphics[width = \columnwidth]{2023/ST/40/figs/fig1.png}
\end{figure}
\\
Using laplace transform to find CDF,
\begin{align}
	M_T\brak{s} &= M_{\frac{1}{10}\sum^{10}_{i=1}{X^2_i}}\brak{s}\\
	&= \prod^{10}_{i=1}{M_{\frac{X_i^2}{10}}\brak{s}}
\end{align}
Now,
\begin{align}
	M_{\frac{X^2}{10}}\brak{s} &= E\brak{e^{\frac{-sX^2}{10}}}\\
	&= \int^{\infty}_{-\infty}{e^{\frac{-sx^2}{10}}\frac{e^{\frac{-x^2}{2\theta^2}}}{\theta\sqrt{2\pi}}dx}\\
	&= \frac{1}{\theta\sqrt{2\pi}}\int^{\infty}_{-\infty}{e^{\frac{-x^2}{2}\brak{\frac{1}{\theta^2} + \frac{s}{5}}}dx}
\end{align}
Using the substitution,
\begin{align}
	u &= x\sqrt{\frac{1}{\theta^2} + \frac{s}{5}}\\
	du &= \brak{\sqrt{\frac{1}{\theta^2} + \frac{s}{5}} }dx
\end{align}
\begin{align}
	M_{\frac{X^2}{10}}\brak{s} &= \frac{1}{\theta\sqrt{\frac{1}{\theta^2} + \frac{s}{5}}}\int^{\infty}_{-\infty}{\frac{1}{\sqrt{2\pi}}e^{\frac{-u^2}{2}}du}
\end{align}
The integral evaluates to 1 since it is the pdf of a normal distribution,
\begin{align}
	M_{\frac{X^2}{10}}\brak{s} &= \frac{1}{\sqrt{1 + \frac{s\theta^2}{5}}}\\
	M_T\brak{s} &= \frac{1}{\brak{1 + \frac{s\theta^2}{5}}^5}
\end{align}
Taking inverse laplace transform gives us the pdf,
\begin{align}
	p_T\brak{t} &= L^{-1}\sbrak{M_T\brak{s}}\\
	&= L^{-1}\sbrak{\frac{1}{\brak{1 + \frac{s\theta^2}{5}}^5}}\\
	&= \frac{3125 t^4e^{\frac{-5t}{\theta^2}}}{24 \theta^{10}}u\brak{t}
\end{align}
ROC of laplace transform:
\begin{align}
	Re\brak{s} > \frac{-5}{\theta^2}
\end{align}
CDF of T:
\begin{align}
	F_T\brak{t} &= \int^t_{-\infty}{p_T\brak{t}dt}\\
	&= \int^t_{-\infty}{\frac{3125 t^4e^{\frac{-5t}{\theta^2}}}{24 \theta^{10}}u\brak{t}dt}\\
	&= \int^t_0{\frac{3125 t^4e^{\frac{-5t}{\theta^2}}}{24 \theta^{10}}dt}
\end{align}
Applying integration by parts to solve the integral,
\begin{align}
	F_T\brak{t} &= \frac{3125}{24\theta^{10}}\brak{\frac{-\theta^2}{5}t^4e^{\frac{-5t}{\theta^2}} + \int^t_0{\frac{4\theta^2}{5}t^3e^{\frac{-5t}{\theta^2}}dt}}\\
	&= \frac{3125}{24\theta^{10}}\brak{-e^{\frac{-5t}{\theta^2}}\brak{\frac{\theta^2}{5}t^4 + \frac{4\theta^4}{25}t^3} + \int^t_0{\frac{12\theta^4}{25}t^2e^{\frac{-5t}{\theta^2}}dt}}\\
	&= \frac{3125}{24\theta^{10}}\brak{-e^{\frac{-5t}{\theta^2}}\brak{\frac{\theta^2}{5}t^4 + \frac{4\theta^4}{25}t^3 + \frac{12\theta^6}{125}t^2}+\int^t_0{\frac{24\theta^6}{125}te^{\frac{-5t}{\theta^2}}dt}}\\
	&= \frac{3125}{24\theta^{10}}\brak{-e^{\frac{-5t}{\theta^2}}\brak{\frac{\theta^2}{5}t^4 + \frac{4\theta^4}{25}t^3 + \frac{12\theta^6}{125}t^2 + \frac{24\theta^8}{625}t} + \int^t_0{\frac{24\theta^8}{625}e^{\frac{-5t}{\theta^2}}dt}}\\
	&= \frac{3125}{24\theta^{10}}\brak{-e^{\frac{-5t}{\theta^2}}\brak{\frac{\theta^2}{5}t^4 + \frac{4\theta^4}{25}t^3 + \frac{12\theta^6}{125}t^2 + \frac{24\theta^8}{625}t + \frac{24\theta^{10}}{3125}} + \frac{24\theta^{10}}{3125}}\\
	&= 1 - \frac{e^{\frac{-5t}{\theta^2}}}{24\theta^8}\brak{625t^4 + 500\theta^2t^3 + 300\theta^4t^2 + 120\theta^6t + 24\theta^8}
\end{align}
Let $\theta = 0.5$ for simulation,
\begin{align}
	F_T\brak{t} &= 1 - \frac{e^{-20t}}{3}\brak{20000t^4 + 4000t^3 + 600t^2 + 60t + 3}
\end{align}
\\
\begin{figure}[!ht]
	\centering
	\includegraphics[width = \columnwidth]{2023/ST/40/figs/cdf.png}
	\caption{Theoretical CDF vs Simulation CDF}
\end{figure}
\\
Simulation procedure:
\begin{enumerate}[label = (\roman*)]
	\item \begin{align}
	u_1 &= (double) \frac{rand()}{RAND\_MAX} \vspace{2pt} \\
	u_2 &= (double) \frac{rand()}{RAND\_MAX}
	\end{align}
	Generates a uniform distribution between 0 and 1.
	\item \begin{align}
	 X_i &= \sqrt{\theta^2}\brak{\sqrt{-2\log{u_1}} \cos{2\pi u_2}} + \mu
	 \end{align}
	 Transforms the uniform distribution into gaussian distribution of desired mean and variance. Ten such random variables are generated.
	\item \begin{align}
	T &= \frac{1}{10}\sum^{10}_{i=1}{X_i^2}
	\end{align} 
	The values of the random variables are squared and then averaged together to generate $T$.
	\item The value of $c$ which minimizes the mean square error is found by calculating $E\brak{\brak{cT - \theta^2}^2}$ for a range of values of $c$ and choosing that value of $c$ which gives the minimum value for the expression.
	\item The cdf is simulated by counting the number of samples less than a certain $t$ and dividing it by the total number of samples. This gives the CDF of $T$ at $t$. 
\end{enumerate}


\end{enumerate}
