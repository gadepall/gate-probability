\iffalse 
 \documentclass[journal,12pt,onecolumn]{IEEEtran}
\usepackage{setspace}
\usepackage{gensymb}
\singlespacing
\usepackage[cmex10]{amsmath}
\usepackage{amsthm}
\usepackage{mathrsfs}
\usepackage{txfonts}
\usepackage{stfloats}
\usepackage{bm}
\usepackage{cite}
\usepackage{cases}
\usepackage{subfig}
\usepackage{longtable}
\usepackage{multirow}
\usepackage{enumitem}
\usepackage{mathtools}
\usepackage{tikz}
\usepackage{circuitikz}
\usepackage{verbatim}
\usepackage[breaklinks=true]{hyperref}
\usepackage{tkz-euclide} % loads  TikZ and tkz-base
\usepackage{listings}
\usepackage{color}    
\usepackage{array}    
\usepackage{longtable}
\usepackage{calc}     

\usepackage{hhline}   
\usepackage{ifthen}   
\usepackage{lscape}     
\usepackage{chngcntr}
\usepackage{float}
\DeclareMathOperator*{\Res}{Res}
\renewcommand\thesection{\arabic{section}}
\renewcommand\thesubsection{\thesection.\arabic{subsection}}
\renewcommand\thesubsubsection{\thesubsection.\arabic{subsubsection}}

\renewcommand\thesectiondis{\arabic{section}}
\renewcommand\thesubsectiondis{\thesectiondis.\arabic{subsection}}
\renewcommand\thesubsubsectiondis{\thesubsectiondis.\arabic{subsubsection}}
\renewcommand\thetable{\arabic{table}}
% correct bad hyphenation here
\hyphenation{op-tical net-works semi-conduc-tor}
\def\inputGnumericTable{}                                 %%

\lstset{
%language=C,
frame=single, 
breaklines=true,
columns=fullflexible
}
%\lstset{
%language=tex,
%frame=single, 
%breaklines=true
%}

\begin{document}
\newtheorem{theorem}{Theorem}[section]
\newtheorem{problem}{Problem}
\newtheorem{proposition}{Proposition}[section]
\newtheorem{lemma}{Lemma}[section]
\newtheorem{corollary}[theorem]{Corollary}
\newtheorem{example}{Example}[section]
\newtheorem{definition}[problem]{Definition}
\newcommand{\BEQA}{\begin{eqnarray}}
\newcommand{\EEQA}{\end{eqnarray}}
\newcommand{\define}{\stackrel{\triangle}{=}}
\bibliographystyle{IEEEtran}
\providecommand{\mbf}{\mathbf}
\providecommand{\pr}[1]{\ensuremath{\Pr\left(#1\right)}}
\providecommand{\qfunc}[1]{\ensuremath{Q\left(#1\right)}}
\providecommand{\sbrak}[1]{\ensuremath{{}\left[#1\right]}}
\providecommand{\lsbrak}[1]{\ensuremath{{}\left[#1\right.}}
\providecommand{\rsbrak}[1]{\ensuremath{{}\left.#1\right]}}
\providecommand{\brak}[1]{\ensuremath{\left(#1\right)}}
\providecommand{\lbrak}[1]{\ensuremath{\left(#1\right.}}
\providecommand{\rbrak}[1]{\ensuremath{\left.#1\right)}}
\providecommand{\cbrak}[1]{\ensuremath{\left\{#1\right\}}}
\providecommand{\lcbrak}[1]{\ensuremath{\left\{#1\right.}}
\providecommand{\rcbrak}[1]{\ensuremath{\left.#1\right\}}}
\theoremstyle{remark}
\newtheorem{rem}{Remark}
\newcommand{\sgn}{\mathop{\mathrm{sgn}}}
\providecommand{\abs}[1]{\left\vert#1\right\vert}
\providecommand{\res}[1]{\Res\displaylimits_{#1}} 
\providecommand{\norm}[1]{\left\lVert#1\right\rVert}
\providecommand{\mtx}[1]{\mathbf{#1}}
\providecommand{\mean}[1]{E\left[ #1 \right]}
\providecommand{\fourier}{\overset{\mathcal{F}}{ \rightleftharpoons}}
\providecommand{\system}[1]{\overset{\mathcal{#1}}{ \longleftrightarrow}}
\newcommand{\solution}{\noindent \textbf{Solution: }}
\newcommand{\cosec}{\,\text{cosec}\,}
\providecommand{\dec}[2]{\ensuremath{\overset{#1}{\underset{#2}{\gtrless}}}}
\newcommand{\myvec}[1]{\ensuremath{\begin{pmatrix}#1\end{pmatrix}}}
\newcommand{\mydet}[1]{\ensuremath{\begin{vmatrix}#1\end{vmatrix}}}
\let\vec\mathbf
\def\putbox#1#2#3{\makebox[0in][l]{\makebox[#1][l]{}\raisebox{\baselineskip}[0in][0in]{\raisebox{#2}[0in][0in]{#3}}}}
     \def\rightbox#1{\makebox[0in][r]{#1}}
     \def\centbox#1{\makebox[0in]{#1}}
     \def\topbox#1{\raisebox{-\baselineskip}[0in][0in]{#1}}
     \def\midbox#1{\raisebox{-0.5\baselineskip}[0in][0in]{#1}}

\vspace{3cm}
\title{}
\author{EE22BTECH11049 - Shivansh Kirar}
\maketitle
\textbf{Question EC 31 2023}\\
The signal-to-noise ratio (SNR) of an ADC with a full-scale sinusoidal input is given to be 61.96 dB. The resolution of the ADC is \hfill{GATE EC 2023}
\fi
\solution
\begin{table}[H]
\input{2023/EC/31/Table/a.tex}
\label{table:Gate.31.2023.0}
\end{table}
\begin{figure}[H]
  \centering
  \includegraphics[width=1\textwidth]{2023/EC/31/Graph/a.png}
  \caption{Quantization of Sinusoidal Signal}
  \label{fig:Gate.31.2023.1}
\end{figure}
\enumerate
\item Signal Power: \\
The power of a continuous-time signal is defined as the average value of the square of the signal over a certain time interval. For a sinusoidal signal \(x(t) = A \sin(2\pi f t + \phi)\), the power (\(P_S\)) is calculated as:

\begin{align}
P_S &= \lim_{{T \to \infty}} \frac{1}{T} \int_{{-\frac{T}{2}}}^{{\frac{T}{2}}} |x(t)|^2 dt
\end{align}

where \(\phi\) is the phase of the signal.
\begin{align}
P_S &= \lim_{{T \to \infty}} \frac{1}{T} \int_{{-\frac{T}{2}}}^{{\frac{T}{2}}} |A \sin(2\pi f t + \phi)|^2 dt\\
P_S &= \lim_{{T \to \infty}} \frac{1}{T} \int_{{-\frac{T}{2}}}^{{\frac{T}{2}}} A^2 \sin^2(2\pi f t + \phi) dt\\
P_S &= \lim_{{T \to \infty}} \frac{1}{T} \int_{{-\frac{T}{2}}}^{{\frac{T}{2}}} A^2 \cdot \frac{1 - \cos(4\pi f t + 2\phi)}{2} dt\\
P_S &= \frac{1}{2} \lim_{{T \to \infty}} \frac{1}{T} \int_{{-\frac{T}{2}}}^{{\frac{T}{2}}} A^2 dt - \frac{1}{2} \lim_{{T \to \infty}} \frac{1}{T} \int_{{-\frac{T}{2}}}^{{\frac{T}{2}}} A^2 \cos(4\pi f t + 2\phi) dt\\
P_S &= \frac{1}{2} \cdot A^2 - 0\\
P_S &= \frac{A^2}{2}\\
\end{align}

Here, \($A$ = 1\), so:

\begin{align}
P_S &= \frac{1}{2}
\end{align}
\item Noise Power:\\
\text{No of Quantization levels is given by $2^{n}$, Where n is resolution or no of bits. }\\
\text{Distance between any two Quantization levels} = Quantization step \\
(\text{No of Quantization levels})*(\text{Quantization step}) = \text{Peak Distance }   (\text{Refer to Fig 1})\\
\begin{align}
q = \frac{\text{Peak distance}}{2^{n}} = \frac{1-(-1)}{2^{n}} = \frac{2}{2^{n}} = 2^{-(n-1)} \label{eq:gate:1}
\end{align}
We know,quantization error has a maximum value of plus or minus half the
step size, so 
\begin{align}
\left| Y \right| \leq \frac{q}{2} \text{ and therefore, } \left| Y \right| \leq 2^{-n}
\end{align}
For a large enough number of quantization steps, the probability density function of the quantization error tends toward being flat 1.
Pdf of error (Y) of quantization is defined as 
\begin{align}
p_Y(y) = 
\begin{cases}
    \frac{1}{q}, & \text{if } -\frac{q}{2} \leq y \leq \frac{q}{2} \\
    0, & \text{otherwise}
\end{cases}
\end{align}
\begin{figure}[H]
  \centering
  \includegraphics[width=0.7\columnwidth]{2023/EC/31/Graph/b.png}
  \caption{plot of pdf of Quantization Error}
  \label{fig:Gate.31.2023.2}
\end{figure}

So, we can calculate its mean power or variance as the 2nd moment of its distribution.\\
Since, the distribution of error is uniform hence E[Y]=0.\\
\begin{align}
E\brak{Y} &= \int_{-\frac{q}{2}}^{\frac{q}{2}} p_Y(y) y \, de \\
&= \frac{1}{q} \cdot \frac{1}{2} \left(\left(\frac{q}{2}\right)^{2} - \left(-\frac{q}{2}\right)^{2}\right) \\
&= \frac{1}{q} \cdot \frac{1}{2} \left(0 - 0\right)\\
&= 0
\end{align}
\begin{align} 
E[Y^{2}] &= \int_{-\frac{q}{2}}^{\frac{q}{2}} p_Y(y) y^{2} \, de \\
&= \frac{1}{q} \cdot \frac{1}{3} \left(\left(\frac{q}{2}\right)^{3} - \left(-\frac{q}{2}\right)^{3}\right)  \\
&=\frac{1}{3q} \cdot \frac{q^{3}}{4}\\
&=\frac{q^{2}}{12}\\
\end{align}

On putting \(q = 2^{-(n-1)}\), we have:
\begin{align}
E[Y^{2}] &\approx \frac{2^{-2n}}{3}  \\
P_Y &\approx \frac{2^{-2n}}{3}\\
P_N &\approx \frac{2^{-2n}}{3}\\
\end{align}
Thus, an ideal ADC would have a signal-to-noise ratio
\begin{align}
SNR = \frac{P_S}{P_N} = 1.5 \cdot 2^{2n}
\end{align}

or, expressed in decibels,
\begin{align}
SNR &= 10 \left(\log_{10}(1.5 \cdot 2^{n})\right)\\
&= 10 \left(\log_{10}(1.5) + \log_{10}(2^{2n})\right)\\
&= 10 \left(0.176 + 2n \cdot 0.3010\right)\\
&= 1.76 + 6.02n
\end{align}

Substituting value of $SNR$
\begin{align}
61.96 &= 1.76 +6.02n \\
n &= \frac{61.96 -1.76}{6.02}\\
&= \frac{60.2}{6.02}\\
&= 10
\end{align}

So, the resolution of the ADC is approximately 10 bits.\\

Simulation Steps :- \\
\begin{enumerate}
\item Initialize the simulation parameters, including the given SNR in dB (which is 61.96 dB) and signal power ($P_s$) (which is 0.5), and create an array of quantization error values ($q$-values) to test.\\
\item Generate random variables uniformly distributed over $-\frac{q}{2}$ to $\frac{q}{2}$. For each quantization error value, calculate mean error, variance error, and average power (which is equivalent to noise power), and then use Noise Power obtained and Signal Power to calculate the SNR. \\
\item The number of bits required for quantization is calculated as $1 - \log_2(q)$.\\
\item Compare the calculated SNR to the target SNR\_dB, and when it's within a specified tolerance, print the corresponding number of bits required for quantization.\\
\item The program breaks the loop and exits after finding the number of bits that achieves the target SNR\_dB.
\end{enumerate}

