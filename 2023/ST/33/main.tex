\iffalse
\let\negmedspace\undefined
\let\negthickspace\undefined
\documentclass[journal,12pt,twocolumn]{IEEEtran}
\usepackage{cite}
\usepackage{amsmath,amssymb,amsfonts,amsthm,bm}
\usepackage{algorithmic}
\usepackage{graphicx}
\usepackage{textcomp}
\usepackage{xcolor}
\usepackage{txfonts}
\usepackage{listings}
\usepackage{enumitem}
\usepackage{mathtools}
\usepackage{gensymb}
\usepackage[breaklinks=true]{hyperref}
\usepackage{tkz-euclide} % loads  TikZ and tkz-base
\usepackage{listings}
\usepackage{gvv}
\usepackage[latin1]{inputenc}                                 
\usepackage{color}                                            
\usepackage{array}                                            
\usepackage{longtable}                                        
\usepackage{calc}                                             
\usepackage{multirow}                                         
\usepackage{hhline}                                           
\usepackage{ifthen}  
\usepackage{float}                                         
%
%\usepackage{setspace}
%\usepackage{gensymb}
%\doublespacing
%\singlespacing

%\usepackage{graphicx}
%\usepackage{amssymb}
%\usepackage{relsize}
%\usepackage[cmex10]{amsmath}
%\usepackage{amsthm}
%\interdisplaylinepenalty=2500
%\savesymbol{iint}
%\usepackage{txfonts}
%\restoresymbol{TXF}{iint}
%\usepackage{wasysym}
%\usepackage{amsthm}
%\usepackage{iithtlc}
%\usepackage{mathrsfs}
%\usepackage{txfonts}
%\usepackage{stfloats}
%\usepackage{bm}
%\usepackage{cite}
%\usepackage{cases}
%\usepackage{subfig}
%\usepackage{xtab}
%\usepackage{longtable}
%\usepackage{multirow}
%\usepackage{algorithm}
%\usepackage{algpseudocode}
%\usepackage{enumitem}
%\usepackage{mathtools}
%\usepackage{tikz}
%\usepackage{circuitikz}
%\usepackage{verbatim}
%\usepackage{tfrupee}
%\usepackage{stmaryrd}
%\usetkzobj{all}
%    \usepackage{color}                                            %%
%    \usepackage{array}                                            %%
%    \usepackage{longtable}                                        %%
%    \usepackage{calc}                                             %%
%    \usepackage{multirow}                                         %%
%    \usepackage{hhline}                                           %%
%    \usepackage{ifthen}                                           %%
  %optionally (for landscape tables embedded in another document): %%
%    \usepackage{lscape}     
%\usepackage{multicol}
%\usepackage{chngcntr}
%\usepackage{enumerate}

%\usepackage{wasysym}
%\documentclass[conference]{IEEEtran}
%\IEEEoverridecommandlockouts
% The preceding line is only needed to identify funding in the first footnote. If that is unneeded, please comment it out.

\newtheorem{theorem}{Theorem}[section]
\newtheorem{problem}{Problem}
\newtheorem{proposition}{Proposition}[section]
\newtheorem{lemma}{Lemma}[section]
\newtheorem{corollary}[theorem]{Corollary}
\newtheorem{example}{Example}[section]
\newtheorem{definition}[problem]{Definition}
%\newtheorem{thm}{Theorem}[section] 
%\newtheorem{defn}[thm]{Definition}
%\newtheorem{algorithm}{Algorithm}[section]
%\newtheorem{cor}{Corollary}
\newcommand{\BEQA}{\begin{eqnarray}}
\newcommand{\EEQA}{\end{eqnarray}}
\newcommand{\define}{\stackrel{\triangle}{=}}
\theoremstyle{remark}
\newtheorem{rem}{Remark}

%\bibliographystyle{ieeetr}
\begin{document}
%

\bibliographystyle{IEEEtran}


\vspace{3cm}

\title{
%	\logo{
Question ST 33.2023
%	}
}
\author{Gagan Singla - EE22BTECH11021}	
%\title{
%	\logo{Matrix Analysis through Octave}{\begin{center}\includegraphics[scale=.24]{tlc}\end{center}}{}{HAMDSP}
%}


% paper title
% can use linebreaks \\ within to get better formatting as desired
%\title{Matrix Analysis through Octave}
%
%
% author names and IEEE memberships
% note positions of commas and nonbreaking spaces ( ~ ) LaTeX will not break
% a structure at a ~ so this keeps an author's name from being broken across
% two lines.
% use \thanks{} to gain access to the first footnote area
% a separate \thanks must be used for each paragraph as LaTeX2e's \thanks
% was not built to handle multiple paragraphs
%

%\author{<-this % stops a space
%\thanks{}}
%}
% note the % following the last \IEEEmembership and also \thanks - 
% these prevent an unwanted space from occurring between the last author name
% and the end of the author line. i.e., if you had this:
% 
% \author{....lastname \thanks{...} \thanks{...} }
%                     ^------------^------------^----Do not want these spaces!
%
% a space would be appended to the last name and could cause every name on that
% line to be shifted left slightly. This is one of those "LaTeX things". For
% instance, "\textbf{A} \textbf{B}" will typeset as "A B" not "AB". To get
% "AB" then you have to do: "\textbf{A}\textbf{B}"
% \thanks is no different in this regard, so shield the last } of each \thanks
% that ends a line with a % and do not let a space in before the next \thanks.
% Spaces after \IEEEmembership other than the last one are OK (and needed) as
% you are supposed to have spaces between the names. For what it is worth,
% this is a minor point as most people would not even notice if the said evil
% space somehow managed to creep in.



% The paper headers
%\markboth{Journal of \LaTeX\ Class Files,~Vol.~6, No.~1, January~2007}%
%{Shell \MakeLowercase{\textit{et al.}}: Bare Demo of IEEEtran.cls for Journals}
% The only time the second header will appear is for the odd numbered pages
% after the title page when using the twoside option.
% 
% *** Note that you probably will NOT want to include the author's ***
% *** name in the headers of peer review papers.                   ***
% You can use \ifCLASSOPTIONpeerreview for conditional compilation here if
% you desire.




% If you want to put a publisher's ID mark on the page you can do it like
% this:
%\IEEEpubid{0000--0000/00\$00.00~\copyright~2007 IEEE}
% Remember, if you use this you must call \IEEEpubidadjcol in the second
% column for its text to clear the IEEEpubid mark.



% make the title area
\maketitle

\newpage

%\tableofcontents

\bigskip

\renewcommand{\thefigure}{\theenumi}
\renewcommand{\thetable}{\theenumi}
%\renewcommand{\theequation}{\theenumi}

Question: Let $\{W_t\}_{t \geq 0}$ be a standard Brownian motion. Then $E\brak{\left.{W_4}^2\right | W_2=2}$ in integer equals
\fi
\solution
\begin{table}[H]
\input{/home/quantum_fusion/gate-probability/2023/ST/33/tables/table.tex}
\end{table}
In standard brownian motion, 
\begin{align}
W_i &\sim N\brak{0,i}\\
Cov\brak{W_i,W_j} &= \min\brak{i,j} 
\end{align}
Now, we know that,
\begin{align}
E\brak{\left.Y^2\right | X} = Var\brak{\left.Y\right | X} + \brak{E\brak{\left.Y\right | X}}^2 \label{33.2023.1}
\end{align}
$X$ and $Y$ can be represented as:
\begin{align}
X &= \sigma_XZ_1 + \mu_X\\
Y &= \sigma_Y\brak{{\rho}Z_1 + \sqrt{1-{\rho}^2}Z_2} + \mu_Y
\end{align}
where $Z_1$ and $Z_2$ are normal distributions.
\begin{align}
Z_1, Z_2 \sim N\brak{0,1}
\end{align}
Writing the above equations in matrix form,
\begin{align}
\myvec{X\\Y} = 
\begin{bmatrix}
    \sigma_X & 0 \\
    \sigma_Y\rho & \sigma_Y\sqrt{1-{\rho}^2} \\
\end{bmatrix}
\myvec{Z_1\\Z_2} + \myvec{\mu_X\\\mu_Y}
\end{align}
This can be represented as,
\begin{align}
\vec{x} = A\vec{z} + \bm{\mu}
\end{align}
Taking expectation both sides,
\begin{align}
E\brak{\vec{x}} &= E\brak{A\vec{z} + \bm{\mu}}\\
		&= AE\brak{\vec{z}} + E\brak{\bm{\mu}}\\
		&= \bm{\mu}
\end{align}
We know that covariance matrix for $X$ and $Y$ is given by:
\begin{align}
\sigma_\vec{z} &= E\brak{\brak{\vec{x}-\vec{\mu}}\brak{\vec{x}-\vec{\mu}}^T}\\
	      &= E\brak{\brak{A\vec{z}}\brak{A\vec{z}}^T}\\
	      &= E\brak{A\vec{z}\vec{z}^TA^T}\\
	      &= AE\brak{\vec{z}\vec{z}^T}A^T
\end{align}
Multiplying $\vec{z}$ and $\vec{z}^T$ we get,
\begin{align}
\vec{z}\vec{z}^T &= \myvec{Z_1\\Z_2}\myvec{Z_1&Z_2}\\
		 &= \myvec{{Z_1}^2&Z_1Z_2\\Z_1Z_2&{Z_2}^2}\label{33.2023.2}
\end{align}
We know that, 
\begin{align}
Var\brak{Z_1} &= E\brak{\brak{Z_1-\mu}^2}\\
E\brak{Z_1^2} &= 1
\end{align}
Same goes for $Z_2$ as $Z_1$ and $Z_2$ are both normal distributions.\\ 
Taking expectation both sides in equation \eqref{33.2023.2},
\begin{align}
E\brak{\vec{z}\vec{z}^T} &= \myvec{E\brak{Z_1^2}&E\brak{Z_1Z_2}\\E\brak{Z_1Z_2}&E\brak{Z_2^2}}\\
			 &= \myvec{1&0\\0&1}
\end{align}
Hence,
\begin{align}
\sigma_\vec{z} &= AA^T\\
	      &= 
\begin{bmatrix}
    \sigma_X & 0 \\
    \sigma_Y\rho & \sigma_Y\sqrt{1-{\rho}^2} \\
\end{bmatrix}
\begin{bmatrix}
    \sigma_X & \sigma_Y\rho \\
    0 & \sigma_Y\sqrt{1-{\rho}^2} \\
\end{bmatrix}\\
	      &= 
\begin{bmatrix}
\brak{\sigma_X}^2 & \sigma_X\sigma_Y\rho \\
\sigma_X\sigma_Y\rho & \brak{\sigma_Y}^2 \\
\end{bmatrix}
\end{align}
The Co-Relation Coefficiant is given by:
\begin{align}
\rho = \frac{Cov\brak{X,Y}}{\sqrt{Var\brak{X}Var\brak{Y}}}  
\end{align}
Substituting value of $Z_1$ in $Y$,
\begin{align}
Y &= \sigma_Y\rho\brak{\frac{X - \mu_X}{\sigma_X}} + \sigma_Y\sqrt{1 - {\rho}^2}Z_2 + \mu_Y
\end{align}
This an equation of $Y$ in terms of $X$. All the terms except $Z_2$ are constants. Taking expectation on both sides, 
\begin{align}
E\brak{\left.Y\right | X=x} &= E\brak{\sigma_Y\rho\brak{\frac{x - \mu_X}{\sigma_X}} + \sigma_Y\sqrt{1 - {\rho}^2}Z_2 + \mu_Y}\\
			    &= E\brak{\sigma_Y\rho\brak{\frac{x - \mu_X}{\sigma_X}} + \mu_Y} + E\brak{\sigma_Y\sqrt{1 - {\rho}^2}Z_2}\\
			    &= \mu_Y +\rho\brak{\frac{\sigma_Y}{\sigma_X}}\brak{x - \mu_X} + \sigma_Y\sqrt{1 - {\rho}^2}E\brak{Z_2}
\end{align}
Now for variance,
\begin{align}
Var\brak{\left.Y\right | X=x} &= Var\brak{\sigma_Y\rho\brak{\frac{x - \mu_X}{\sigma_X}} + \sigma_Y\sqrt{1 - {\rho}^2}Z_2 + \mu_Y}\\
			      &= Var\brak{\sigma_Y\rho\brak{\frac{x - \mu_X}{\sigma_X}} + \mu_Y} + Var\brak{\sigma_Y\sqrt{1 - {\rho}^2}Z_2}
\end{align}
Variance of constants terms is 0.
\begin{align}		      
Var\brak{\left.Y\right | X=x} &= Var\brak{\sigma_Y\sqrt{1 - {\rho}^2}Z_2}\\
			      &= {\brak{1 - {\rho}^2}{\sigma_Y}^2}Var\brak{Z_2}
\end{align}
$Z_2$ is a normal distribution so,
\begin{align}
E\brak{Z_2} &= 0\\
Var\brak{Z_2} &= 1
\end{align}
By substituting these values in above equations,
\begin{align}
E\brak{\left.Y\right | X=x} &= \mu_Y +\rho\brak{\frac{\sigma_Y}{\sigma_X}}\brak{x - \mu_X}\\
Var\brak{\left.Y\right | X=x} &= \brak{1 - {\rho}^2}{\sigma_Y}^2
\end{align}
In our case, 
\begin{align}
Y &= W_4\\
X &= W_2\\
x &= 2
\end{align}
Hence, we get that,
\begin{align}
\mu_X = \mu_Y &= 0\\ 
\sigma_X &= \sqrt{2}\\
\sigma_Y &= 2\\
\rho &= \frac{2}{\sqrt{8}}\\
     &= \frac{1}{\sqrt{2}}
\end{align} 
Substituting the values in above equations,
\begin{align}
E\brak{\left.Y\right | X=2} &= \frac{1}{\sqrt{2}}\cdot\frac{2}{\sqrt{2}}\cdot 2\\
			    &= 2\\
Var\brak{\left.Y\right | X=2} &= \brak{1 - \frac{1}{2}}\brak{2}^2\\
			      &= \frac{1}{2}\cdot 4\\
			      &= 2
\end{align}
Substituting these values in \eqref{33.2023.1},
\begin{align}
E\brak{\left.Y^2\right | X=2} &= 2 + \brak{2}^2\\
			      &= 6\\
E\brak{\left.{W_4}^2\right | W_2=2} &= 6
\end{align}
\newline\newline\newline
Steps for Simulation:
\begin{enumerate}
\item Set the number of samples to be generated as 10000
\item Write a function to generate a uniform distribution
\item Write a function to generate a normal distribution by using the uniform distribution function defined above through box muller method
\item Write a function to generate value of $\vec{z}$ from one the column of the dist matrix
\item Write a function to calculate $\vec{x}$ by matrix multiplication: $\vec{x} = A\vec{z} + \vec{u}$.
\item Write a function to store values of each vector $\vec{x}$ obtained in a 2D array of size 2*numsamples
\item Make a temporary 2D array(dist) of size 2x10000 to store the values of $Z_1$ and $Z_2$ obtained after calling the normal distribution function
\item Use a for loop to store the values obtained by calling the normal function multiple times 
\item 
\item Assume values for constants,
\begin{align}
\sigma_X = 0.5\\
\sigma_Y = 0.8\\
\rho = 0.5\\
\mu_X = 1\\
\mu_Y = 1.5
\end{align}
\item Create $A$ matrix and vector $\bm{\mu}$ and fill the values assumed above in these, the dimension for matrix is 2x2 and the vector 2x1
\item Make a 2D array(ans) to store the final values of $\vec{x}$ obtained after matrix multiplication
\item For each iteration, create a $\vec{z}$ assign values from one of the columns from dist to $\vec{z}$ using the function defined above
\item For each iteration, Generate $\vec{x}$  vector and call the function to solve to matrix multiplication and give value for $\vec{x}$  
\item Store the values obtained from matrix multiplication in the ans array using the store function defined above 
\item The ans array has the values of $\vec{x}$ for 10000 simulations.
\end{enumerate}




