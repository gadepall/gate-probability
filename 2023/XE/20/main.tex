\iffalse
\let\negmedspace\undefined
\let\negthickspace\undefined
\documentclass[journal,12pt,twocolumn]{IEEEtran}
\usepackage{cite}
\usepackage{amsmath,amssymb,amsfonts,amsthm}
\usepackage{algorithmic}
\usepackage{graphicx}
\usepackage{textcomp}
\usepackage{xcolor}
\usepackage{txfonts}
\usepackage{listings}
\usepackage{enumitem}
\usepackage{mathtools}
\usepackage{gensymb}
\usepackage{comment}
\usepackage[breaklinks=true]{hyperref}
\usepackage{tkz-euclide} 
\usepackage{listings}
\usepackage{gvv}                                        
\def\inputGnumericTable{}                                 
\usepackage[latin1]{inputenc}                                
\usepackage{color}                                            
\usepackage{array}                                            
\usepackage{longtable}                                       
\usepackage{calc}                                             
\usepackage{multirow}                                         
\usepackage{hhline}                                           
\usepackage{ifthen}                                           
\usepackage{lscape}

\newtheorem{theorem}{Theorem}[section]
\newtheorem{problem}{Problem}
\newtheorem{proposition}{Proposition}[section]
\newtheorem{lemma}{Lemma}[section]
\newtheorem{corollary}[theorem]{Corollary}
\newtheorem{example}{Example}[section]
\newtheorem{definition}[problem]{Definition}
\newcommand{\BEQA}{\begin{eqnarray}}
\newcommand{\EEQA}{\end{eqnarray}}
\newcommand{\define}{\stackrel{\triangle}{=}}
\theoremstyle{remark}
\newtheorem{rem}{Remark}
\begin{document}

\bibliographystyle{IEEEtran}
\vspace{3cm}

\title{ASSIGNMENT}
\author{EE22BTECH11016-Chinthalapudi Yashwanth$^{*}$% <-this % stops a space
}
\maketitle
\newpage
\bigskip
\renewcommand{\thefigure}{\theenumi}
\renewcommand{\thetable}{\theenumi}

\textbf{Question 20.2023}
The probability of a person telling the truth is 4/6 . An unbiased die is thrown by
the same person twice and the person reports that the numbers appeared in both
the throws are same. Then the probability that actually the numbers appeared in
both the throws are same is ?\\
\fi
\solution
Random variables on $i \in \{1,2\} $ defined as
\begin{table}[!ht]
	\input{tables/table.tex}
\end{table}\\
$p_Y(0)$ = Probability that person telling the lie\\
$p_Y(1)$ = Probability that person telling the truth\\
Consider
\begin{align}
Z&=X_1-X_2
\end{align}
\begin{table}[!ht]
	\input{tables/table1.tex}
\end{table}
\begin{align}
p_Y(i) &= \begin{cases}
	    \frac{2}{3} & \text{if } i = 1\\
	    \frac{1}{3} & \text{if } i = 0\\
	    0 & \text{otherwise}\\
          \end{cases}
\end{align}
Given in the question that the person reports that the numbers appeared in both
the throws are same. the probability that actually the numbers appeared in
both the throws are same, that is simply, the probability of person's truth given that numbers on both dice are same that is, $\Pr(Y=1 / Z=0)$.
\begin{align} 
\Pr(Y=1 / Z=0) &= \frac{\Pr((Z=0) . (Y=1))}{\Pr(Z=0)}
\end{align}
Since $X_i$ and $Y$ are independent events
\begin{align}
\Pr(Y=1 / Z=0) &= p_Y(1)\\
&= \frac{2}{3}\\
&\approx 0.667
\end{align}
